\section{Valutazione sugli altri capitolati}
	\subsection{Capitolato C1 - Ajarvis}
		\subsubsection{Descrizione}
	    Lo scopo di questo capitalo denominato \emph{Ajarvis: assistente virtuale di cerimonie Agile} è quello di realizzare un software in grado di registrare lo \gl{standup} mattutino dell'azienda. Nel dettaglio è richiesto di registrare la conversazione, convertirla in testo e analizzarne il contenuto in modo tale da poter realizzare una dashboard che rappresenta mediate grafici le diverse tematiche affrontate durante il discorso.  
		\subsubsection{Studio del dominio}
			\paragraph{Dominio applicativo} \Spazio
			 L'utilità di questo prodotto è sicuramente molto rilevante per quegli ambienti organizzati in team dove è necessario stilare un documento riassuntivo dell'incontro o comunque aver accesso alle informazioni dello standup in modo digitale e categorizzato.
			\paragraph{Dominio tecnologico}
			\begin{itemize}
				\item \textbf{\gl{Node.js}} per interfacciarsi con le piattaforme Google;
				\item  \textbf{\gl{Google Cloud Platform}} necessaria per salvare il \gl{file} audio; 
				\item  \textbf{\gl{Google SQL}} per realizzare la base dei dati che visualizza l'utilizzatore;
				\item  \textbf{\gl{Speech To Text}} per registrare concretamente il file audio;
				\item  \textbf{\gl{Framework} Bootstrap} per realizzare la dashboard nella quale vengono rappresentati i dati dopo esser stati elaborati;
				\item  \textbf{\gl{Natural Language Api}} per effettuare un'analisi preliminare sul testo.	
			\end{itemize}
			
			\paragraph{Aspetti positivi} \Spazio
				\begin{itemize}
				\item {Interessante in particolare per le tecnologie proposte in quanto l'analisi della voce è un tema attuale che in futuro potrà avere interessanti sviluppi;}
				\item {Il pacchetto di tecnologie Google necessarie per lo sviluppo del capitolato è un bagaglio di cultura notevole extra-universitario.}	
			\end{itemize}
			\paragraph{Aspetti negativi} \Spazio
			\begin{itemize}
				\item {Sicuramente le nuove tecnologie da apprendere che probabilmente richiederebbero molto più tempo di quello a disposizione;}
				\item {Poca disponibilità da parte dell'azienda di svolgere incontri frontali con i gruppi fornitori.}	
			\end{itemize} 
			\paragraph{Conclusioni} \Spazio
			Il capitolato proposto ha avuto un giudizio positivo da parte di tutti i membri del gruppo. Non è stato scelto in quanto sono risultate di maggiore interesse altre tecnologie.
			
	\subsection{Capitolato C2 - BlockCV}
	\subsubsection{Descrizione}
	Il progetto dal titolo \emph{BlockCV: blockchain per gestione di CV certificati} prevede la creazione di una piattaforma distribuita per la pubblicazione di Curriculum Vitae con la possibilità di ricercare offerte di lavoro. Il sistema dev'essere basato sulla tecnologia Blockchain.
	Il software deve essere integrabile nell'attuale sistema lavorativo, quindi dalla pubblicazione del CV, all'arricchimento dello stesso grazie a maggiori esperienze acquisite dall'utente e deve dare la possibilità alle varie realtà che hanno possibilità di assunzione di confermare le competenze degli utenti.    
	\subsubsection{Studio del dominio}
	\paragraph{Dominio applicativo} \Spazio
	Il software è utilizzabile nel mercato online e molti utenti potrebbero usufruirne sia per cercare lavoro, sia per avere un CV online sempre accessibile e aggiornato con il tempo con la conferma delle competenze da entità verificate.
	\paragraph{Dominio tecnologico} \Spazio
	\begin{itemize}
		\item \textbf{Blockchain};
		\item \textbf{Hyperledger Fabric};
		\item  \textbf{Java EE};
		\item  \textbf{Play} framework per l'interfaccia grafica;
		\item  \textbf{\gl{MongoDB}} per il database;
		\item  \textbf{\gl{HTML} e \gl{CSS}} per realizzare l'interfaccia utente.
	\end{itemize}
	
	\paragraph{Aspetti positivi} \Spazio
	\begin{itemize}
		\item {Il blockchain è una tecnologia innovativa e molto attuale nell'ultimo periodo anche grazie al collegamento con la \gl{criptovaluta} Bitcoin;}
		\item {Esperienza da parte di diversi membri del team nello sviluppare interfacce web.}
	\end{itemize} 
	\paragraph{Aspetti negativi} \Spazio
	Per alcuni dei componenti del gruppo le tecnologie da utilizzare e alcuni framework imposti come vincolo obbligatorio dall'azienda sono sconosciuti.
	\paragraph{Conclusioni} \Spazio
	Per via degli aspetti precedentemente elencati il gruppo ha preferito concentrarsi su un capitolato diverso.
	
	
	\subsection{Capitolato C3 - DeSpeect}
	\subsubsection{Descrizione}	
		Il progetto \emph{DeSpeect: interfaccia grafica per Speect} prevede la realizzazione di un'interfaccia grafica di una libreria Open Source per lo sviluppo di \gl{frontend} e \gl{backend} di un sistema in sintesi vocale.
		L'esigenza nasce dalla necessità di monitorare il comportamento dei plugin della libreria in questione.
	\subsubsection{Studio del dominio}
	
	\paragraph{Dominio applicativo} \Spazio
     Il capitolato è posizionato nell'ambito della sintesi vocale, tecnologia molto presente nel mercato ordierno. 
	\paragraph{Dominio tecnologico}
	\begin{itemize}
		\item \textbf{Linux} richiesta la compatibilità del software con questo sistema operativo;
		\item  \textbf{Linguaggio C} necessario per integrarsi con la libreria sulla quale si basa il progetto;
		\item  \textbf{Speect} tecnologia madre del progetto.
	\end{itemize}
	\paragraph{Aspetti positivi} \Spazio
	Interessante la sintesi vocale in quanto è una tecnologia innovativa e ancora in fase di sviluppo in molti settori.
	\paragraph{Aspetti negativi} \Spazio 
	\begin{itemize}
		\item L'azienda richiede che venga realizzata un'interfaccia grafica, così facendo si ha poco modo di prendere dimestichezza con la sintesi vocale;
		\item Poca chiarezza nella descrizione del capitolato.
	\end{itemize}
	\paragraph{Conclusioni} \Spazio
	Per i fattori appena elencati, il gruppo si è orientato in un capitolato diverso.
	
	\subsection{Capitolato C4 - ECoRe}
	\subsubsection{Descrizione}
    L'idea del capitolato \emph{ECoRe: enterprise content recommendation} è quella di sviluppare un software in grado di suggerire contenuti che potrebbero risultare efficienti ed interessanti all'utente durante lo svolgimento del suo lavoro. Il sistema, dovrà essere in grado di ottenere queste informazioni da più fonti.
	
	\subsubsection{Studio del dominio} 
	\paragraph{Dominio applicativo} \Spazio
	Il software risulta essere utile per essere di supporto a determinate figure aziendali come per esempio i business consultant o chi si occupa di marketing.
	\paragraph{Dominio tecnologico}
		\begin{itemize}
		\item \textbf{\gl{Apache SolR}}; 
		\item \textbf{Elasticsearch} utilizzato per memorizzare ed elaborare le informazioni;
		\item \textbf{\gl{Apache Nutch}} motore di ricerca per importare informazioni dal web nel software.
	\end{itemize}
	
	\paragraph{Aspetti positivi} \Spazio
	\begin{itemize}
		\item Il progetto e le tecnologie necessarie per lo sviluppo risultano essere innovative;
		\item Il progetto è applicabile in molti ambiti.
	\end{itemize}
	\paragraph{Aspetti negativi} \Spazio
     Molte tecnologie necessarie per portare a compimento il progetto in tempi ragionevoli.

	\paragraph{Conclusioni} \Spazio
	Sebbene il progetto sia stato ritenuto interessante dai membri del team è stato scartato per via della mancanza di esperienza pregressa da parte di tutti i componenti.
	
	\subsection{Capitolato C5 - IronWorks}
		\subsubsection{Descrizione}
		Il capitolato C5 \emph{IronWorks, utilità per la costruzione di software robusto} è proposto dall'azienda Zucchetti S.r.l. e pone come obbiettivo la generazione automatica del codice da diagrammi \gl{UML} per rendere più facile seguire le buone regole di programmazione. In particolare chiede la realizzazione di un'\gl{editor} per la costruzione di diagrammi UML (l'interesse è rivolto verso i soli diagrammi di robustezza) con la relativa generazione di codice Java per le entità persistenti e per i metodi di scrittura e lettura verso un database relazionale. 
		\subsubsection{Studio del dominio}
			\paragraph{Dominio applicativo} \Spazio
			L'applicazione deve essere di aiuto per la progettazione di un buon software: deve permettere la realizzazione di diagrammi UML da cui deve essere possibile la generazione di codice. Potranno essere disegnati diagrammi di robustezza seguendo le regole con cui i tre tipi di oggetti rappresentabili dai diagrammi di robustezza (le interfacce, le procedure e le entità persistenti) possono reagire tra di loro.
			\paragraph{Dominio tecnologico} \Spazio
			E' richiesta la conoscenza delle seguenti tecnologie per la realizzazione dell'applicazione web:
				\begin{itemize}
					\item Per la parte server:
					\begin{itemize}
						\item Java;
						\item \gl{TomCat};
						\item JavaScript;
						\item Node.js.
					\end{itemize}
					\item Per la parte client:
					\begin{itemize}
						\item HTML5;
						\item CSS.
					\end{itemize}
					\item Per l'archiviazione dei dati su file di testo o su database:
					\begin{itemize}
						\item \gl{XML};
						\item JSON;
						\item SQL.
					\end{itemize}
				\end{itemize}
		\subsubsection{Aspetti positivi}
		\begin{itemize}
			\item Il team presenta già conoscenze sulle tecnologie da utilizzare per il lato client dell'applicazione web;
			\item Il \gl{proponente} fornisce dei software di riferimento, alcuni open source.
		\end{itemize}
		\subsubsection{Aspetti negativi}
		\begin{itemize}
			\item I diagrammi di robustezza sono poco conosciuti e poco utilizzati;
			\item Nel mercato sono già presenti molti software con le caratteristiche richieste.
		\end{itemize}
		\subsubsection{Conclusioni}
		Questo capitolato, sebbene non presentasse eccessive criticità, non ha suscitato particolare interesse rispetto ad altri. La limitazione alla rappresentazione dei soli diagrammi di robustezza è stata ritenuta troppo vincolante per la creazione di un prodotto completo e veramente utilizzabile nello sviluppo software.
		
		
	\subsection{Capitolato C6 - Marvin}
		\subsubsection{Descrizione}
		Il capitolato \emph{Marvin, dimostratore di Uniweb su Ethereum}, proposto dall'azienda RedBabel, propone di realizzare una versione di Uniweb come una \gl{Dapp} che giri su \gl{Ethereum Virtual Machine}. \\  Infatti, come su Uniweb, devono poter interagire:
		\begin{itemize} 
			\item studenti, per aver accesso alla propria carriera universitaria, registrarsi ad esami, accettare e rifiutare voti; 
			\item professori, per pubblicare liste di esami, pubblicare voti;
			\item Università, per gestire corsi, orario, spazi, e altro.
		\end{itemize}
	 	Le interazioni tra questi tre attori vengono tradotte con una serie di smart contracs.
		\subsubsection{Studio del dominio}
			\paragraph{Dominio applicativo} \Spazio
			Questo capitolato si pone l'obbiettivo unire le tecnologie ad oggi in forte ascesa con il mondo universitario. Come per Uniweb saranno studenti, professori ed Università ad utilizzare il prodotto risultante per ottimizzare studio e lavoro.
			\paragraph{Dominio tecnologico} \Spazio
			Per comprendere a fondo il dominio e per realizzare il progetto è richiesta la conoscenza delle seguenti tecnologie:
			\begin{itemize}
				\item Ethereum;
				\item \gl{Truffle};
				\item \gl{Etherscan.io};
				\item Javascript;
				\item \gl{ESLint};
				\item \gl{React};
				\item \gl{SCSS}.
			\end{itemize}
		\subsubsection{Aspetti positivi}
		\begin{itemize}
			\item Particolare interesse da parte di vari membri del team verso il dominio del capitolato;
			\item Le tecnologie utilizzate sono sempre più richieste nel mondo del lavoro e una loro conoscenza approfondita gioverebbe ad ognuno dei componenti del gruppo.
		\end{itemize}
		\subsubsection{Aspetti negativi}
		\begin{itemize}
			\item Il dominio applicativo del software è molto vasto e la conoscenze di esso da parte dei membri del team non sono sufficienti;
			\item Può risultare scomodo contattare il proponente dato che ha sede ad Amsterdam.
		\end{itemize}
		\subsubsection{Conclusioni}
		Sebbene le conoscenze acquisite dallo sviluppo di questo capitolato siano direttamente spendibili nel mondo del lavoro, una buona progettazione richiederebbe uno studio approfondito di tutti i suoi campi applicativi, il quale risulterebbe troppo oneroso rispetto al tempo a disposizione.
	
	\subsection{Capitolato C8 - TuTourSelf}
		\subsubsection{Descrizione}
		Il capitolato d'appalto C8 \emph{TuTourSelf, piattaforma di prenotazioni per artisti in tournee} è proposto dall'omonima start-up TuTourSelf S.r.l. che con questo progetto mette in gioco una propria idea.\\ Il prodotto richiesto è un'applicazione web che permetta agli artisti di organizzare il proprio tour mettendosi in contatto personalmente con i locali disponibili, adatti ad ospitare l'evento e che corrispondano alle esigenze dell'artista. 
		\subsubsection{Studio del dominio} 
			L'applicazione web richiesta vuole rendere l’organizzazione delle performance live il più semplice possibile così da aiutare il percorso degli artisti che vogliono rimanere indipendenti e rendere possibile l'ascesa dei giovani talenti che ancora non sono affiancati da agenzie.
			\paragraph{Dominio tecnologico} \Spazio
			Per la realizzazione dell'applicazione web si chiede al gruppo la profonda conoscenza delle seguenti tecnologie:
			\begin{itemize}
				\item HTML5;
				\item CSS;
				\item JavaScript;
				\item React.
			\end{itemize}
		\subsubsection{Aspetti positivi}
		\begin{itemize}
			\item La conoscenza delle tecnologie utilizzate sono molto richieste nel mondo lavorativo;
			\item Interesse nella realizzazione di un'applicazione che funzioni anche su dispositivi mobile.
		\end{itemize}
		\subsubsection{Aspetti negativi}
		\begin{itemize}
			\item Nel mercato sono già presenti software strutturati in modo simile;
			\item Troppa libertà per la realizzazione del lato back-end.
		\end{itemize}
		\subsubsection{Conclusioni}
		Il team possiede già delle basi con le tecnologie richieste e preferisce avere un'esperienza progettuale con delle tecnologie nuove. Inoltre si ritiene più istruttivo lavorare a contatto di un'azienda già affermata nel territorio e nel mondo lavorativo che con una start-up.
	
	