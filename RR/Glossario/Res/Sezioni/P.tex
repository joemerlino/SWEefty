\lettera{P}

\parola{Package}{Nell'Unified Modeling Language è usato per raggruppare elementi e fornire un namespace per gli elementi raggruppati. Può contenere altri package, fornendo così un'organizzazione gerarchica dei package.}  

\parola{Plotly.js}{Libreria grafica JavaScript open source.}  

\parola{Plugin}{Programma non autonomo che interagisce con un altro programma per ampliare o estendere le funzionalità originarie.}  

\parola{PNG (Portable Network Graphics)}{Formato di file grafico bitmap per memorizzare immagini.}  

\parola{Processo}{Insieme di attività collegate tra loro che trasformano ingressi in uscite secondo regole fissate e tramite risorse limitate.}  

\parola{Prodotto}{Indica il risultato di un'attività, sia esso un documento, del codice sorgente o un qualsiasi risultato verificabile che possa essere offerto per soddisfare un bisogno o un’esigenza.}  

\parola{Progettista}{Persona con competenze tecniche e tecnologiche aggiornate e ampia esperienza professionale. Si occupa dello sviluppo della soluzione al problema presentato tramite le attività di progettazione, spesso assumendo anche responsabilità di scelta e gestione.}  

\parola{Progetto}{Insieme di attività e compiti che prevedono il raggiungimento di determinati obiettivi con specifiche fissate. Sono definite date di inizio e fine durante la quale si può predisporre di limitate risorse che vengono consumate nello svolgersi delle attività. }  

\parola{Proponente}{Colui che presenta una proposta di progetto, nel nostro caso l'azienda IKS.}

\parola{Push}{Atto di apportare delle modifiche a file presenti in un repository Git}