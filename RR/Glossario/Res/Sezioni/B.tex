\lettera{B}

\parola{Backend}{Parte dell'architettura non visibile dall'utente, generalmente si occupa di elaborare i dati ricevuti in input dall'utente. }

\parola{Big data}{Termine con la quale ci si riferisce all'insieme delle tecnologie e delle metodologie di analisi di dati massivi. Inoltre il termine denota anche la capacità di estrapolare e mettere in relazione un'enorme mole di dati eterogenei, strutturati o non per scoprirne gli eventuali legami soggiacenti.}

\parola{Blockchain}{Database distribuito composto da una lista, i cui elementi sono collegati tra loro e resi sicuri mediante la crittografia. La caratteristica principale è che riesce a registrare le transazioni tra due parti in modo efficiente, verificabile e permanente.}

\parola{Bot}{Applicazione software che esegue scripts in maniera continua e automatizzata su Internet. Conosciuto anche come web robot o Internet bot.}

\parola{Bug}{In italiano "baco", identifica un errore nella scrittura del codice sorgente che porta a comportamente anomali e non previsti del programmma. Un bug può essere introdotto anche in fase di compilazione o di progettazione del programma.}

\parola{Bugfix}{Porzione di codice progettata per risolvere i malfunzionamenti o i comportamente anomali introdotti da un bug.}