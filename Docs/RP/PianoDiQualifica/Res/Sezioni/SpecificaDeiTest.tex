\section{Specifica dei test}
Per produrre software di qualità, il gruppo SWEefty definirà dei test per assicurarsi che le unità prodotte funzionino in maniera corretta. Il tracciamento dei test ed il loro esito verrà riportato in questo documento.
	\subsection{Tipi di test}
		
	\subsubsection{Test di Unità}
    lo scopo di questa tipologia di test è quello di verificare la più piccola parte di lavoro prodotta da un programmatore. Questo significa tendenzialmente verificare i metodi e le funzioni scritte, questi test verranno implementati durante la progettazione architetturale.
	
	\subsubsection{Test di Integrazione}
	lo scopo di questa tipologia di test è quello di verificare le componenti di sistema. Più	precisamente, l’obiettivo è quello di testare il funzionamento dei vari \gl{package} prodotti, sia singolarmente che nel loro insieme, questi test verranno implementati assieme durante la progettazione architetturale.
		
		
	\subsubsection{Test di Sistema}
	Questo tipo di test serve per verificare che il comportamento dinamico del sistema sia conforme ai requisiti specificati nel documento \textit{Analisi dei Requisiti\_v2.0.0}.
	
		
	\subsubsection{Test di Accettazione}
	Questi test vengono utilizzati durante il collaudo finale in presenza del committente.
	
	
