\section{Specifica dei test}
Per produrre software di qualità, il gruppo SWEefty definirà dei test per assicurarsi che le unità prodotte funzionino in maniera corretta. Il tracciamento dei test ed il loro esito verrà riportato in questo documento.
	\subsection{Tipi di test}
		
	\subsubsection{Test di unità}
    Lo scopo di questa tipologia di test è quello di verificare la più piccola parte di lavoro prodotta da un programmatore. Questo significa tendenzialmente verificare i metodi e le funzioni scritte, questi test verranno implementati durante la progettazione architetturale.
	
	\subsubsection{Test di integrazione}
	Lo scopo di questa tipologia di test è quello di verificare le componenti di sistema. Più	precisamente, l’obiettivo è di testare il funzionamento dei vari \gl{package} prodotti, sia singolarmente che nel loro insieme, questi test verranno implementati assieme durante la progettazione architetturale.
		
		
	\subsubsection{Test di sistema}
	Questo tipo di test serve per verificare che il comportamento dinamico del sistema sia conforme ai requisiti specificati nel documento \textit{Analisi dei Requisiti\_v2.0.0}.
	\newcolumntype{H}{>{\centering\arraybackslash}m{7cm}}
	\normalsize
	\begin{longtable}{|c|H|c|}
		\hline
		\textbf{Id Test} & \textbf{Descrizione} & \textbf{Stato}\\
		\hline
		\endhead
		TS0F1.1&Viene verificato che sia possibile visualizzare le varie componenti del sistema.& Non implementato\\ \hline
		TS0F1.1.1&Vene verificato che sia possibile visualizzare i server nella mappa.&Non implementato \\ \hline
		TS0F1.1.2&Viene verificato che sia possibile visualizzare nella mappa i database dell'applicazione monitorata.&Non implementato \\ \hline
		TS0F1.1.3&Viene verificato che sia possibile visualizzare nella mappa i server cluster dell'applicazione monitorata.&Non implementato \\ \hline
		TS1F1.2&Viene verificato che ogni componente dell'applicazione venga visualizzato in modo differente.&Non implementato \\ \hline
		TS1F1.2.1&Viene verificato che i server siano rappresentati sotto forma di cerchi.&Non implementato \\ \hline
		TS1F1.2.2&Viene verificato che i database siano rappresentati sotto forma di cilindri.&Non implementato \\ \hline
		TS1F1.2.3&Viene verificato che sia possibile visualizzare il numero di server che compongono un cluster.&Non implementato \\ \hline
		TS1F1.3&Viene verificato che sia possibile visualizzare le informazioni riguardanti i componenti della mappa topologica dell'applicazione.&Non implementato \\ \hline
		TS2F1.3.1&Viene verificato che sia possibile visualizzare il linguaggio d'implementazione dei server tramite un rettangolo informativo vicino al componente.&Non implementato \\ \hline
		TS1F1.3.2&Viene verificato che sia possibile visualizzare vicino ad ogni componente dell'applicazione monitorata un identificativo per tale entità.&Non implementato \\ \hline
		TS1F1.4&Viene verificato ch l'utente possa riposizionare i componenti all'interno della mappa topologica.&Non implementato \\ \hline
		TS2F1.4.1&Viene verificato che sia possibile riposizionare ogni componente della mappa trascinando con il puntatore.&Non implementato \\ \hline
		TS2F1.4.2&Viene verificato che sia possibile riposizionare automaticamente i componenti all'interno della mappa.&Non implementato \\ \hline
		TS0F1.5&Viene verificato che sia possibile visualizzare ciascun insieme di richieste fra due componenti della mappa topologica sotto forma di arco tra i due.&Non implementato \\ \hline
		TS2F1.5.1&Viene verificato che il colore degli archi cambi in base al tempo medio di un insieme di richieste fra due componenti dell'applicazione monitorata.&Non implementato \\ \hline
		TS2F1.5.1.1&Viene verificato che se il tempo di esecuzione medio di un insieme di richieste fra due componenti dell'applicazione monitorata sale oltre i 3 secondi l'arco che li unisce diventi rosso.&Non implementato \\ \hline
		TS2F1.5.1.2&Viene verificato che se il tempo di esecuzione medio di un
		insieme di richieste fra due componenti dell'applicazione monitorata è inferiore o uguale a 3 secondi l'arco che li unisce sia nero.&Non implementato \\ \hline
		TS0F1.5.2&Viene verificato che sia possibile visualizzare gli archi in base al tipo di richiesta eseguita fra due componenti della mappa topologica.&Non implementato \\ \hline
		TS0F1.5.2.1&Viene verificato che sia possibile visualizzare sotto forma di arco un insieme di richieste fra un server e un database.&Non implementato \\ \hline
		TS0F1.5.2.2&Viene verificato che sia possibile visualizzare sotto forma di arco un insieme di richieste HTTP fra due server.&Non implementato \\ \hline
		TS1F1.6&Viene verificato che sia possibile visualizzare informazioni sull'insieme di richieste fra due componenti dell'applicazione.&Non implementato \\ \hline
		TS1F1.6.1&Viene verificato che sia possibile visualizzare il tempo medio di risposta di un insieme di richieste fra due componenti dell'applicazione monitorata.&Non implementato \\ \hline
		TS1F1.6.2&Viene verificato che sia possibile visualizzare sotto forma di etichetta il tipo dell'insieme di richieste fatte fra due componenti.&Non implementato \\ \hline
		TS1F1.6..2.1&Viene verificato che sia possibile visualizzare delle etichette con la scritta "DB" sugli archi che presentano delle richieste fra server e database.&Non implementato \\ \hline
		TS1F1.6.2.2&Viene verificato che sia possibile visualizzare delle etichette con la scritta "HTTP" sugli archi che rappresentano delle richieste HTTP fra server e server.&Non implementato \\ \hline
		TS2F1.7&Viene verificato che sia possibile ridimensionare la grandezza della mappa topologica.&Non implementato \\ \hline
		TS2F1.7.1&Viene verificato che sia possibile ingrandire i componenti della mappa topologica.&Non implementato \\ \hline
		TS21.7.2&Viene verificato che sia possibile restringere la dimensione dei componenti della mappa.&Non implementato \\ \hline
		TS2F1.7.3&Viene verificato che sia possibile visualizzare la mappa topologica in modalità a schermo intero.&Non implementato \\ \hline
		TS1F1.8&Viene verificato che nel caso in cui ci sia un errore nel caricamento dei dati della mappa topologica venga visualizzato un messaggio d'errore. &Non implementato \\ \hline
		TS0F2&Viene verificato che sia possibile visualizzare una lista delle trace dell'applicazione monitorata.&Non implementato \\ \hline
		TS02.1&Viene verificato che sia possibile visualizzare i dettagli relativi ad ogni singola trace.&Non implementato \\ \hline
		TS&&Non implementato \\ \hline
		TS&&Non implementato \\ \hline
		TS&&Non implementato \\ \hline
		TS&&Non implementato \\ \hline
		TS&&Non implementato \\ \hline
		TS&&Non implementato \\ \hline
		
	\end{longtable}
	
		
	\subsubsection{Test di accettazione}
	Questi test vengono utilizzati durante il collaudo finale in presenza del committente.
	
	\begin{longtable}{|c|H|c|}
		\hline
		\textbf{Id Test} & \textbf{Descrizione} & \textbf{Stato}\\
		\hline
		
	\end{longtable}
	
	
