\appendix
\section{Resoconto delle attività di verifica}
	\subsection{Riassunto delle attività di verifica}
	\subsubsection{Revisione dei requisiti}
	L'attività di verifica, compito del \emph{\gl{Verificatore}}, è stata effettuata in corrispondenza della terminazione della stesura di ogni documento. La verifica dei documenti è stata eseguita facendo riferimento alle indicazioni contenute nelle \emph{Norme di Progetto v1.0.0} e misurando le metriche specificate in questo documento.
	\subsubsection{Revisione di progettazione}
	L'attività di verifica, compito del \emph{Verificatore}, è stata effettuata successivamente all'incremento di ogni documento. La verifica è stata eseguita facendo riferimento alle indicazioni contenute nelle \emph{Norme di Progetto v2.0.0} e misurando le metriche specificate in questo documento.
	
	\subsection{Dettaglio delle verifiche tramite analisi}
	\subsubsection{Analisi dei requisiti di massima}
	\paragraph{Documenti} \Spazio
	Vengono qui scritti i valori dell'indice di Gulpease misurato per ogni documento prodotto durante l'attività di \textit{Analisi dei Requisiti di Massima} e il relativo esito, basato sui valori precedentemente stabiliti.
	
	\begin{table}[H]
		\centering
		\begin{tabular}{|C{7cm}|C{2cm}|C{2cm}|}
			\hline
			\textbf{Documento} & \textbf{Valore} & \textbf{Esito}  \\
			\hline
			\textit{Analisi dei Requisiti v1.0.0} & 48.91 & superato \\
			\hline
			\textit{Glossario v1.0.0} & 55.03 & superato \\
			\hline
			\textit{Norme di Progetto v1.0.0} & 52.05 & superato \\
			\hline
			\textit{Piano di Progetto v1.0.0} & 48.30	5 & superato \\
			\hline
			\textit{Piano di Qualifica v1.0.0} & 51.25 & superato \\
			\hline
			\textit{Studio di Fattibilità v1.0.0} & 51.43 & superato \\
			\hline
			
		\end{tabular}
		\caption{Esiti del calcolo dell'indice Gulpease - \textit{Analisi dei Requisiti di Massima}}
	\end{table}
	\paragraph{Processi} \Spazio
	Vengono qui scritti i valori degli indici \textit{Schedule Variance} e \textit{Budget Variance}, per ogni documento prodotto durante l'attività di \textit{Analisi dei Requisiti di Massima.} 
	
	\begin{table}[H]
		\centering
		\begin{tabular}{|C{7cm}|C{2cm}|C{2cm}|}
			\hline
			\textbf{Documento} & \textbf{SV} & \textbf{BV}  \\
			\hline
			\textit{Analisi dei Requisiti v1.0.0} & 0\% & -4\%  \\
			\hline
			\textit{Glossario v1.0.0} & 0\% &  3\% \\
			\hline
			\textit{Norme di Progetto v1.0.0} & 0\% & 3.5\% \\
			\hline
			\textit{Piano di Progetto v1.0.0} & 0\% & 0\% \\
			\hline
			\textit{Piano di Qualifica v1.0.0} & 0\% & 2\% \\
			\hline
			\textit{Studio di Fattibilità v1.0.0} & 0\% & -2.5\%\\
			\hline
			
		\end{tabular}
		\caption{Esiti del calcolo degli indici Budget Variance e Schedule Variance - \textit{Analisi dei Requisiti di Massima}}
	\end{table}
     
     \label{AdR_Dettaglio}
    \subsubsection{Analisi dei requisiti di dettaglio e progettazione architetturale}
    \paragraph{Documenti} \Spazio
    Vengono qui scritti i valori dell'indice di Gulpease misurato per ogni documento aggiornato durante le attività di \textit{Analisi dei Requisiti di Dettaglio} e di \emph{Progettazione Architetturale} con annesso il relativo esito, basato sui valori precedentemente stabiliti.
    
    \begin{table}[H]
    	\centering
    	\begin{tabular}{|C{7cm}|C{2cm}|C{2cm}|}
    		\hline
    		\textbf{Documento} & \textbf{Valore} & \textbf{Esito}  \\
    		\hline
    		\textit{Analisi dei Requisiti v2.0.0} & 49.51 & superato \\
    		\hline
    		\textit{Glossario v2.0.0} & 54.83 & superato \\
    		\hline
    		\textit{Norme di Progetto 2.0.0} & 52.20 & superato \\
    		\hline
    		\textit{Piano di Progetto v2.0.0} & 48.84 & superato \\
    		\hline
    		\textit{Piano di Qualifica v2.0.0} & 50.62 & superato \\
    		\hline
    		\textit{Studio di Fattibilità v2.0.0} & 51.43 & superato \\
    		\hline
    		
    	\end{tabular}
    	\caption{Esiti del calcolo dell'indice Gulpease - \textit{Analisi dei Requisiti di Dettaglio} e \textit{Progettazione Architetturale}}
    \end{table}
    \paragraph{Processi} \Spazio
    Vengono qui scritti i valori degli indici \textit{Schedule Variance} e \textit{Budget Variance}, per ogni documento aggiornato durante l'attività di \textit{Analisi dei Requisiti di Dettaglio} e \textit{Progettazione Architetturale}. 
    
    \begin{table}[H]
    	\centering
    	\begin{tabular}{|C{7cm}|C{2cm}|C{2cm}|}
    		\hline
    		\textbf{Documento} & \textbf{SV} & \textbf{BV}  \\
    		\hline
    		\textit{Analisi dei Requisiti v1.0.0} & 0\% & -4\%  \\
    		\hline
    		\textit{Glossario v1.0.0} & 0\% &  3\% \\
    		\hline
    		\textit{Norme di Progetto v1.0.0} & 0\% & 3.5\% \\
    		\hline
    		\textit{Piano di Progetto v1.0.0} & 0\% & 0\% \\
    		\hline
    		\textit{Piano di Qualifica v1.0.0} & 0\% & 2\% \\
    		\hline
    		\textit{Studio di Fattibilità v1.0.0} & 0\% & -2.5\%\\
    		\hline
    		
    	\end{tabular}
    	\caption{Esiti del calcolo degli indici Budget Variance e Schedule Variance - \textit{Analisi dei Requisiti di Dettaglio} e \textit{Progettazione Architetturale}}
    \end{table}