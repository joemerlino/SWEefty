\section{Introduzione}
\textcolor{red}{ERRORI CORRETTI: aggiunte parole al glossario}
\subsection{Scopo del documento}
Il seguente documento ha l'obiettivo di esplicitare le norme, le convenzioni e la strumentazione che sarà adottata dal gruppo SWEefty durante l'intero svolgimento del progetto. Con questa prospettiva deve essere visionato da tutti i membri del gruppo, i quali sono tenuti ad osservare quanto scritto per mantenere consistenza ed omogeneità in ogni aspetto, durante il ciclo di vita del software che sarà prodotto durante il progetto.
\subsection{Scopo del prodotto}
Lo scopo del prodotto è sviluppare dei plugin Kibana in attraverso la libreria Javascript D3.
% - TODO: questa roba va completata quando sapremo più cose

\subsection{Ambiguità}
Per scongiurare ogni malinteso ed ogni ambiguità nella terminologia impiegata viene fornito il \textit{Glossario}, il quale contiene le definizioni dei termini in corsivo con una G a pedice.

\subsection{Riferimenti}
	\subsubsection{Normativi}
	\textcolor{red}{che è sta roba?}
		\begin{itemize}
			\item \textbf{Verbale di incontro interno:}
			\item \textbf{Verbale di incontro esterno:}		
		\end{itemize}
	\subsubsection{Informatici}
	\textcolor{red}{DA CORREGGERE: mettere linka  guide angular, js, kibana ed elastic serach}
	\begin{itemize}
		\item \LaTeX:  \href{https://en.wikibooks.org/wiki/LaTeX}{https://en.wikibooks.org/wiki/LaTeX}
	\end{itemize}
	