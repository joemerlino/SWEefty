\section{Introduzione}

\subsection{Scopo del documento}
Il seguente documento ha l'obiettivo di esplicitare le norme, le convenzioni e la strumentazione che sarà adottata dal gruppo SWEefty durante l'intero svolgimento del progetto. Con questa prospettiva deve essere visionato da tutti i membri del gruppo, i quali sono tenuti ad osservare quanto scritto per mantenere consistenza ed omogeneità in ogni aspetto del ciclo di vita del software che sarà prodotto durante il progetto.
\subsection{Scopo del prodotto}
Lo scopo del $prodotto_G$ è sviluppare dei plugin Kibana in attraverso la libreria Javascript D3.
%todo: questa roba va completata quando sapremo più cose
.
.
.
.

\subsection{Ambiguità}
Per scongiurare ogni malinteso ed ogni ambiguità nella terminologia impiegata viene fornito il \textit{Glossario}, il quale contiene le definizioni dei termini in corsivo con una G a pedice.

\subsection{Riferimenti}
	\subsubsection{Normativi}
		\begin{itemize}
			\item \textbf{ISO/IEC 12207} https://en.wikipedia.org/wiki/ISO/IEC_12207
			\item \textbf{Capitolato:} http://www.math.unipd.it/~tullio/IS-1/2017/Progetto/C7.pdf
			\item textbf{Verbale di incontro interno:}
			\item textbf{Verbale di incontro esterno:}		
		\end{itemize}
	\subsubsection{Informatici}
	eventuali guide latex, git, però devono essere serie....