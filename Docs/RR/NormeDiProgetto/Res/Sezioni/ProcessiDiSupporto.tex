\section{Processi di Supporto}


\subsection{Documentazione}
	
	\subsubsection{Scopo}
	Questo processo include e descrive le modalità di redazione e manutenzione dei documenti e le convenzioni 
	adottate per la scrittura di questi durante il ciclo di vita del prodotto software.
	
	\subsubsection{Aspettative}
	I risultati che ci aspettiamo di ottenere da una corretta implementazione di questo processo sono:
	\begin{itemize}  
		\item una visione precisa ed univoca dei documenti che vanno redatti durante il ciclo di vita del software;
		\item la stesura di documenti formali e coerenti.
		\item l’individuazione di una collezione di norme e convenzioni per la redazione di documentazione coerente e valida;
	\end{itemize}

	\subsubsection{Descrizione}
	In questa sezione devono essere indicate tutte le norme e le convenzioni adottate dal
	gruppo, per consentire la stesura di una documentazione valida e coerente.
	
	\subsubsection{Procedure}
	Per la redazione della documentazione il gruppo ha utilizzato il linguaggio di markup {\LaTeX}.
	
		\paragraph{Approvazione dei documenti}
		Ogni documento non formale in corrispondenza del completamento della stesura dovrà essere sottoposto al \textit{Responsabile di Progetto}, che dovrà delegare ai \textit{Verificatori} il controllo del contenuto e della forma. Nel caso tali \textit{Verificatori} rilevino degli errori, sarà loro compito riportarli al \textit{Responsabile di Progetto}, che a sua volta incaricherà il redattore del documento di correggerli. Questo ciclo va ripetuto fino a che il documento non è reputato completamente corretto dai  \textit{Verificatori}. In caso di assenso sulla correttezza e sulla qualità il documento può essere considerato come un documento formale. In caso contrario il \textit{Responsabile di Progetto} dovrà comunicare le motivazioni per cui il documento non è stato approvato, esplicitando le modifiche da apportare.
		
	\subsubsection{Template}
	Per agevolare la redazione della documentazione è stato creato un template \LaTeX contenente tutte le impostazioni stilistiche e grafiche citate in questo documento.
	
	\subsubsection{Struttura dei documenti}
	
		\paragraph{Prima pagina}
		Ogni documento è caratterizzato da una prima pagina che contiene le seguenti informazioni sul documento:
		\begin{itemize}
			\item Logo del gruppo;
			\item Titolo del documento;
			\item Nome del gruppo;
			\item Nome del progetto;
			\item Versione del documento;
			\item Cognome e nome dei redattori del documento;
			\item Cognome e nome dei verificatori del documento;
			\item Cognome e nome del responsabile approvatore del documento;
			\item Destinazione d’uso del documento;
			\item Lista di distribuzione del documento;
			\item Una breve descrizione del documento.
		\end{itemize}
	
		\paragraph{Registro delle modifiche}
		La seconda pagina di ogni documento contiene il diario delle modifiche del documento.
		Ogni riga del diario delle modifiche contiene:
		\begin{itemize}
			\item Un breve sommario delle modifiche svolte;
			\item Cognome e nome dell’autore;
			\item Ruolo dell’autore;
			\item Data della modifica;
			\item Versione del documento dopo la modifica.
		\end{itemize}
		La tabella contenente le modifiche è ordinata per data in ordine decrescente, affinchè la prima riga contenga la versione attuale del documento.
		
		\paragraph{Indice}
		In ogni documento, esclusi i verbali, è presente un indice delle sezioni, un indice delle figure e un indice delle tabelle. Nel caso non siano presenti figure o tabelle i rispettivi indici verranno omessi.
		
		\paragraph{Contenuto principale}
		\paragraph{Note a piè di pagina}
	\subsubsection{Versionamento}
	\subsubsection{Norme tipografiche}
		\paragraph{Stile del testo}
		\paragraph{Elenchi puntati}
		\paragraph{Formati comuni}
		\paragraph{Sigle}
	\subsubsection{Elementi grafici}
		\paragraph{Tabelle}
		\paragraph{Immagini}
	\subsubsection{Classificazione dei documenti}
		\paragraph{Documenti informali}
		\paragraph{Documenti formali}
		\paragraph{Verbali}
	\subsubsection{Strumenti}
		\paragraph{\LaTeX}
		\paragraph{TexStudio}
		\paragraph{Lucidchart}
	
\subsection{Verifica}

	\subsubsection{Scopo}
	\subsubsection{Aspettative}
	\subsubsection{Descrizione}
	\subsubsection{Attività}
		\paragraph{Analisi}
			\subparagraph{Analisi Statica}
			\subparagraph{Analisi dinamica}
		\paragraph{Test}
			\subparagraph{Test di unità}
			\subparagraph{Test di integrazione}
			\subparagraph{Test di sistema}
			\subparagraph{Test di regressione}					
			\subparagraph{Test di accettazione}
	\subsubsection{Strumenti}
			\paragraph{Verifica ortografica}
			\paragraph{Validazione W3C}
			\paragraph{Analisi statica}
			\paragraph{Analisi dinamica}
			\paragraph{Metriche}
			