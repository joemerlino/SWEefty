\section{Processi primari}

\subsection{Fornitura}
	
	\subsubsection{Scopo}
		Lo scopo di questo $processo_G$ è di trattare i termini e le norme, dalle piu' triviali alle piu' importanti, che tutti i componenti del gruppo SWEefty sono tenuti a rispettare per diventare fornitori dell'azienda IKS s.r.l e dei committenti Prof. Tullio Vardanega e Prod. Riccardo Cardin
	\subsubsection{Aspettative}
	Nel corso dell'intero progetto il gruppo intende instaurare con IKS in particolare nelle figure dei referente Stefano Bertolin e Stefano Lazzaro un rapporto di costante collaborazione al fine di:
    \begin{itemize}
	\item Stimare i costi del prodotto finale
	\item Concordare la qualifica del prodotto 
	\item Determinare vincoli su processi e requisiti
	\item Determinare aspetti fondamentali al fine di soddisfare la necessità del proponente
	\end{itemize}
	\subsubsection{Descrizione}
	Il gruppo intende mantenere un constante dialogo con il proponente in modo da poter avere un riscontro sull'efficacia del lavoro svolto e sull'applicazione delle tecnologie conivolte
	\subsubsection{Attività}
		\paragraph{Studio di fattibilità}
		\paragraph{Piano di Progetto}
		\paragraph{Piano di Qualifica}
	\subsubsection{Descrizione}
	Intendiamo mantenere un costante rapporto con il proponente in modo tale da rivecere un $feedback_G$ costante sul lavoro svolto.
	\subsubsection{Attività} 
		\paragraph{Studio di fattibilità} \mbox{} \\
		Sono state organizzate diverse riunioni al fine di permettere ai membri del team di scambiarsi opinioni sui capitolati proposti. 
		Abbiamo valutato il capitolato secondo diversi criteri:
		\begin{itemize}
			\item \textbf{Dominio tecnologico:} E' stata presa in considerazione la conoscenza attuale delle tecnologie richieste per portare a termine i vari capitolati, in base
			anche a esperienze passate avute con problemi simili;
			\item \textbf{Individuazione rischi:} sono state analizzate le varie difficoltà che potrebbero essere incontrare durante la realizzazione dei vari capitolati considerando 
			in modo particolare la mancanza di conoscenze adeguate relative alle tecnologie necessarie per realizzare i capitolati.
		\end{itemize}  
		\paragraph{Piano di Progetto} \mbox{} \\
		Il \emph{Progect Manager} affiancato dagli \emph{Amministratori} dovrà stilare un piano sa seguire per la realizzazione del progetto.
		Il documento dovrà rispettare e seguire i seguenti punti:
		\begin{itemize}
		\item \textbf{Analisi dei rischi:} si analizzano nel dettaglio i rischi che si potrebbero presentare durante lo svolgimento del progetto, capendo con quale probabilità potrebbero accadere e qual è il livello di gravità di ogni rischio;
		\item \textbf{Pianificazione:} si pianificano le attività da svolgere nel corso del progetto fornendo delle scadenze temporali precise;
		\item \textbf{Preventivo:} si stima, secondo la pianificazione, la quantità di lavoro necessaria per portare a termine ogni fase, in modo tale da riuscire a proporre un preventivo finale per il costo totale del progetto.
		\end{itemize} 
		
		\paragraph{Piano di Qualifica} \mbox{} \\
		Ai \emph{Verificatori} è assegnato il compito di scegliere un metodo per la $verifica_G$ e $validazione_G$ del materiale realizzato dal team.
		Il documento deve contenere:
		\begin{itemize}
		\item \textbf{Metodo di verifica:} vengono stabilite le procedure di controllo sulla qualità di processo e di prodotto. 
		\item \textbf{Metriche:} è necessario stabilire delle metriche oggettive per i documenti e i processi software;
		\item \textbf{Gestione della revisione:} si devono stabilire delle modalità per comunicare le anomalie e le procedure per controllare la qualità di processo;
		\item \textbf{Pianificazione collaudo:} è necessario definire dettagliatamente le motodologie di collaudo del progetto;
		\item \textbf{Resoconto attività verifica:} alla fine di ogni attività svolta si devono riportare le metriche relative e un resoconto sulla verifica effettuata.
		\end{itemize}
	
\subsection{Sviluppo}
	\subsubsection{Scopo}
	Questa sezione affrontà le attività e i compiti svolti dal gruppo al fine di produrre il software richiesto dall'azienda IKS Srl.
	
	\subsubsection{Aspettative}
	Al fine di implementare correttamente questa attività vi sono le seguenti aspettative:
	\begin{itemize}
		\item Realizzare un prodotto software conforme alle richieste del proponente
		\item Realizzare un prodotto software che soddisfi i test di verifica
		\item Realizzare un prodotto software che soddisfi i test di validazione
		\item Fissare degli obiettivi di $sviluppo_G$
		\item Fissare eventuali vincoli tecnologici
		\item Fissare i vincoli di design
	\end{itemize}
	
	\subsubsection{Descrizione}
	Il processo di sviluppo si svolge in accordo con lo standard ISO/IEC 12207.Pertanto si compone delle seguenti attività:
	\begin{itemize}
		\item Analisi dei requisiti
		\item Progettazione
		\item Codifica
	\end{itemize}
	
	\subsubsection{Attività}
		\paragraph{Analisi dei requisiti}
			\subparagraph{Scopo}
			Lo scopo di questa analisi è quella di elencare nel modo più preciso possibile, tutti i requisiti del progetto. 
			I requisiti sono estrapolati da diverse fonti:
				\begin{itemize}
				\item Specifica del capitolato 
				\item Incontri tenuti direttamente con il proponente
				\item Casi d'uso
			\end{itemize}
		    Dopo aver svolto questo processo viene stilato in modo formale un documento di analisi chiamato \textit{Analisi dei Requisiti v1.0.0}.
		    Quest'ultimo è steso dagli \emph{Analisti} e contiene una lista dei requisiti e dei casi d'uso.
		    In questo modo sarà possibile effettuare e definire dei test di superamento del software. 
			\subparagraph{Aspettative}
			L'obiettivo di questa attività è quello di definire in modo formale e dettagliato tutti i requisiti richiesti dal proponente.
			\subparagraph{Descrizione}
			Nel documento inerente a quest'analisi sono specificati i requisiti analizzati.
			Il metodo utilizzato per l'analisi dei requisiti è quella dei casi d'uso.
			\subparagraph{Casi d'uso}
			\subparagraph{Codice identificativo}
			\subparagraph{Requisiti}
			\subparagraph{Codice identificativo}
			\subparagraph{UML}
			I diagrammi UML devono essere realizzati usando la versione del linguaggio \textit{v2.0}
		\paragraph{Progettazione}
			\subparagraph{Scopo}
			\subparagraph{Aspettative}
			\subparagraph{Descrizione}
			\subparagraph{Specifica Tecnica}
			\subparagraph{Definizione di Prodotto}
		\paragraph{Codifica}
			\subparagraph{Scopo}
			Questa attività ha come scopo l'effettiva implementazione del prodotto software richiesto. In questa fase dunque si concretizzano attraverso la codifica le funzionalità previste dai requisiti concordati.
			\subparagraph{Aspettative}
			Obiettivo dell'attività è la creazione di un prodotto software confome alle richieste del proponente
			\subparagraph{Descrizione}
			L'attività deve rispettare quando imposto dai documenti \textit{Definizione di prodotto} e \textit{Piano di Qualifica}
			\subparagraph{Stile di codifica}
			Al fine di garantire uniformità all'intera $codebase_G$, ciascun membro del gruppo è obbligato a rispettare le seguenti norme:
			\begin{itemize}
			\item Formattazione: è richiesto l'uso di uno spazio (" ") ove possibile per rendere il codice di facile comprensione.
			Di seguito alcuni  esempi di buona e cattiva formattazione.
\begin{lstlisting}
	int a=3; // BAD
	int a = 3; // GOOD
	int a = 3,b = 5; // BAD
	int a = 3, b = 4; // GOOD

	getFoo(a,b,c,d) // BAD
	getFoo(a, b, c, d) // GOOD

	if(a==5){ // BAD
	if (a == 5) { // GOOD
\end{lstlisting}
			
			\item Indentazione: sono richiesti 4 spazi (si consiglia di impostare adeguatamente il proprio $IDE_G$), inoltre non sono permessi spazi bianchi o tabulazioni a fine riga.
			
			\item Lunghezza linee di codice: una riga di codice non deve superare i 100 caratteri, in caso contrario spezzare la riga di codice andando a capo
			
			\item Nomi: i nomi delle funzioni e delle variabili devono essere significativi e devono cominciare con la lettera minuscola e seguire la notazione $camelCase_G$. I nomi delle classi devono avere la prima lettera maiuscola mentre i nomi delle costanti devono essere scritti interamente in maiuscolo.
\begin{lstlisting}
var foo; // BAD
var profiledata // BAD
var profileData // GOOD
\end{lstlisting}
			
			\item Lingua: i nomi delle variabili i metodi le funzioni e i commenti vanno scritti in inglese.
			
			\item Lunghezza metodi e funzioni: i corpo dei metodi e delle funzioni non deve superare le 40 linee e i due gradi di indentazione. Superare tali limiti è chiaro segno che la funzione / metodo necessita di essere spezzata. A volte però può risultare preferibile mantenere tutta la logica su un solo metodo per facilitarne la comprensibilità. 
			
			\item Strutture di controllo (if, switch, for, etc...): le parentesi graffe per i costrutti che le prevedono devono essere in linea con il costrutto stesso. E' prevista l'omissione delle graffe soltanto quando il corpo della struttura è di una sola riga.
\begin{lstlisting}
	if (...) { // GOOD
		// CODICE
	} else {
		// CODICE
	}


	if (...) // BAD
	{
	 	// CODE
	}
	else 
	{
		// CODE
	}
\end{lstlisting}

			\item Commenti: i commenti devono essere esaustivi e non prolissi in modo da aiutare il lettore a comprendere al meglio ciò che si sta leggendo.
			
			
			E' inoltre consentito l'utilizzo di particolari tipi di commento con lo scopo di aiutare la stesura del codice. Vi sono:
			\begin{itemize}
				\item TODO: soluzione temporanee, memento di ogni tipo o zone che necessitano di un'implementazione.
				\begin{lstlisting}
				// TODO: this solution stinks
				// TODO: Write that query
				\end{lstlisting}
				
				\item HOW: per segnalare che non si ha capito l'implementazione o il funzionamento di una porzione di codice e che si necessita di uno studio più approfondito
				\begin{lstlisting}
				// HOW: need to study this API call, I'm puzzled
				\end{lstlisting}
				
				\item FIX: quando una particolare implementazione necessita di essere riparata o sistemata.
				\begin{lstlisting}
				// FIX: this implementation doesn't work with IE
				\end{lstlisting}
				Questi commenti vanno scritti tutti in maiuscolo e seguiti da due punti (:), questa sintassi favorisce la loro individuazione da particolari $tool_G$ che ne permetto una consultazione agevolata.
			\end{itemize}

			\end{itemize}
			\subparagraph{Intestazione}
			Ogni $file_G$ contenente codice sorgente deve avere la seguente intestazione:
			\begin{lstlisting}
				/*
				
				* File : nome file
				* Version : versione file
				* Type : tipo file
				* Date : data di creazione
				* Author : nome autore / i
				* E - mail : email autore / i
				*
				* License : tipo licenza				
				*				
				* Avvertenze : lista avvertenze e limitazioni
				*
				* Descrizione: breve descrizione del contenuto del file
				*
				* Registro modifiche :
				* Autore || Data || breve descrizione modifiche
				*
				*/
			\end{lstlisting}
			\subparagraph{Versionamento}
			\subparagraph{Ricorsione}
			E' caldamente sconsigliato l'utilizzo di ricorsione, esistono però dei casi in cui la soluzione ricorsiva rende il problema triviale, in tal caso è richiesta la verifica di correttezza e di terminazione della ricorsione.
	\subsubsection{Strumenti}
		\paragraph{Trender}
		\paragraph{Astah}
		\paragraph{IntelliJ IDEA}
			
			
