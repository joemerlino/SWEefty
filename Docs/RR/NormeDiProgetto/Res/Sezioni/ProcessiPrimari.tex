\section{Processi primari}

\subsection{Fornitura}
	
	\subsubsection{Scopo}
		Lo scopo di questo $processo_G$ (aggiungere pedice  G ?) e' di trattare i termini e le norme, dalle piu' triviali alle piu' importanti, che tutti i componenti del gruppo SWEefty sono tenuti a rispettare per diventare fornitori dell'azienda IKS s.r.l e dei committenti Prof. Tullio Vardanega e Prod. Riccardo Cardin
	\subsubsection{Aspettative}
	Nel corso dell'[intero progetto il gruppo intende instaurare con IKS in particolare nelle figure dei referente Stefano Bertolin e Stefano Lazzaro un rapporto di costante collaborazione al fine di:
	\begin{itemize}
	\item Determinare aspetti chiave per soddisfare i bisogni del proponente
	\item determinare i vincoli sui processi e sui requisiti
	\item stimare i costi 
	\item concordare la qualifica del prodotto
	\end{itemize}
	\subsubsection{Descrizione}
	Il gruppo intende mantenere un constante dialogo con il proponente in modo da poter avere un riscontro sull'efficacia del lavoro svolto e sull'applicazione delle tecnologie conivolte
	\subsubsection{Attività}
		\paragraph{Studio di fattibilità}
		\paragraph{Piano di Progetto}
		\paragraph{Piano di Qualifica}
	
\subsection{Sviluppo}
	\subsubsection{Scopo}
	\subsubsection{Aspettative}
	\subsubsection{Descrizione}
	\subsubsection{Attività}
		\paragraph{Analisi dei requisiti}
			\subparagraph{Scopo}
			\subparagraph{Aspettative}
			\subparagraph{Descrizione}
			\subparagraph{Casi d'uso}
			\subparagraph{Codice identificativo}
			\subparagraph{Requisiti}
			\subparagraph{Codice identificativo}
			\subparagraph{UML}
		\paragraph{Progettazione}
			\subparagraph{Scopo}
			\subparagraph{Aspettative}
			\subparagraph{Descrizione}
			\subparagraph{Specifica Tecnica}
			\subparagraph{Definizione di Prodotto}
		\paragraph{Codifica}
			\subparagraph{Scopo}
			Questa attivita' ha come scopo l'effettiva implementazione del prodotto software richiesto. In questa fase dunque si concretizzano attraverso la codifica le funzionalita' previste dai requisiti concordati.
			\subparagraph{Aspettative}
			Obiettivo dell'attivita' e' la creazione di un prodotto software confome alle richieste del proponente
			\subparagraph{Descrizione}
			L'attivita' deve rispettare qunado imposto dai documentiv \textit{Definizione di prodotto} e \textit{Piano di Qualifica}
			\subparagraph{Stile di codifica}
			Al fine di garantire uniformita'all'intera $codebase_G$, ciascun membro del gruppo e' obbligato a rispettare le sequenti norme:
			\begin{itemize}
			\item Formattazione: e' richiesto l'uso di uno spazio (" ") ove possibile per rendere il codice di facile comprensione.
			Di seguito alcuni  esempi di buona e cattiva formattazione
			\begin{lstlisting}
				int a=3; // BAD
				int a = 3; // GOOD
				int a = 3,b = 5; // BAD
				int a = 3, b = 4; // GOOD

				getFoo(a,b,c,d) // BAD
				getFoo(a, b, c, d) // GOOD

				if(a==5){ // BAD
				if (a == 5) { // GOOD
			\end{lstlisting}
			
			\item Indentazione: una tabulazione, inoltre non sono permessi spazi biachi o tabulazioni a fine riga
			
			\item Nomi: i nomi delle funzioni e delle variabili devono essere significativi e devono seguire la notazione $camelCase_G$. I nomi delle classi devono avere la prima lettera maiuscola.
			
			\item Strutture di controllo (if, switch, for, etc...): le parentesi graffe per i costrutti che le prevedono devono essere in linea con il costrutto stesso
			\begin{lstlisting}
				if (...) { // GOOD
 				   // CODICE
				} else {
				    // CODICE
				}
	

				if (...) // BAAAAD
				{
					 // CODE
				}
				else 
				{
				  // CODE
				}
			\end{lstlisting}
			\end{itemize}
			\subparagraph{Intestazione}
			\subparagraph{Versionamento}
			\subparagraph{Ricorsione}
	\subsubsection{Strumenti}
		\paragraph{Trender}
		\paragraph{Astah}
		\paragraph{IntelliJ IDEA}
			
			
