
\section{Introduzione}
	\subsection{Scopo del documento}
	Questo documento specifica la strategia che SWEefty adotterà per portare a termine il progetto.
	Nel documento verrà illustrato come SWEefty ha organizzato le attività, l'utilizzo delle risorse e la gestione dei rischi.
	Inoltre viene presentato il consuntivo delle risorse utilizzate durante lo svolgimento del progetto.
	
	\subsection{Scopo del prodotto}
	Il prodotto che SWEefty è tenuto a realizzare consiste in una coppia di plugin per Kibana che devono fornire due funzionalità fondamentali:
	\begin{itemize}
		\item \textbf{Visualizzazione mappa topologica del sistema:} il plugin deve visualizzare in maniera chiara ed intuitiva come le componenti del sistema interagiscono tra di loro, con annesse informazioni utili;
		\item \textbf{Visualizzazione della stack trace:} vengono visualizzate sotto forma di lista le interazioni fra i componenti e le richieste HTTP effettuate ai server. Per ogni trace della lista inoltre verrà visualizzata la rispettiva call tree e le queries effettuate ai database.
	\end{itemize}

	\subsection{Glossario}
	Nel \emph{Glossario} verranno riportati i termini tecnici e quelli ritenuti ambigui. Essi sono riconoscibili dal pedice G con la quale sono contrassegnati.
	
	\subsection{Riferimenti}
			\subsubsection{Normativi}
			\subsubsection{Informativi}
	\subsection{Scadenze}
	SWEefty ha deciso di rispettare le seguenti scadenze:
	\begin{itemize}
		\item \textbf{RR:}
		\item \textbf{RP:}
		\item \textbf{RQ:}
		\item \textbf{RA:}
	\end{itemize}
	\subsection{Ciclo di vita}	
	Il modello che SWEefty adotterà sarà quello incrementale. Il ciclo del modello incrementale è composto da 6 fasi, ogni ripetizione del ciclo identifica un "ciclo di incremento", il gruppo reitererà il ciclo fino a che la valutazione del prodotto sarà soddisfacente rispetto ai requisiti richiesti.
	Le motivazioni che ci hanno spinto a scegliere questo modello di sviluppo sono:
	\begin{itemize}
		\item Avendo una visione completa del prodotto da costruire, grazie alla possibilità di poter intervistare approfonditamente il proponente, il modello incrementale è particolarmente adatto;
		\item Riduce il rischio di fallimento, ed è una proprietà fondamentale per noi data la nostra inesperienza in progetti informativi;
		\item I requisiti utente sono trattati e classificati in base alla loro importanza strategica, quindi le funzionalità più importanti vengono implementate per prime;
		\item Prevede rilasci multipli e successivi e ciascuno realizza un incremento di funzionalità. Ciò permette di sottoporre continuamente al proponente un prototipo con le funzionalità implementate fino a quel momento così da poter ricevere una valutazione in corso d'opera e poter identificare immediatamente eventuali modifiche da fare;
		\item Prevede la scomposizione dei processi in attività, rendendo più facile la loro gestione e parallelizzazione. In questo modo le risorse utilizzate vengono efficientemente utilizzate.
	\end{itemize}
