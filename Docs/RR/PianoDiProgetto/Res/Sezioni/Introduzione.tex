\section{Introduzione}
	\subsection{Scopo del documento}
	Questo documento specifica la strategia che SWEefty adotterà per portare a termine il progetto.
	Nel documento varrà illustrato come SWEefty ha organizzato le attività, l'utilizzo delle risorse e la gestione dei rischi.
	Inoltre viene presentato il consuntivo delle risorse utilizzate durante lo svolgimento del progetto.
	
	\subsection{Scopo del prodotto}
	Il prodotto che SWEefty è tenuto a  realizzare è un plugin per Kibana che deve fornire due funzionalità fondamentali:
	\begin{itemize}
		\item \textbf{Visualizzazione topologica del sistema:} il plugin deve visualizzare a video la topologia del sistema sulla quale viene eseguito, così da fornire una chiara ed intuitiva visione del sistema con annesse informazioni utili (tempi medi di risposta, total execution time, self execution time, etc...);
		\item \textbf{Visualizzazione dello stack trace:} vengono mostrate le chiamate a metodi eseguiti dalla componente presa in analisi. Per ogni stack trace inoltre verrà visualizzata la rispettiva call tree.
	\end{itemize}

	\subsection{Glossario}
	Nel \emph{Glossario} verranno riportati i termini tecnici e quelli ritenuti ambigui. Essi sono riconoscibili dal pedice G con la quale sono contrassegnati.
	
	\subsection{Riferimenti}
			\subsubsection{Normativi}
			\subsubsection{Informativi}
	\subsection{Scadenze}
	SWEefty ha deciso di rispettare le seguenti scadenze:
	\begin{itemize}
		\item \textbf{RR:}
		\item \textbf{RP:}
		\item \textbf{RQ:}
		\item \textbf{RA:}
	\end{itemize}
	\subsection{Ciclo di vita}
	Il modello che SWEefty adotterà sarà quello incrementale. Il ciclo del modello incrementale è composto da 6 fasi, ogni ripetizione del ciclo identifica una "iterazione", il gruppo reitererà il ciclo fino a che la valutazione del prodotto sarà soddisfacente rispetto ai requisiti richiesti.
	Le motivazioni che ci hanno spinto a scegliere questo modello di sviluppo sono:
	\begin{itemize}
		\item Avendo una visione completa del prodotto da costruire, grazie alla possibilità di poter intervistare approfonditamente il proponente, il modello incrementale è particolarmente adatto;
		\item Riduce il rischio di fallimento, ed è una proprietà fondamentale per noi data la nostra inesperienza in progetti informativi;
		\item I requisiti utente sono trattati e classificati in base alla loro importanza strategica, quindi le funzionalità più importanti vengono implementate per prime;
		\item Prevede rilasci multipli e successivi e ciascuno realizza un incremento di funzionalità. Ciò permette di sottoporre continuamente al proponente un prototipo con le funzionalità implementate fino a quel momento così da poter ricevere una valutazione in corso d'opera e poter identificare immediatamente eventuali modifiche da fare;
		\item Prevede la scomposizione dei processi in attività, rendendo più facile la loro gestione e parallelizzazione. In questo modo le risorse utilizzate vengono efficientemente utilizzate.
	\end{itemize}
