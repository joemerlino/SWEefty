\section{Qualità di processo}
La capacità di raggiungere gli obiettivi prefissati è fortemente influenzata dalla qualità dei processi che portano al loro raggiungimento. Il gruppo SWEEefty utilizzerà la normativa ISO/IEC 15504 (conosciuta anche come SPICE) la quale fornisce un modello per poter giudicare lo stato di maturità di un processo.
Lo standard prevede sei diversi livelli di maturità di un processo. Di seguito vengono riportati soltanto quelli ritenuti necessari e ragionevoli per le dimensioni del progetto da svolgere.
\begin{itemize}
	\item \textbf{Level 0 - Incomplete Process:}
	il processo non è implementato o non raggiunge  gli  obiettivi  prefissati,  gli  output  del  processo  sono  pochi  o	inesistenti.  
	Non sono previsti attributi di processo per questo livello.
	\item \textbf{Level 1 - Performed process:}
		il  processo  viene  attuato  e  raggiunge  gli
		obiettivi prefissati, tuttavia non viene scrupolosamente controllato.  Gli
		attributi di tale processo sono:
		\begin{itemize}
			\item \textbf{P.A. 1.1 - Process performance:}
			capacità  di  raggiungere  i  propri
			obiettivi e produrre output identificabili.
		\end{itemize}
	\item \textbf{Level 2 - Managed process:}
		il processo viene attuato, controllato, tracciato e gli output prodotti raggiungono standard prefissati.  Gli attributi di tale processo sono:
		\begin{itemize}
			\item \textbf{P.A. 2.1 - Performance management:}
				capacità  di  produrre  output che raggiungano gli obiettivi prefissati;
			\item \textbf{P.A. 2.2 - Work product management:}
				capacit`a di produrre output controllato e tracciato.
		\end{itemize}
	\item \textbf{Level 3 - Established process:}
		il processo viene attuato e controllato seguendo  i  principi  dell’ingegneria  del  software.   Gli  attributi  di  tale processo sono:
		\begin{itemize}
			\item \textbf{P.A. 3.1 - Process definition:}
			capacità  di  produrre  output  che  si
			attengano agli standard dell’ingegneria del software;
			\item \textbf{P.A. 3.2 - Process resource:}
			capacità  di  produrre  output  efficace-
			mente utilizzando una quantità di risorse ragionevole.
		\end{itemize}
	\item \textbf{Level 4 - Predictable process:}
		il processo viene attuato con vincoli determinati  e  raggiunge  gli  obiettivi  previsti.   Il  processo  risulta  essere ben collaudato nella pratica.  Gli attributi di tale processo sono:
		\begin{itemize}
			\item \textbf{P.A. 4.1 - Process measurement:}
			capacità  di  utilizzare  le  misure ottenute durante l’esecuzione del processo per verificare in futuro il raggiungimento degli obiettivi prefissati;
			\item \textbf{P.A. 4.2 - Process control:}
			capacità  di  modificare  l’esecuzione  del processo in seguito ai dati raccolti.
		\end{itemize}
	\item \textbf{Level 5 - Optimizing process:}
		il  processo  ha  una  certa  consistenza nel raggiungere i propri obiettivi, viene ottimizzato per adempiere al meglio	agli obiettivi correnti e futuri.  
		Gli attributi di tale processo sono:
		\begin{itemize}
			\item \textbf{P.A. 5.1 - Process change:}
			capacità di tracciare tutti i cambiamenti del processo, siano essi strutturali o di esecuzione;
			\item \textbf{P.A. 5.2 - Continuous improvement:}
			capacità di implementare le modifiche applicate.
		\end{itemize}

		Per tutti gli attributi di processo, SPICE fornisce un metro di valutazione per misurare il loro raggiungimento:
		\begin{itemize}
			\item \textbf{N:} non posseduto (0\% - 15\%);
			\item \textbf{P:} parzialmente posseduto (16\% - 50\%);
			\item \textbf{L:} largamente posseduto (51\% - 85\%);
			\item \textbf{F:} totalmente posseduto (86\% - 100\%).
		\end{itemize}
\end{itemize}
	SWEefty decide di utilizzare il miglioramento continuo e trova nel ciclo PCDA la migliore tecnica per realizzare tale intento.
	Grazie al PDCA è possibile controllare costantemente i processi, tale ciclo è composto da:
	\begin{itemize}
		\item \textbf{Plan:} in questa fase si decidono gli obiettivi e i risultati desiderati;
		\item \textbf{Do:} si mettono in atto il piano deciso nella fase precedente;	
		\item \textbf{Check:} si controllano i risultati ottenuti dalla fase Do con i risultati previsti;
		\item \textbf{Act:} si individuano le cause che possono aver interferito con il raggiungimento degli obiettivi e si decidono le azioni da intraprendere per migliorare il processo.
	\end{itemize}

	\subsection{Project Assessment and Control Process}
		Questo processo ha l'obiettivo di determinare lo stato del lavoro svolto e assicurare che tutte le attivit' stiano rispettando i piani e i limiti  imposti	\subsubsection{Obiettivi}
			\begin{itemize}
				\item ogni elemento del gruppo dovrà svoglere il task(glossario) assegnatoli entro i limiti di tempo stabiliti
				\item  le risorse utilizzate dovranno rispettare i limiti prestabiliti
			\end{itemize}
	\subsubsection{Strategie}
	Il Project Manager dovrà mantenere un rapporto di constante dialogo con gli esecutori di task in modo da rilevare prontamente eventuali ritardi  e/o eccedenze nell'utilizzod delle risorse in modo da poter risolvere il problema tempestivamente.
	\subsubsection{Metriche}
		\paragraph{Schedule Variance}
		fornisce una misura su quanto lo stato del progetto è in ritardo o anticipo rispetto la pianificazione delle attività.
		\begin{itemize}
			\item  \textbf{Misurazione: } SV =  BCWP - BCWS dove il minuendo indica le attività svolte finora ed il sottraendo le attività che dovrebbero essere svolte finora;
			\item Valore ottimale: $\geq$ 0
			\item Valore accettato: $\geq$ 0
		\end{itemize}
		\paragraph{Cost Variance}
		Indica se si è speso più o meno di quanto è previsto
		\begin{itemize}
			\item \textbf{Misurazione: } BV = BCWS - ACWP, il minuendo indica il costo pianificato delle attività svolte ad una certa data ed il sottrando il costo effettivo delle attività svolte a tale data;
			\item Valore ottimale $\geq$ 0;
			\item Valore accettato $\geq$ 0.
		\end{itemize}

	\subsection{Risk Management Process}
Quesoto processo ha lo scopo di identificare, analizzare, valutare, gestire e monitorare i rischi durante la durata del processo.
\subsubsection{Obiettivi}
\begin{itemize}
	\item identificare tutti i possibili rischi che posso compromettere l'esito del progetto;
	\item sviluppare delle strategie per la prevenzione e la mitigazione degli stessi.
\end{itemize}
\subsubsection{Strategie}
La metodologia che SWEefty seguirà per assicurarsi che gli obiettivi vengano rispettati è riassumibile in cinque punti:
\begin{itemize}
	\item \textbf{Identificazione del rischio:} i membri del gruppo individuano tutti i possibili tutti rischi che potenzialmente posso interferire con il progetto ed i suoi output(glossario). Tutti i rischi individuati verranno registrati;

	\item \textbf{Analisi dei rischi: } per ogni rischio si determina la probabilità di avvenimento e le conseguenze derivate;

	\item \textbf{Valutazione del rischio: } ad ogni rischio viene assegnato un indice calcolato in base ai dati ricavati dall'attività precedente. Tale indice rappresenta quanto gravemente un rischio può intaccare gli esiti del progetto.

	\item \textbf{Trattamento del rischio: } l'indice precedentemente calcolato permette di stilare una classifica di gravità dei rischi. Il Project Manager, a partire dai rischi con indice maggiore, pianifica una strategia finalizzata a minimizzare i rischi o a riportare gli indici a valori accettabili. Completata questa attività il Project Manager avrà elaborato un piano per la mitigazione dei rischi ed il contenimento degli effetti collaterali.

	\item \textbf{Controllo e revisione dei rischi: } si monitorano e si revisionano i rischi finchè il progetto procede

	\textcolor{red} {TODO: ricordati di parlare della scala di valutazione dei rischi, come dai i voti? Da 1 a 5, 1 a 10? Fai delle ricerche}
\end{itemize}
\subsubsection{Metriche}
\paragraph{Rischi non individuati}
\begin{itemize}
	\item Rischi non individuati: indice del numero di rischi non individuati nella fase di analisi
	\begin{itemize}
		\item Misurazione: SV = BCWP - BCWS, il minuendo indica le attività svolte finora ed il sottraendo le attività che dovrebbero essere state svolte finora.
		\item Valore ottimale: 0;
		\item Valore accettato: 0 - 4.
	\end{itemize}
\end{itemize}

	\subsection{System Requirements Analysis Process}
\subsubsection{Obiettivi}
\begin{itemize}
	\item Individuare i requisiti di sistema;
	\item associare ad ogni requisito un test che verifichi il suo soddisfacimento; 
	\item tracciare ogni requisito in modo da poter risalire alle modifiche fatte ad esso durante lo sviluppo del prodotto;
	\item ogni requisito deve essere approvato dal committente.
\end{itemize}
\subsubsection{Strategie}
I requisiti vengono tracciati utilizzando SWEgo.
\subsubsection{Metriche}
\paragraph{Requisiti obbligatori soffisfatti}
\begin{itemize}
	\item \textbf{Misurazione:} $P=100\times\frac{N_S}{N_O}$ dove $N_S$ sono i requisiti soddisfatti, $N_O$ sono i requisiti obbligatori mentre $P$ è la percentuale dei requisiti obbligatori soddisfatti;
	\item \textbf{Range ottimale:} 100\%;
	\item \textbf{Range accettato:} 100\%.
\end{itemize}

	\subsection{System Architectural Design Process}
		\subsubsection{Obiettivi}
		\begin{itemize}
			\item associare ai requisiti una o più componenti del sistema;
			\item ogni componente dovrà essere associata ad un requisito, ciò la rende tracciabile;
			\item ottenere un sistema finale semplice con basso accoppiamento ed alta coesione;
			\item ottenere componenti basandosi sui buoni principi della programmazione ad oggetti come il riuso del codice,l'incapsulamento e la modularizzazione.
		\end{itemize}
		\subsubsection{Strategie}
		\textcolor{red} {TODO: che strategia usiamo qui?}
		\subsubsection{Metriche}
			\paragraph{Structural Fan-In - SFIN} 
			Indice di utilità da massimizzare, misura il numero di componenti che utilizzano un dato modulo, un valore alto denota un alto riutilizzo del codice.
			\begin{itemize}
				\item \textbf{Range ottimale:} $\geq 2$;
				\item \textbf{Range accettato:} $\geq 0$.
			\end{itemize}
			\paragraph{Structural Fan-Out - SFOUT}
			Indice di dipendenza da minimizzare, misura quante componenti vengono utilizzate dalla componente in esame, un valore alto denota un alto accoppiamento
			\begin{itemize}
				\item \textbf{Range ottimale:} 0-1;
				\item \textbf{Range accettato:} 0-5.
			\end{itemize}
            

	\subsection{System Detailed Design Process}
		\subsubsection{Obiettivi}
		\begin{itemize}
			\item fornire una progettazione di dettaglio del prodotto che andrà ad implementare i requisiti individuati;
			\item il grado di dettaglio deve dare informazioni sufficienti alla codifica e il testing senza bisogno di ulteriori informazioni.
		\end{itemize}
		\subsubsection{Strategie}
		Le componenti individuate durante l'analisi verranno suddivise in piccole unità codificabili e testabili.
		\subsubsection{Metriche}
			\paragraph{Metodi per classe}
			Indica il numero di metodi definiti in una classe.
			\begin{itemize}
				\item \textbf{Range ottimale:} 1-7; 
				\item \textbf{Range accettato:} 1-10.
			\end{itemize}
			\paragraph{Parametri per metodo}
			Indica il numero di parametri passati ad un metodo.
			\begin{itemize}
				\item \textbf{Range ottimale:} 0-4;
				\item \textbf{Range accettato:} 0-8.
			\end{itemize}
			
	\subsection{System Construction Process}
		\subsubsection{Obiettivi}
		\begin{itemize}
			\item definire le attività principali per la produzione di unità software;
			\item ottenere codice codice di bassa complessità per facilitarne la comprensibilità e la verifica;
			\item ottenere codice manutenibile;
			\item minimizzare gli sdoppiamenti di flusso.
		\end{itemize}
		\subsubsection{Strategie}
		Mantenere una bassa complessità del codice durante la stesura ed implementare al più presto dei test che ne verifichino il corretto funzionamento.
		\subsubsection{Metriche}
			\paragraph{Complessità ciclomatica}
			Metrica software utilizzata per misurare la complessità di un programma,misura il numero di cammini linearmente indipendenti attraverso il grafo di controllo di flusso.
			\begin{itemize}
				\item \textbf{Range ottimale:} 1-10;
				\item \textbf{Range accettato:} 1-15.
			\end{itemize}
			\paragraph{Linee di codice per linee di commento}
			Misura la percentuale di linee di commento presenti all'interno del codice di un programma.
			\begin{itemize}
				\item \textbf{Misurazione:} $P=100\times\frac{N_C}{N_{SLOC}}$ dove $N_C$ è il numero di linee di commento e $N_{SLOC}$ è il numero di linee di codice prodotte;
				\item \textbf{Range ottimale:} $\geq30$;
				\item \textbf{Range accettato:} $\geq25$.
			\end{itemize}
			\paragraph{Halstead Difficulty per-function}
			Misura la complessità di una funzione.
			\begin{itemize}
				\item \textbf{Misurazione:} $D=\frac{N_{UOP}}{2}\times\frac{N_{OD}}{N_{UOD}}$ dove $N_{UOP}$ e i l numero di operatori distinti, $N_{OD}$ è il numero totale di operandi mentre $N_{UOD}$ è il numero di operandi distinti;
				\item \textbf{Range ottimale:} 0-15;
				\item \textbf{Range accettato:} 0-25.
			\end{itemize}
			\paragraph{Halstead Volume per-function}
			Misura la dimensione dell'implementazione di un algoritmo basandosi sul numero di operazioni eseguite e sugli operandi di una funzione.
			\begin{itemize}
				\item \textbf{Misurazione:} $V=(N_{OP} + N_{OD})\times log_2(N_{UOP} + N_{UOD})$  dove $N_{OP}$ è il numero totale di operatori, $N_{OD}$ `e il numero totale di operandi, $N_{UOP}$ è
				il numero di operatori distinti e $N_{UOD}$ è il numero di operandi distinti;
			\item \textbf{Range ottimale:} 20-1000;
			\item \textbf{Range accettato:} 20-1500.
			\end{itemize}
			\paragraph{Halstead Effort per-function}
			rappresenta il costo necessario a scrivere il codice di una funzione.
			\begin{itemize}
				\item \textbf{Misurazione:} $E=D\times V$ dove $D$ è l'Halstead Difficulty e $V$ è l' Halstead Volume per-function;
				\item \textbf{Range ottimale:} 0-300;
				\item \textbf{Range accettato:} 0-400.
			\end{itemize}
			\paragraph{Indice di manutenibilità} 
			Questo indice da una misura di quanto il codice prodotto è manutenibile.
			\begin{itemize}
				\item \textbf{Misurazione}:$MI=171-3.42\times ln(E)-0.23\times ln(C)-16.2\times ln(LOC)$ dove $E$  l'Halstead Effort medio per modulo, $C$ è la complessità ciclomatica media per modulo mentre LOC è il numero medio di linee di codice per modulo;
				\item \textbf{Range ottimale:} 120-171;
				\item \textbf{Range accettato:} 100-171.
			\end{itemize}

	\subsection{Software Documentation Management Process}
		\subsubsection{Obiettivi}
		\begin{itemize}
			\item Produrre e manutenere le informazioni sul software prodotto;
			\item ottenere una documentazione comprensibile;
			\item mantenere la documentazione aggiornata allo stato attuale del progetto.
		\end{itemize}
		\subsubsection{Strategie}
		Mantenere un diario delle modifiche e la versione del documento, inoltre per semplificare la comprensione dei documenti viene fornito un glossario che spiega il significato dei termini tecnici. 
		\subsubsection{Metriche}
			\paragraph{Indice Gulpease}
			L'Indice Gulpease è un indice di leggibilità di un testo tarato sulla lingua italiana, per il suo calcolo vengono considerate due variabili linguistiche: la lunghezza delle parole e la lunghezza delle frasi rispetto al numero delle lettere.
			\begin{itemize}
				\item \textbf{Misurazione:} $G=89+\frac{300\times N_F-10\times N_L}{N_P}$ dove $N_F$ è il numero delle frasi, $N_L$ è il numero di lettere e $N_P$ è il numero di parole nel testo;
				\item \textbf{Range ottimale:} 50-100;
				\item \textbf{Range accettato:} 40-100.
			\end{itemize}

	\subsection{Software Verification Process}
		\subsubsection{Obiettivi}
		\begin{itemize}
			\item Verificare che tutte le componenti del sistema;
			\item verificare che le varie componenti abbiano il comportamento desiderato;
			\item utilizzare l'inspection per la verifica del codice;
			\item automatizzare il più possibile i test dinamici.
		\end{itemize}
		\subsubsection{Strategie}
		Stilare una checklist per poi effettuare inspection su documenti e codice.
		\subsubsection{Metriche}
			\paragraph{Branch Coverage} 
			Misura della percentuale di rami decisionali percorsi dai test dinamici.
			\begin{itemize}
			\item \textbf{Misurazione:} $BC=\frac{R_P}{R_T}\times 100$ dove $R_P$ è il numero di rami decisionali percorsi dai test mentre $R_T$ è il numeri di rami decisionali totali presenti nel software;
			\item \textbf{Range ottimale:} 70\%-100\%;
			\item \textbf{Range accettato:} 50\%-100\%.
		    \end{itemize}
			\paragraph{Statement Coverage}
			Misura la percentuale linee di codice eseguite almeno una volta dai test dinamici.
				\begin{itemize}
				\item \textbf{Misurazione:} $SC=\frac{L_E}{L_T}\times 100$ dove $L_E$ è il numero di linee di codice eseguite almeno una volta dai test mentre $L_T$ è il numero di linee di codice totali all'interno del software.
				\item \textbf{Range ottimale:} 70\%-100\%;
				\item \textbf{Range accettato:} 50\%-100\%.
		    	\end{itemize}
