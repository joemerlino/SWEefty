\section{Introduzione}
	\subsection{Scopo del documento}
	Lo scopo di questo documento è di riportare dettagliatamente tutti i processi e le metodologie che SWEefty rispetterà per assicurarsi che qualsiasi prodotto, sia esso documento o codice, rispetti i vincoli di qualità imposti dal proponente.
	Il rispetto di tali vincoli qualitativi impone un costante processo di verifica sulle attività svolte in modo da poter rilevare immediatamente eventuali problemi e poterli risolvere prontamente.
	
	\subsection{Scopo del prodotto}
	Il prodotto che SWEefty è tenuto a  realizzare è un plugin per Kibana che deve fornire due funzionalità fondamentali:
	\begin{itemize}
		\item \textbf{Visualizzazione topologica del sistema:} il plugin deve visualizzare a video la topologia del sistema sulla quale viene eseguito, così da fornire una chiara ed intuitiva visione del sistema con annesse informazioni utili (tempi medi di risposta, total execution time, self execution time, etc...);
		\item \textbf{Visualizzazione dello stack trace:} vengono mostrate le chiamate a metodi eseguiti dalla componente presa in analisi. Per ogni stack trace inoltre verrà visualizzata la rispettiva call tree.
	\end{itemize}

	\subsection{Glossario}
	Volendo evitare incomprensioni  ed equivoci per rendere la lettura del documento più semplice e chiara viene allegato il \emph{Glossario v1.0.0} nel quale sono contenute le definizioni dei termini tecnici, dei vocaboli ambigui, degli acronimi e delle abbreviazioni. Questi termini sono evidenziati nel presente documento con una G al pedice (esempio: $Glossario_{G}$).
	\subsection{Riferimenti}
		\paragraph{Normativi}
			\begin{itemize}
				\item \textbf{ISO/IEC 15504:} \url{https://en.wikipedia.org/wiki/ISO/IEC_15504};
				\item \textbf{ISO/IEC 25010:} \url{https://en.wikipedia.org/wiki/ISO/IEC_9126};
				\item \textbf{IEEE 610.12.90:} \url{https://standards.ieee.org/findstds/standard/610.12-1990.html};
				\item \textbf{Norme di Progetto:} \emph{Norme di Progetto v1.0.0}.
				\item \textbf{Piano di Progetto:} \emph{Piano di Progetto v1.0.0}.
			\end{itemize}

		\paragraph{Informativi} 
			\begin{itemize}
				\item \textbf{Qualità di processo:} \url{http://www.math.unipd.it/~tullio/IS-1/2017/Dispense/L15.pdf};
				\item \textbf{Indice di Gulpease:} \url{https://it.wikipedia.org/wiki/Indice_Gulpease}.
			\end{itemize}