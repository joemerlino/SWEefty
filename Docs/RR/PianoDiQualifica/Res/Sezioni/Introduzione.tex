\section{Introduzione}
	\subsection{Scopo del documento}
	Lo scopo di questo documento è di riportare dettagliatamente tutti i processi e le metodologie che SWEefty rispetterà per assicurarsi che qualsiasi prodotto, sia esso documento o codice, rispetti i vincoli di qualità imposti dal proponente.
	Il rispetto di tali vincoli qualitativi impone un costante processo di verifica sulle attività svolte in modo da poter rilevare immediatamente eventuali problemi e poterli risolvere prontamente.
	
	\subsection{Scopo del prodotto}
	Il prodotto che SWEefty è tenuto a  realizzare è un plugin per Kibana che deve fornire due funzionalità fondamentali:
	\begin{itemize}
		\item \textbf{Visualizzazione topologica del sistema:} il plugin deve visualizzare a video la topologia del sistema sulla quale viene eseguito, così da fornire una chiara ed intuitiva visione del sistema con annesse informazioni utili (tempi medi di risposta, total execution time, self execution time, etc...);
		\item \textbf{Visualizzazione dello stack trace:} vengono mostrate le chiamate a metodi eseguiti dalla componente presa in analisi. Per ogni stack trace inoltre verrà visualizzata la rispettiva call tree.
	\end{itemize}

	\subsection{Glossario}
	Nel \emph{Glossario} verranno riportati i termini tecnici e quelli ritenuti ambigui. Essi sono riconoscibili dal pedice G con la quale sono contrassegnati.
	\subsection{Riferimenti}
		\paragraph{Normativi}
		\textcolor{red}{TODO: sistemare link...sentire mich}
			\begin{itemize}
				\item \textbf{ISO/IEC 15504:}
				\item \textbf{ISO/IEC 25010:}
			\end{itemize}

		\paragraph{Informativi}
		\textcolor{red}{TODO: aggiungere se ce ne saranno}