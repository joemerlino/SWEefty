\section{Qualità di Prodotto}
Per quantificare e valutare la qualità del prodotto software il gruppo \textit{SWEefty} ha deciso di attenersi allo standard ISO/IEC 25010, conosciuto anche come SQuaRE, che individua e definisce otto caratteristiche da considerare per costruire un prodotto software con un elevato livello di qualità. Ognuna di queste caratteristiche viene suddivisa in varie sottocaratteristiche(31 in totale), le quali, attraverso dei parametri misurabili  permettono una valutazione oggettiva del grado di conseguimento della caratteristica considerata.
\textcolor{red}{PORTABILITY E COMPATIBILITY}
 
	\subsection{Functional Suitability}
	Con functional suitability si intende il livello di soddisfacimento dei bisogni espliciti o impliciti, raggiunto dal prodotto attraverso le sue funzionalità quando utilizzato sotto determinate condizioni.
		\subsubsection{Obiettivi}
			\begin{itemize}
				\item {\textbf{Functional completeness:} livello fino al quale l'insieme di funzioni copre tutti i compiti specificati e gli obiettivi dell'utente;}
				\item {\textbf{Functional correctness:} livello fino al quale un prodotto fornisce risultati corretti con il livello di precisione richiesto.}
			\end{itemize}
		\subsubsection{Metriche}
			\paragraph{Completezza dell'implementazione funzionale}\Spazio
			 Misura la quantità in percentuale di requisiti funzionali soddisfatti dalla corrente implementazione.
			\begin{itemize}
				\item {\textbf{Misurazione:}  $BC=1-\frac{N_{FM}}{N_{FI}}\times 100$, dove $N_{FM}$  è il numero di funzionalità mancanti nell'implementazione e $N_FI$ è il numero di funzionalità individuate nell'attività di analisi};
				\item {\textbf{Range ottimale:} 100;}
				\item {\textbf{Range accettazione:} 100.}
			\end{itemize} 

	\subsection{Reliability}
		Con reliability si intende il livello di performance mantenuto da specifiche funzioni di un prodotto in specifiche condizione per uno specifico periodo di tempo.
		\subsubsection{Obiettivi}
		\begin{itemize}
			\item {\textbf{Maturity:} livello fino al quale un prodotto garantisce Reliability durante un utilizzo normale;}
			\item {\textbf{Fault Tolerance:} livello fino al quale un prodotto riesce ad operare come previsto in presenza di fault di natura hardware o software.}
		\end{itemize}
		\subsubsection{Metriche}
			\paragraph{Densità di failure} \Spazio
			Misura la quantità percentuale di test che si sono conclusi con una failure.
			\begin{itemize}
				\item {\textbf{Misurazione:} $F=\frac{N_{FR}}{N_{TE}}\times 100$, dove $N_{FR}$ è il numero di failure rilevati durante l'attività di testing e ${N_TE}$ è il numero di test-case eseguiti};
				\item {\textbf{Range ottimale:} 0;}
				\item {\textbf{Range accettazione:} 0-10.}
			\end{itemize} 
			
	\subsection{Usability}
		Con usability si intende il livello a cui il dotto può essere utilizzato da degli utenti specifici per raggiungere specifici obbiettivi con efficacia, efficienza e soddisfazione in uno specifico contesto d'uso.  
		\subsubsection{Obiettivi}
		\begin{itemize}
			\item {\textbf{Appropriateness Recognizability:} livello a cui gli utenti riescono a riconoscere se il prodotto è adeguato per i loro bisogni;}
			\item {\textbf{Learnability:} livello di facilità con cui il prodotto può essere appreso dagli utenti per portare a termine determinati obiettivi con efficacia, efficienza, sicurezza e soddisfazione;} 
			\item {\textbf{User Interface Aesthetics:} livello a cui un interfaccia utente risulta piacevole per l'utente che la utilizza. }
			
		\end{itemize}
		\subsubsection{Metriche}
			\paragraph{Comprensibilità delle funzionalità offerte} \Spazio 
			Misura la quantità in percentuale di operazioni comprese dall'utente che non richiedono la consultazione del manuale.
			\begin{itemize}
				\item {\textbf{Misurazione:} $C=\frac{N_{FC}}{N_{FO}}\times 100$, dove $N_{FC}$ è il numero di funzionalità comprese in modo immediato dall'utente e $N_{FO}$ è il numero di funzionalità totali offerte dal sistema};
				\item {\textbf{Range ottimale:} 90-100;}
				\item {\textbf{Range accettazione:} 70-100.}
			\end{itemize} 
			\paragraph{Facilità di apprendimento} \Spazio 
			Misura il tempo medio che occorre ad un utente per imparare ad usare in maniera corretta una certa funzionalità.
			\begin{itemize}
				\item {\textbf{Misurazione:} indice dei minuti impiegati mediamente da un utente per apprendere una funzionalità offerta dal prodotto};
				\item {\textbf{Range ottimale:} 0-10;}
				\item {\textbf{Range accettazione:} 0-20.}
			\end{itemize} 
			
	\subsection{Performance Efficency}
		Con performance efficiency si intende il livello di performance relativo all'ammontare di risorse usate sotto determinate condizioni. 
		\subsubsection{Obiettivi}
			\begin{itemize}
				\item {\textbf{Time behaviour:} livello con cui il tempo di risposta e di processamento di un prodotto soddisfano certi requisiti durante l'esecuzione delle loro funzionalità;}
				\item {\textbf{Resource behaviour:} livello con cui la quantità e il tipo di risorse utilizzate da un prodotto durante l'esecuzione soddisfano certi requisiti.}
			\end{itemize}
		\subsubsection{Metriche}
			\paragraph{Tempo di risposta} \Spazio
			Misura la differenza media di tempo trascorsa dall’esecuzione di una funzionalità e la restituzione dell’eventuale risultato.
			\begin{itemize}
				\item {\textbf{Misurazione:} 
				$T_RISP=\frac{\sum\limits_{i=1}^n {T_i }}{n}$, dove $T_i$ è il tempo (in secondi) trascorso dalla richiesta di una funzionalità ed il completamento di questa con un eventuale restituzione del risultato};
				\item {\textbf{Range ottimale:} 0-3;}
				\item {\textbf{Range accettazione:} 0-8.}
			\end{itemize} 
			
	\subsection{Maintainability}
		Con maintainability si intende il livello di efficacia e di efficienza con cui un prodotto può essere modificato dai manutentori incaricati.
		\subsubsection{Obiettivi}
		\begin{itemize}
			\item {\textbf{Analysability:} livello di efficacia ed efficienza con cui è possibile quantificare l'impatto di modifiche intenzionali. È anche la capacità di diagnosticare mancanze e cause di failures all'interno del prodotto; }
			\item{\textbf{Modifiability:} livello di efficacia ed efficienza con cui un prodotto può essere modificato senza introdurre difetti e degradare le qualità del prodotto già presenti;}
			\item{\textbf{Testability:} livello di efficacia ed efficienza con cui si possono essere creati dei test per valutare il prodotto.}
		\end{itemize}
		\subsubsection{Metriche}
			\paragraph{Capacità di analisi failure} \Spazio
			Misura la quantità in percentuale di failures incontrate di cui sono state tracciate le cause.
			\begin{itemize}
				\item {\textbf{Misurazione:} $I=\frac{N_{FI}}{N_{FR}}\times 100 $, dove $N_{FI}$ è il numero di failure delle quali sono state individuate le cause e $N_{FR}$ è il numero di failures rilevate};
				\item {\textbf{Range ottimale:} 80-100;}
				\item {\textbf{Range accettazione:} 60-100.}
			\end{itemize} 
			\paragraph{Impatto delle modifiche} \Spazio
			Misura la quantità in percentuale di modiche introdotte per risolvere failures che hanno introdotto nuove failures nel prodotto.
			\begin{itemize}
				\item {\textbf{Misurazione:} $I=\frac{N_{FRE}}{N_{FR}}\times 100 $, dove $N_{FRE}$ è il numero di failure risolte introducendo nuove failure e $N_{FR}$ è il numero di failures risolte};
				\item {\textbf{Range ottimale:} 0-10;}
				\item {\textbf{Range accettazione:} 0-20.}
			\end{itemize} 
