\section{Valutazione sugli altri capitolati}
	\subsection{Capitolato C1 - Ajarvis}
		\subsubsection{Informazioni sul capitolato}
		\subsubsection{Descrizione}
		\subsubsection{Studio del dominio}
			\paragraph{Dominio applicativo}
			\paragraph{Dominio tecnologico}
		\subsubsection{Aspetti positivi}
		\subsubsection{Fattori di rischio}
		\subsubsection{Conclusioni}
			
			
	\subsection{Capitolato C2 - BlockCV}
		\subsubsection{Informazioni sul capitolato}
		\subsubsection{Descrizione}
		\subsubsection{Studio del dominio}
			\paragraph{Dominio applicativo}
			\paragraph{Dominio tecnologico}
		\subsubsection{Aspetti positivi}
		\subsubsection{Fattori di rischio}
		\subsubsection{Conclusioni}
	
	\subsection{Capitolato C3 - DeSpeect}
		\subsubsection{Informazioni sul capitolato}
		\subsubsection{Descrizione}
		\subsubsection{Studio del dominio}
			\paragraph{Dominio applicativo}
			\paragraph{Dominio tecnologico}
		\subsubsection{Aspetti positivi}
		\subsubsection{Fattori di rischio}
		\subsubsection{Conclusioni}
	
	\subsection{Capitolato C4 - ECoRe}
		\subsubsection{Informazioni sul capitolato}
		\subsubsection{Descrizione}
		\subsubsection{Studio del dominio}
			\paragraph{Dominio applicativo}
			\paragraph{Dominio tecnologico}
		\subsubsection{Aspetti positivi}
		\subsubsection{Fattori di rischio}
		\subsubsection{Conclusioni}
	
	\subsection{Capitolato C5 - IronWorks}
		\subsubsection{Descrizione}
		Il capitolato C5 \emph{IronWorks, utilità per la costruzione di software robusto} è proposto dall'azienda Zucchetti s.r.l. e pone come obbiettivo la generazione automaticamente del codice da diagrammi $UML_G$ per rendere più facile seguire le buone regole di programmazione. In particolare chiede la realizzazione di un'$editor_G$ per la costruzione di diagrammi UML \emph{diagrammi di $robustezza_G$} con la relativa generazione di codice Java per le $entit\acute a_G$ persistenti e per i metodi di scrittura e lettura verso un database relazionale. 
		\subsubsection{Studio del dominio}
			\paragraph{Dominio applicativo}
			\mbox{}\\
			L'ambito del capitolato è la realizzazione di diagrammi UML da cui deve essere possibile la generazione di codice: potranno essere disegnati diagrammi di robustezza seguendo le regole con cui i tre tipi di oggetti rappresentabili dai diagrammi di robustezza (le $interfacce_G$, le $procedure_G$ e le entità persistenti) possono reagire tra di loro.
			\paragraph{Dominio tecnologico}
			\mbox{}\\
			E' richiesta la conoscenza delle seguenti tecnologie per la realizzazione dell'applicazione web:
				\begin{itemize}
					\item Per la parte server:
					\begin{itemize}
						\item[-] Java
						\item[-] $TomCat_G$
						\item[-] JavaScript
						\item[-] $Node.js_G$
					\end{itemize}
					\item Per la parte client:
					\begin{itemize}
						\item[-] $HTML5_G$
						\item[-] $CSS_G$
					\end{itemize}
					\item Per l'archiviazione dei dati su file di testo o su database:
					\begin{itemize}
						\item[-] $XML_G$
						\item[-] $JSON_G$
						\item[-] $SQL_G$
					\end{itemize}
				\end{itemize}
		\subsubsection{Aspetti positivi}
		\begin{itemize}
			\item Il team presenta già conoscenze sulle tecnologie da utilizzare per il lato client dell'applicazione web;
			\item Il proponente fornisce dei software di riferimento, alcuni open source;
		\end{itemize}
		\subsubsection{Fattori di rischio}
		\begin{itemize}
			\item I diagrammi di robustezza sono poco conosciuti e poco utilizzati;
			\item Nel mercato sono già presenti molti software con le caratteristiche richieste;
		\end{itemize}
		\subsubsection{Conclusioni}
		Questo capitolato, sebbene non presentasse eccessive criticità, non ha suscitato particolare interesse rispetto ad altri. La limitazione alla rappresentazione dei soli diagrammi di robustezza è stata ritenuta troppo vincolante per la creazione di un prodotto completo e veramente utilizzabile nello sviluppo software.
		
		
	\subsection{Capitolato C6 - Marvin}
		\subsubsection{Descrizione}
		\subsubsection{Studio del dominio}
			\paragraph{Dominio applicativo}
			\mbox{}\\
			\paragraph{Dominio tecnologico}
			\mbox{}\\
			Per comprendere a fondo il dominio e per realizzare il progetto è richiesta la conoscenza delle seguenti tecnologie:
			\begin{itemize}
				\item $Truffle_G$
				\item $Etherscan.io_G$
				\item Javascript
				\item $ESLint_G$
				\item $React_G$
				\item $SCSS_G$
			\end{itemize}
		\subsubsection{Aspetti positivi}
		\subsubsection{Fattori di rischio}
		\begin{itemize}
			\item Il dominio applicativo del software è considerevolmente vasto e uno studio approfondito di tutti i suoi campi applicativi sarebbe troppo oneroso date le tempistiche ristrette.
		\end{itemize}
		\subsubsection{Conclusioni}
	
	\subsection{Capitolato C8 - TuTourSelf}
		\subsubsection{Descrizione}
		Il capitolato d'appalto C8 \emph{TuTourSelf, piattaforma di prenotazioni per artisti in tournee} è proposto dall'omonima start-up TuTourSelf S.r.l. che con questo progetto mette in gioco una propria idea.\\ Il prodotto richiesto è un'applicazione web che permetta agli artisti di organizzare il proprio tour mettendosi in contatto personalmente con i locali disponibili, adatti ad ospitare l'evento e che corrispondano alle esigenze dell'artista. 
		\subsubsection{Studio del dominio}
			\paragraph{Dominio applicativo}
			\mbox{}\\
			L'applicazione web richiesta vuole rendere l’organizzazione delle performance live il più semplice possibile così da aiutare il percorso degli artisti che vogliono rimanere indipendenti e rendere possibile l'ascesa dei giovani talenti che ancora non sono affiancati da agenzie.
			\paragraph{Dominio tecnologico}
			\mbox{}\\
			Per la realizzazione dell'applicazione web si chiede al gruppo la profonda conoscenza delle seguenti tecnologie:
			\begin{itemize}
				\item $HTML5_G$
				\item CSS
				\item JavaScript
				\item $React_G$
			\end{itemize}
		\subsubsection{Aspetti positivi}
		\begin{itemize}
			\item La conoscenza delle tecnologie utilizzate sono molto richieste nel mondo lavorativo;
			\item Interesse nella realizzazione di un'applicazione che funzioni anche su dispositivi mobile.
		\end{itemize}
		\subsubsection{Fattori di rischio}
		\begin{itemize}
			\item Nel mercato sono già presenti software strutturati in modo simile;
			\item Troppa libertà per la realizzazione del lato back-end.
		\end{itemize}
		\subsubsection{Conclusioni}
		Il team possiede già delle basi con le tecnologie richieste e preferisce avere un'esperienza progettuale con delle tecnologie nuove. Inoltre si ritiene più istruttivo lavorare a contatto di un'azienda già affermata nel territorio e nel mondo lavorativo che con una start-up.
	
	