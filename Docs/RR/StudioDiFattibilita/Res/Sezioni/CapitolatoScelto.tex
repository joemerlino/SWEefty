\newpage
\section{Capitolato scelto: C7 - OpenAPM}
	\subsection{Descrizione}
	Il capitolato d'appalto \emph{OpenAPM: cruscotto di Application Performance Management},  proposto dall'azienda IKS S.r.l., ha come obbiettivo la realizzazione di uno strumento APM basato su tecnologie Open Source. In particolare si vuole utilizzare Elastic Search e Kibana per:
	\begin{itemize}
		\item realizzare dashboard di monitoraggio  basate sulle view standard Kibana per il check in tempo reale delle performance applicative;
		\item realizzare plugin aggiuntivi per Kibana sviluppando widget e funzionalità aggiuntive legate allo use case APM. 
	\end{itemize}
	
	\subsection{Studio del Dominio}
		\subsubsection{Dominio applicativo}
		Il progetto si colloca nell'ambito dell'Application Performance Management (APM) e dei Big Data. Si vuole individuare e diagnosticare in modo semplice problematiche complesse che impattano sul servizio erogato dell'applicazione attraverso l'analisi di grandi moli di dati.
		\subsubsection{Dominio tecnologico}
		Per la comprensione del dominio applicativo e per lo sviluppo di dashboard di monitoraggio e plugin aggiuntivi per Kibana, si chiede al gruppo la conoscenza delle seguenti tecnologie:
		\begin{itemize}
			\item Elastic Search: server di ricerca, con supporto ad architetture distribuite. \MakeUppercase{è} strutturato ad indici e le informazioni sono gestite come in un database NoSQL;
			\item JSON: formato standard dei dati gestiti da Elastic Search;
			\item Kibana: applicazione web che presenta i dati salvati su Elastic Search;
			\item JavaScript: per la creazione di grafica 2d o 3d con l'aiuto di librerie come:
			\begin{itemize}
				\item[-] D3.js;
				\item[-] CanvasJS;
				\item[-] Chart.js.
			\end{itemize}
			\item Java: per l'elaborazione su ElasticSearch dei dati grezzi provenienti dall'applicazione da analizzare.
		\end{itemize}
	\subsection{Motivazioni della scelta}
		\subsubsection{Aspetti positivi}
			\begin{itemize}
				\item Interesse nel lavorare con software open source (Elastic Search, Logstash e Kibana) e di grande utilizzo nel mondo aziendale;
				\item Interesse verso il dominio applicativo del prodotto;
				\item Alcuni componenti del gruppo hanno già familiarità con le tecnologie richieste;
				\item Interesse per la realizzazione di un prodotto open source.
			\end{itemize}
		\subsubsection{Aspetti negativi}
			\begin{itemize}
				\item Rispetto ad altri linguaggi di programmazione, le conoscenze del linguaggio JavaScript sono incomplete per alcuni membri del gruppo che devono procedere ad un'approfondimento prima della fase di analisi dei requisiti;
				\item E' inoltre previsto l'utilizzo di specifiche librerie JavaScript per la creazione di grafica 2d e 3d non conosciute dal gruppo.
			\end{itemize}
	\subsection{Conclusioni}	
	Il gruppo ha ritenuto questo capitolato come quello più promettente per tecnologie e dominio applicativo. L'APM risulta oggi sempre più importante in ambito informatico e offre l'opportunità di lavorare con software e tecnologie molto richieste a livello lavorativo. Inoltre le criticità pervenute non sono state ritenute limitanti per affrontare un progetto che può portare a grandi risultati e, soprattutto, può essere molto stimolante per tutti i membri del gruppo.
			 