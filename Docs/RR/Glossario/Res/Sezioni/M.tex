\lettera{M}

\parola{Major feature}{Funzionalità importante di un software, che lo contraddistingue da applicativi con scopo simile.}

\parola{Major improvement}{Grandi modifiche riespetto alla versione precedente. Sono miglioramenti a funzionalità di grande importanza che vengono apportati ad un prodotto (software o documentazione). E.g: aggiunta gestione di dati provenienti da un databse, implementata l'interfaccia grafica.}

\parola{Management}{Complesso delle funzioni amministrative, direttive e gestionali di un'impresa o di un'azienda.}

\parola{Mappa topologica}{Visualizzazione sotto forma di grafo dei componenti di un'applicazione in cui i componenti come server e database sono i nodi e le interazioni fra essi, ovvero le richieste fatte da un componente ad un altro (come richieste HTTP o queries), sono gli archi.}

\parola{Mercurial}{Strumento per la gestione del controllo di versione del software.}

\parola{Minor improvement}{Letteralmente "piccoli miglioramenti", sono miglioramenti a funzionalità di lieve importanza che vengono apportati ad un prodotto (software o documentazione). E.g: aggiunti singoli metodi, corretti i nomi di alcune variabili.}

\parola{Modello incrementale}{Modello di sviluppo software basato sulla successione di sei passi fondamentali: pianificazione, analisi dei requisiti, progetto, implementazione, prove e valutazione.
} 

\parola{MongoDB}{Database Managment System  non relazionale, orientato ai documenti libero ed open source. Classificato come un database di tipo NoSQL, non si basa su tabelle come per i  database relazionali ma si bassa su documenti BSON, rendendo l'integrazione di dati di alcuni tipi di applicazioni più facile e veloce. } 

\parola{Mozilla Firefox}{Web browser libero e multipiattaforma, mantenuto da Mozilla Foundation.}

\parola{Multipiattaforma}{Applicazione software, linguaggio di programmazione o un dispositivo hardware che funziona su più di un sistema o piattaforma.}