\lettera{M}

\parola{major feature}{Funzionalità importante di un software, che lo contraddistingue da applicativi con scopo simile.}
\parola{major improvement}{Grandi modifiche riespetto alla versione precedente. Sono miglioramenti a funzionalità di grande importanza che vengono apportati ad un prodotto (software o documentazione). E.g: aggiunta gestione di dati provenienti da un databse, implementata l'interfaccia grafica.}
\parola{mappa topologica}{Visualizzazione sotto forma di grafo dei componenti di un'applicazione in cui i componenti come server e database sono i nodi e le interazioni fra essi, ovvero le richieste fatte da un componente ad un altro (come richieste HTTP o queries), sono gli archi.}
\parola{Mercurial}{Strumento per la gestione del controllo di versione del software.}
\parola{minor improvement}{Letteralmente "piccoli miglioramenti", sono miglioramenti a funzionalità di lieve importanza che vengono apportati ad un prodotto (software o documentazione). E.g: aggiunti singoli metodi, corretti i nomi di alcune variabili.}
\parola{MongoDB}{Database Managment System  non relazionale, orientato ai documenti libero ed open source. Classificato come un database di tipo NoSQL, non si basa su tabelle come per i  database relazionali ma si bassa su documenti JSON, rendendo l'integrazione di dati di alcuni tipi di applicazioni più facile e veloce. }  
\parola{multipiattaforma}{Un'applicazione software multipiattaforma è un'applicazione che può funzionare su più di un sistema.}
