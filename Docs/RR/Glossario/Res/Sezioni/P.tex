\lettera{P}

\parola{package}{nell'Unified Modeling Language è usato per raggruppare elementi e fornire un namespace per gli elementi raggruppati. Può contenere altri package, fornendo così un'organizzazione gerarchica dei package.}  
\parola{Plotly.js}{libreria grafica JavaScript open source.}  
\parola{plugin}{programma non autonomo che interagisce con un altro programma per ampliarne o estenderne le funzionalità originarie.}  
\parola{PNG}{Portable Network Graphics (abbreviato PNG) è un formato di file grafico bitmap per memorizzare immagini.}  
\parola{processo}{insieme di attività collegate tra loro che trasformano ingressi in uscite secondo regole fissate e trmite risorse limitate.}  
\parola{prodotto}{indica il risultato di un'attività, sia esso un documento, del codice sorgente o un qualsiasi risultato verificabile che possa essere offerto per soddisfare un bisogno o un’esigenza.}  
\parola{Progettista}{persona con competenze tecniche e tecnologiche aggiornate e ampia esperienza professionale. Si occupa dello sviluppo della soluzione al problema presentato tramite le attività di progettazione, spesso assumendo anche responsabilità di scelta e gestione.}  
\parola{progetto}{insieme di attività e compiti che prevedono il raggiungimento di determinati obiettivi con specifiche fissate. SOno definite date di inizio e fine durante la quale si può predisporre di limitate risorse che vengono consumate nello svolgersi delle attività. }  
\parola{proponente}{colui che presenta una proposta di progetto, nel nostro caso l'azienda IKS.}
\parola{Push}{atto di apportare delle modifiche a file presenti in un repository Git}