\lettera{C}

\parola{Call tree}{Albero delle chiamate, rappresenta l'insieme delle chiamate a funzioni o funzionalità del sistema eseguite da un programma.}

\parola{Camel Case}{Notazione usata per scrivere parole composte o frasi unendo tutte le parole tra loro, eliminando quindi gli spazi. Viene molto utilizzato nei linguaggi di programmazione in quanto nei nome delle variabili o delle funzioni non sono permessi spazi.}

\parola{Canali di comunicazione}{Strumenti attraverso la quale è possibile trasmettere informazioni. La voce, internet e Whatsapp sono esempi di canali di comunicazione.}
 
\parola{CanvasJS}{Libreria Javascript per disegnare grafici interattivi.}

\parola{Capitolato}{Documento tecnico, in genere allegato ad un contratto d'appalto,  che contiene il dettaglio delle opere e delle modalità realizzative delle stesse per la corretta realizzazione degli intenti del committente.}

\parola{Change Significance}{In italiano "significato del cambiamento", indica nell'ambito del versionamento la politica con cui la versione di un file va aggiornata. In base alla significatività del cambiamento, cioè a quanto importanti sono le informazioni introdotte, la versione di un file cambia conseguentemente.}

\parola{Chart.js}{Libreria Javascript per disegnare grafici interattivi.}

\parola{Cluster}{Insieme di computer connessi tra loro tramite una rete telematica. Lo scopo di un cluster è quello di distribuire una elaborazione molto complessa tra i vari computer.}

\parola{Codebase}{Indica l'intera collezione di codice sorgente usata per costruire una particolare applicazione o un particolare componente.}

\parola{Committente}{Chi ordina un lavoro, una prestazione, o si impegna all'acquisto di una merce per conto proprio.}

\parola{Criptovaluta}{Valuta paritaria, decentralizzata e digitale la cui implementazione si basa sui principi della crittografia per convalidare le transazioni e la generazione di moneta in sé.}

\parola{Crossplatform}{Applicazione software, linguaggio di programmazione o un dispositivo hardware che funziona su più di un sistema o piattaforma.}

\parola{CSS}{Linguaggio usato per definire la formattazione di documenti HTML, XHTML e XML ad esempio i siti web e relative pagine web.}

\parola{Cytoscape.js}{Libreria Javascript per disegnare grafici interattivi.}