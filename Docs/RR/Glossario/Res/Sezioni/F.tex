\lettera{F}


\parola{fault}{È la causa che sta alla base di uno stato incorretto del sistema. Può essere di natura umana, come un design fault o un logic fault, oppure di natura casuale come un hardware fault.}
\parola{file}{Viene utilizzato per riferirsi a un contenitore di informazioni/dati in formato digitale, tipicamente presenti su un supporto digitale di memorizzazione. }
\parola{feature}{È una caratteristica distintiva di un prodotto software. Può essere legato alla portabilità, alla performance oppure alle funzionalità.}
\parola{feedback}{Il feedback è un giudizio dato da un utente di un sistema sulle azioni compiute dal sistema.}
\parola{framework}{Con framework si identifica un'architettura logica di supporto (spesso un’implementazione logica di un particolare design pattern) su cui un software può essere costruito per facilitarne lo sviluppo da parte del programmatore.}
\parola{frontend}{Con frontend si intende la parte di un software che un utente può vedere e con il quale può interagire.}