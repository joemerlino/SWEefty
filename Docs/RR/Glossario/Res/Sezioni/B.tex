\lettera{B}

\parola{backend}{È la parte dell'architettura che l'utente non vede, generalmente si occupa di elaborare i dati ricevuti dall'utente. }

\parola{big data:}{Termine con la quale ci si riferisce all'insieme delle tecnologie e delle metodologie di analisi di dati massivi. Inoltre il termine denota anche la capacità di estrapolare e mettere in relazione un'enorme mole di dati eterogenei, strutturati o non per scoprirne gli eventuali legami soggiacenti.}

\parola{blockchain}{È un database distribuito composto da una lista i cui elementi sono collegati tra loro e resi sicuri mediante la crittografia. La caratteristica degna di nota è riesce a registrare le transazioni tra due parti in modo efficiente, verificabile e permanente.}

\parola{Bot}{Conosciuto anche come web robot o Internet bot è un'applicazione software che esegue scripts in maniera continua e automatizzata su Internet.}

\parola{bug}{In italiano baco, identifica un errore nella scrittura del codice sorgente che porta a comportamente anomali e non previsti del programmma. Non solo, un bug può essere introdotto anche in fase di compilazione o di progettazione del programma.}

\parola{bugfix}{E' una porzione di codice progettata per risolvere i malfunzionamenti o i comportamente anomali introdotti da un bug.}