% - PER STILARE QUESTO DOCUMENTO VANNO INSERITI I DATI.
% - VISTO CHE IL FILE È LUNGO ED INTRICATO PER SEMPLIFICARE HO MESSO DEI COMMENTI
% - CHE INDICANO I PUNTI DOVE INSERIRE LE INFORMAZIONI CHE MANCANO.
% - INSERISCILI IN CORRISPONDENZA DEI COMMENTI:

% - "INSERT TITOLO"
% - "INSERT USO"
% - "INSERT DATA"
% - "INSERT DESCRIZIONE"
% - "INSERT INFORMAZIONI INCONTRO"
% - "INSERT DOMANDE E RISPOSTE" (intanto lo strutturiamo così, con domande e rispose, se servono altri tipi di struttura cambieremo)

% - "INSERT TITOLO" (verosimilmente sarà una cosa tipo "Verbale Esterno/Interno del xx/yy") 
\newcommand{\Titolo}{Verbale riunione 19 Dicembre 2017}

\newcommand{\Gruppo}{SWEefty}

\newcommand{\Versione}{1.0.0}

\newcommand{\Redazione}{Elia Montecchio}

\newcommand{\ACapoRedazione}{Elia Montecchio}


\newcommand{\Distribuzione}{Prof.Tullio Vardanega \newline Prof.Riccardo Cardin \newline SWEefty}

\newcommand{\InformazioniDocumento}{}
% - "INSERT USO (INTERNO/ESTERNO)"
\newcommand{\Uso}{Esterno}
% - "INSERT DATA"
\newcommand{\Data}{22 Dicembre 2017}
% - "INSERT DESCRIZIONE"
\newcommand{\DescrizioneDoc}{Questo documento riporta in modo formale il riassunto della rinuione tenutasi nella data sopra riportata.  }



\documentclass[a4paper, oneside, openany]{article}

\usepackage{graphbox}
% permette di modificare i margini
\usepackage[top=3.1cm, bottom=3.1cm, left=2.2cm, right=2.2cm]{geometry}

\usepackage{lastpage} %info sul # dell'ultima pagina del documento
\usepackage{fancyhdr} %per modificare dimensioni,margini, intestazioni e righe a piè di pagina
\fancypagestyle{plain}{
  % cancella tutti i campi di intestazione e piè di pagina
  \fancyhf{}
  
  \lhead{\includegraphics[width=3cm]{../../../CommonImages/logo.jpg}}
  
  \rhead{sweeftyteam@gmail.com}
  
  \lfoot{ %piè di pagina
   {\Titolo} \ - \textit{{\Gruppo}}
  }
  \rfoot{Pagina \thepage{} di \pageref{LastPage}} %es: pag: 4 di 10

  %linea orizzontale alle posizioni top e bottom della pagina
  \renewcommand{\headrulewidth}{0.2	pt}  
  \renewcommand{\footrulewidth}{0.2pt}
}
\pagestyle{plain}

%Comando Spazio
\newcommand{\Spazio}{\mbox{} \\ \mbox{} \\ }  


%\usepackage{calc} %introduce la notazione infissa per le op. aritmetiche interne a LaTeX

\usepackage[utf8]{inputenc}
\usepackage[T1]{fontenc}
\usepackage[italian]{babel} %il documento è in italiano
%\usepackage{textcomp} %The pack­age sup­ports the Text Com­pan­ion fonts, which pro­vide many text sym­bols
%(such as baht, bul­let, copy­right, mu­si­cal­note, onequar­ter, sec­tion, and yen), in the TS1 en­cod­ing.

\usepackage{graphicx}       %permette di inserire delle immagini
\usepackage{caption}        %numerazione figure e loro descrizione testuale
\usepackage{subcaption}     %sottofigure numerabili
\usepackage{float}  %permette di inserire un # qualsiasi di figure fluttuanti
\usepackage{xcolor}
\usepackage{rotating} %permette di ruotare le immagini
%\usepackage{changepage} %utile se c'è bisogno di aggiustare margini per centrare figure

%package utili per la math mode ( $ ... $ o \[ ... \] )
\usepackage{amsmath}
\usepackage{amssymb}
\usepackage{amsfonts}
%\usepackage{euler}    %font 'ams euler', lo stesso di 'Concrete Mathematics' di Knuth
\usepackage{amsthm}
\usepackage{mathtools}

% package utili per tabelle(\thead in particolare)
\usepackage{array, booktabs, caption}
\usepackage{makecell}
\renewcommand\theadfont{\bfseries}
\usepackage{boldline}

\usepackage{listings} %permette di inserire degli spezzoni di codice

\usepackage{tikz} %disegno di immagini vettoriali a schermo. Utile per grafi
\usetikzlibrary{arrows.meta}
\usetikzlibrary{graphs}
\usetikzlibrary{arrows}
%\usepackage{tikz-uml} %serve per disgnare l'UML, fantastica guida:
%https://perso.ensta-paristech.fr/~kielbasi/tikzuml/var/files/doc/tikzumlmanual.pdf
%download package: http://perso.ensta-paristech.fr/~kielbasi/tikzuml/

%package per le tabelle
\usepackage{booktabs} %permette di poter usare delle liste nelle tabelle
\usepackage{tabularx} 
\usepackage{longtable} %una tabella può continuare su più pagine
\usepackage{multirow} %utile per visualizzare una cella su più righe
%\usepackage{multicolumn} %cella su più colonne
%\usepackage[table]{xcolor} %rende disponibile l'utilizzo di un colore per lo sfondo
                        %delle celle di una tabella

%crea una cella per le tabelle in grado di andare a capo con \newline
%https://tex.stackexchange.com/questions/12703/how-to-create-fixed-width-table-columns-with-text-raggedright-centered-raggedlef
\usepackage{array}
\newcolumntype{L}[1]{>{\raggedright\let\newline\\\arraybackslash\hspace{0pt}}m{#1}}
\newcolumntype{C}[1]{>{\centering\let\newline\\\arraybackslash\hspace{0pt}}m{#1}}
\newcolumntype{R}[1]{>{\raggedleft\let\newline\\\arraybackslash\hspace{0pt}}m{#1}}


%indice con i puntini
\usepackage{tocloft}
\renewcommand\cftsecleader{\cftdotfill{\cftdotsep}}

%http://ctan.mirror.garr.it/mirrors/CTAN/macros/latex/contrib/appendix/appendix.pdf
\usepackage{appendix} %aggiunge dei comandi per l'appendice
\usepackage{parskip} %aiuta LaTeX a trovare il miglior stile per i page break
\setcounter{secnumdepth}{5} % numera i sottoparagrafi
\setcounter{tocdepth}{5} %aggiunge all'indice i sottoparagrafi
%\usepackage{titlesec} %\begin{paragraph} si può usare come subsubsubsection!


\usepackage{breakurl}%\url{...} può continare alla linea successiva. (si può andare a capo)

\definecolor{Maroon}{cmyk}{0, 0.87, 0.68, 0.32}
\usepackage[colorlinks=true]{hyperref}
\hypersetup{
    colorlinks=true,
    citecolor=black,
    filecolor=black,
    linkcolor=black, % colore dei link interni
    urlcolor=Maroon  % colore dei link interniesterni
}

%impostazioni per il codice che deve finire dentro a
%\begin{lstlisting}

\definecolor{listinggray}{gray}{0.9}
\definecolor{lbcolor}{rgb}{0.9,0.9,0.9}
\lstset{
backgroundcolor=\color{lbcolor},
    tabsize=4,    
%   rulecolor=,
    language=[GNU]C++,
    basicstyle=\scriptsize,
    upquote=true,
    aboveskip={1.5\baselineskip},
    columns=fixed,
    showstringspaces=false,
    extendedchars=true,
    inputencoding=utf8,
    breaklines=true,
    prebreak = \raisebox{0ex}[0ex][0ex]{\ensuremath{\hookleftarrow}},
    frame=single,
    numbers=left,
    showtabs=false,
    showspaces=false,
    showstringspaces=false,
    identifierstyle=\ttfamily,
    keywordstyle=\color[rgb]{0,0,1},
    commentstyle=\color[rgb]{0.026,0.112,0.095},
    stringstyle=\color[rgb]{0.627,0.126,0.941},
    numberstyle=\color[rgb]{0.205, 0.142, 0.73},
%        \lstdefinestyle{C++}{language=C++,style=numbers}’.
}
\lstset{
  backgroundcolor=\color{lbcolor},
  tabsize=4,
  language=C++,
  captionpos=b,
  tabsize=3,
  frame=lines,
  numbers=left,
  numberstyle=\tiny,
  numbersep=5pt,
  breaklines=true,
  showstringspaces=false,
  basicstyle=\footnotesize,
  identifierstyle=\color{magenta},
  keywordstyle=\color[rgb]{0,0,1},
  commentstyle=\color{orange},
  stringstyle=\color{red}
}


 \newgeometry{top=4cm}

\begin{document}
\begin{titlepage}
	\begin{center}
		
		\begin{center}
			%% qui metteteci l'immagine di copertina. Io ho messo quella dell'uni,
			%voi mettete quella del vostro grupo
			\centerline{\includegraphics[scale=0.24]{../../../CommonImages/logo.jpg}}
		\end{center}
		
		\vspace{1cm}
		
		\begin{Huge}
			\textbf{\Titolo{}} \\
		\end{Huge}
		
		\vspace{9pt}  
		
		\begin{large}
			\Gruppo{}\ - \Data{}
		\end{large}	  
		
		\vspace{15pt}
		
		\bgroup
		\def\arraystretch{1.3}
		\centering
		\begin{tabular}{c|L{5cm}}
			\multicolumn{2}{c}{\textbf{\InformazioniDocumento{}} } \\ \hline
			Versione &  \Versione{}\\
			Redazione & \ACapoRedazione{} \\

			Uso & \Uso \\
			Distribuzione & \Distribuzione{}
		\end{tabular}
		\egroup
		
		\vspace{15pt}
		
		\begin{center}
			\textbf{Descrizione\\}
			\DescrizioneDoc{}
		\end{center}
		
	\end{center}
\end{titlepage}

\restoregeometry
	
	\section{Informazioni generali}
		\subsection{Informazioni incontro}
			% - INSERT INFORMAZIONI INCONTRO
			\begin{itemize}
				\item { \textbf{Luogo:} Azienda IKS Srl  }
				\item { \textbf{Data:} 19 Dicembre 2017 }
				\item { \textbf{Ora:} 15.00 }
				\item { \textbf{Partecipanti del gruppo:} Gruppo al completo }
				\item { \textbf{Partecipanti esterni:} Iks Srl }
			\end{itemize}
		
	
	\subsection{Argomenti affrontati}
    Nel corso della riunione sono stati approfonditi i punti fondamentali per la realizzazione del progetto.
    In particolare, è stato affrontato:

    \begin{itemize}
	\item { Requisiti obbligatori,opzionali e desiderabili }
	\item { Funzionamento delle trace }
	\item { Use case }
	\item { Tecnologie obbligatorie per il progetto}
     \end{itemize}
 
	Nella riunione è stato specificato che il capitolato è composto da due parti principali da realizzare:
			\begin{itemize}
				\item { \textbf{Mappa dell'applicazione:} è un plugin che permette di rappresentare in modo grafico i dati raccolti nell'applicazione. In particolare si vuole disegnare i vari componenti che compongono l'applicazione monitorata, far visualizzare all'utilizzatore i tempi medi di risposta nell'effettuare una richiesta da un nodo all'altro.
				I nodi devono essere collegati da una linea contenente una freccia che mediante il verso indica la direzione della richiesta da un nodo all'altro, il colore della freccia deve cambiare diventando rossa se il tempo delle richieste in quel punto della piattaforma e superiore alla norma, verde altrimenti.
				I componenti della mappa devono poter essere riposizionabili sia in modo manuale che automatico.
				Per quanto riguarda il modo manuale, l'idea è quella di poter muovere mediante click del mouse i vari componenti disegnati sulla mappa, mentre automaticamente, dev'essere presente un apposti bottone, che una volta cliccato, fa ritornare in una posizione grafica di default tutti i nodi della piattaforma. 
				Inoltre, visto che un'applicazione aziendale non è installata in un solo server, dev'essere possibile visualizzare il numero di server che compongono un cluster.
				Il plugin dev'essere sviluppato in linguaggio AngularJs.	
				}
				\item { \textbf{Call Tree:} è un plugin che rappresenta in modo grafico e gerarchico i metodi chiamati dall'applicazione quando viene effettuata una richiesta.
				L'idea è di rappresentare i metodi da quello radice fino alla query che viene eseguita nel database così da ottenere i dati necessari per fornire una risposta all'utente. 
			    Per ogni metodo è stato richiesto che vi sia presente anche il tempo di esecuzione dello stesso.
		        Per quanto riguarda la lista delle query eseguite (traces) dev'essere possibile visualizzarle in ordine cronologico di esecuzione e dev'essere presente l'identificativo associato ad ognuna.
		         
	    }
		
			\end{itemize}
		
		E' stato richiesto che vi sia un disaccoppiamento fra logica applicativa e le tecnologie utilizzate, in questo modo è possibile cambiare, per esempio, la base di dati dove le informazioni vengono memorizzate senza dover rivoluzionare il software.
		
		
		
		
		
			
	
	
	
\end{document}