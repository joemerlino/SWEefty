\subsection{Caso d'uso UC1: Visualizzazione mappa topologica}
\begin{itemize}
\item \textbf{Attori}: Utente
\item \textbf{Descrizione}: L'attore intende visualizzare la mappa topologica dell'applicazione monitorata.
\item \textbf{Precondizione}: Kibana deve avere una dashboard che contenga il plugin della mappa.
\item \textbf{Flusso principale degli eventi}: L'utente richiede di visualizzare la mappa dell'applicazione monitorata 
Il plugin carica ed elabora i dati relativi all'applicazione monitorata
Il plugin mostra la mappa topologica dell'applicazione monitorata
\item \textbf{Postcondizione}: Viene mostrata la mappa topologica associata all'applicazione monitorata.
\end{itemize}
\subsection{Caso d'uso UC2: Visualizzazione stack trace}
\begin{itemize}
\item \textbf{Attori}: Utente
\item \textbf{Descrizione}: L'utente intende visualizzare la lista delle trace dell'applicazione monitorata. 
\item \textbf{Precondizione}: Kibana deve avere una dashboard che contenga il plugin della stack trace.
\item \textbf{Flusso principale degli eventi}: L'utente richiede a Kibana di visualizzare la lista delle trace dell'applicazione monitorata
Il plugin carica ed elabora i dati relativi alle traces
Il plugin mostra all'utente la lista delle traces
\item \textbf{Postcondizione}: Viene mostrato l'elenco delle traces.
\end{itemize}
\subsection{Caso d'uso UC3: Call tree}
\begin{itemize}
\item \textbf{Attori}: Utente
\item \textbf{Descrizione}: L'attore intende visualizzare il call tree di una trace.
\item \textbf{Precondizione}: Deve essere stato caricato da Kibana il plugin della lista delle traces.
\item \textbf{Flusso principale degli eventi}: L'attore seleziona una trace dalla lista di quelle disponibili nella lista delle traces
Viene visualizzata la call tree della trace
\item \textbf{Postcondizione}: Viene visualizzato il call tree della trace.
\item \textbf{Estensioni}:
\begin{itemize}
\item Visualizzazione stack trace (UC2)
\end{itemize}
\end{itemize}
\subsection{Caso d'uso UC4: Queries di tutte le traces}
\begin{itemize}
\item \textbf{Attori}: Utente
\item \textbf{Descrizione}: L'attore intende visualizzare la lista delle queries eseguite dall'applicazione monitorata ai databases.
\item \textbf{Precondizione}: Deve essere stato caricato da Kibana il plugin della lista delle traces.
\item \textbf{Flusso principale degli eventi}: L'attore richiede di visualizzare la lista delle queries eseguite dall'applicazione monitorata
Vengono visualizzate tutte le queries eseguite
\item \textbf{Scenari alternativi}: Se non sono presenti queries viene visualizzato un messaggio che informi l'attore del fatto che non sono presenti queries.
\item \textbf{Postcondizione}: Viene visualizzata la lista delle queries eseguite dalle traces ai databases.
\item \textbf{Estensioni}:
\begin{itemize}
\item Visualizzazione stack trace (UC2)
\end{itemize}
\end{itemize}
\subsection{Caso d'uso UC5: Queries singolo call tree}
\begin{itemize}
\item \textbf{Attori}: Utente
\item \textbf{Descrizione}: L'attore intende visualizzare le queries eseguite in un call tree.
\item \textbf{Precondizione}: Deve essere stato caricato da Kibana il call tree di una trace.
\item \textbf{Flusso principale degli eventi}: L'attore richiede di visualizzare la lista delle queries eseguire all'interno di un call tree
La lista delle queries ai databases eseguiti dal call tree viene visualizzato
\item \textbf{Scenari alternativi}: Se non sono presenti queries viene visualizzato un messaggio che informi l'attore del fatto che non sono presenti queries.
\item \textbf{Postcondizione}: Viene visualizzato l'elenco delle queries eseguite all'interno di un call tree.
\item \textbf{Estensioni}:
\begin{itemize}
\item Call tree (UC3)
\end{itemize}
\end{itemize}
\subsection{Caso d'uso UC6: Riposizionamento componente}
\begin{itemize}
\item \textbf{Attori}: Utente
\item \textbf{Descrizione}: L'attore intende spostare un componente all'interno della mappa topologica dell'applicazione monitorata.
\item \textbf{Precondizione}: Deve essere stato caricato da Kibana il plugin della mappa topologica dell'applicazione.
\item \textbf{Flusso principale degli eventi}: L'attore trascina un componente della mappa topologica con il puntatore
Il componente trascinato viene riposizionato
\item \textbf{Postcondizione}: Il componente trascinato viene riposizionato
\item \textbf{Estensioni}:
\begin{itemize}
\item Visualizzazione mappa topologica (UC1)
\end{itemize}
\end{itemize}
