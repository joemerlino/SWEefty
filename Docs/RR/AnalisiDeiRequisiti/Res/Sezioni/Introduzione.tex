\section{Introduzione}
	\subsection{Scopo del documento}
	Il presente documento nasce con l'intento di analizzare, specificare e classificare i requisiti e i casi d'uso che sono stati individuati grazie all'analisi del capitolato d'appalto C7 "OpenAPM" e agli incontri con il proponente, l'azienda IKS.\\
	Questo documento rappresenta un vincolo tra il fornitore, che si impegna a sviluppare un software conforme alle caratteristiche riportate di seguito, e il proponente, che riconosce tali requisiti come le caratteristiche ricercate.\\
	In fase di collaudo la conformità ai requisiti concordati costituirà il criterio per l'accettazione del prodotto da parte del committente.

	\subsection{Scopo del prodotto}
	Il prodotto che SWEefty è tenuto a realizzare consiste in una coppia di plugin per Kibana che devono fornire due funzionalità fondamentali:
	\begin{itemize}
		\item \textbf{Visualizzazione mappa topologica del sistema:} il plugin deve visualizzare in maniera chiara ed intuitiva come le componenti del sistema interagiscono tra di loro, con annesse informazioni utili;
		\item \textbf{Visualizzazione della stack trace:} vengono visualizzate sotto forma di lista le interazioni fra i componenti e le richieste HTTP effettuate ai server. Per ogni trace della lista inoltre verrà visualizzata la rispettiva call tree e le queries effettuate ai database.
	\end{itemize}

	\subsection{Glossario}
		Volendo evitare incomprensioni  ed equivoci per rendere la lettura del documento più semplice e chiara viene allegato il \emph{Glossario v1.0.0} nel quale sono contenute le definizioni dei termini tecnici, dei vocaboli ambigui, degli acronimi e delle abbreviazioni. Questi termini sono evidenziati nel presente documento con una g al pedice (esempio: $Glossario_{G}$).
	\subsection{Riferimenti}
	\subsubsection{Normativi}
	\begin{itemize}
		\item \textbf{Norme di Progetto:} \emph{Norme di Progetto v1.0.0};
		\item \textbf{IEEE Std 830-1998} - IEEE Recommended Practice for Software Requirements Specifications;
		\item \textbf{Capitolato d'appalto C7:} OpenAPM: cruscotto di Application Performance Management \\ \url{http://www.math.unipd.it/~tullio/IS-1/2017/Progetto/C7.pdf};
		\item \textbf{Verbali di incontro esterni} con il proponente IKS: \begin{itemize}
			\item \emph{VE\textunderscore2017-12-06};
			\item \emph{VE\textunderscore2017-12-19}.
		\end{itemize}
		\item \textbf{Verbali di incontro interni} con i componenti del gruppo:
		\begin{itemize}
			\item \emph{VI\textunderscore2017-12-07};
			\item \emph{VI\textunderscore2017-12-20}.
		\end{itemize}
	\end{itemize}
	
	\subsubsection{Informativi}
	\begin{itemize}
		\item \textbf{Seminario tecnologico da parte dell'azienda IKS:} Data analysis con ElasticSearch e Kibana \\ \url{http://www.math.unipd.it/~tullio/IS-1/2017/Dispense/P03.pdf};
		\item \textbf{APM con Elastic Stack:} \url{https://www.elastic.co/guide/en/apm/get-started/current/overview.html};
		\item \textbf{Kibana User Guide:} \url{https://www.elastic.co/guide/en/kibana/current/index.html}.
	
	\end{itemize}
	