\section{Descrizione del Prodotto}
	\subsection{Prospettive sul prodotto}
	Il prodotto di questo progetto vuole permettere l'analisi di dati significativi del comportamento di un'applicazione monitorata attraverso due plugin per Kibana.
	È un prodotto correlato al settore degli APM, ed è indirizzato a coloro che devono amministrare applicazioni con conoscenze pregresse informatiche. Esso interagisce quindi con Kibana che permette la visualizzazione dei dati sullo stato e sulll'andamento dell'applicazione presi da ElasticSearch.
	\subsection{Funzioni del prodotto}
	I due plugin utilizzano ed elaborano i dati sullo stato e sull'andamento dell'applicazione da monitorare per ottenere:
	\begin{itemize}
		\item 	\textbf{una mappa topologica}: in essa vengono visualizzati i vari tipi di componenti attivi nel periodo di tempo selezionato con le relative interazioni e delle informazioni che possono aiutare a capire lo stato complessivo dell'applicazione. Inoltre è possibile interagire con la mappa modificando a piacimento la posizione dei componenti e la dimensione per rendere la visualizzazione della mappa più accomodante ai bisogni dell'utente.
		\item \textbf{una stack trace}: in essa vengono visualizzate tutte le richieste che sono state fatte da parte degli utenti dell'applicazione da monitorare con delle informazioni testuali relative al funzionamento tecnico di ognuna di essa. La lista può essere riordinata in base alle esigenze secondo le informazioni visualizzate.
		Inoltre, per ogni richiesta, è possibile visualizzare il relativo call tree: un albero delle funzioni e delle proprie sottochiamate che vengono eseguite dall'applicazione monitorata durante quella specifica richiesta. Per ogni call tree può essere visualizzato la lista delle sole queries eseguite dai metodi presenti e, anch'essa, può essere riorganizzata in base ai bisogni dell'utente.
	\end{itemize} 
	\subsection{Caratteristiche degli utenti}
	Il prodotto si rivolge essenzialmente a due categorie di utenti:
	\begin{itemize}
		\item \textbf{utente sviluppatore}: utilizza il prodotto implementandolo in una dashboard personalizzata in base alle esigenze di una determinata applicazione da monitorare. Si deve preoccupare dell'installazione dei plugin su Kibana.
		\item \textbf{utente finale}: utilizza il prodotto integrato in una dashboard Kibana organizzata appositamente per il monitoraggio di una specifica applicazione. Devono comunque avere una conoscenza di base sul funzionamento tecnico della loro applicazione per poter interpretare i dati e le informazioni.
	\end{itemize}
	\subsection{Vincoli generali}
	I plugin sviluppati devono essere compatibili  Kibana v.6.
	Per quanto riguarda i browser Kibana v.6 risulta compatibile con: 
	\begin{itemize}
		\item Google Chrome v.55
		\item Mozilla Firefix v.50
		\item Safari v.10
		\item  Explorer v.11
	\end{itemize}
	Dato che Kibana è sviluppata in JavaScript i relativi plugin hanno come vincolo lo sviluppo attraverso lo stesso linguaggio.
	
	\subsection{Assunzioni e dipendenze}
	Per utilizzare il prodotto deve essere installato Kibana e su di essa i due plugin sviluppati. 
	
