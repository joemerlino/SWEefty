\section{Checklist}
Questa checklist va utilizzata dal \emph{Verificatore} per controllare che un \emph{documento} sia conforme alle norme stabilite

\begin{enumerate}
	\item E' è un tumore. Cercare su google la e accentata maiuscole e usare quella
	\item Titoli dei documenti e ruoli di progetto vanno dentro il tag emph (NON TEXTTT!) e hanno lettere maiuscole delle parole "improtanti". Nota la differenza fra titolo del documento e file del documento
	\item Quando ci si riferisce ad un ruolo, esso va specificato al singolare. Es: il \emph{Verificatore} e non i \emph{Verificatori}
	\item Per ogni parola che va nel glossario \emph{per ogni documento} solo la prima occorrenza va nella math mode e poi messa nel glossario con la G maiuscola in fondo
	\item Gli strumenti vanno nel glossario
	\item I nomi dei files vanno in upper camel case
	\item I nomi dei files nei documenti, i comandi e i pezzi di codice(classi, metodi etc) vanno dentro il tag texttt: \texttt{NormeDiProgetto.tex}
	\item Controllo di italiano formale
	\begin{itemize}
		\item Tempo verbale: presente
		\item Parlare del gruppo in terza persona e non con il "noi":\\ \textcolor{red}{"Questa cosa è quella che \emph{usiamo} perchè blablabla"}\\
		\textcolor{green}{"Questa cosa è quella \emph{utilizzata} dal gruppo perchè blablabla"}
	\end{itemize}
	\item Usare lo spell checker di texstudio
	\item Ogni section,subsection, subsubsection, paragraph e subparagraph abbia un punto alla fine
	\item è stato centrato il punto della situazione? Nel senso, si sta parlando della cosa giusta?
	\item Le immagini, in particolare i diagrammi UML, rispecchiano ciò che rappresentano \emph{coerentemente}?
\end{enumerate}