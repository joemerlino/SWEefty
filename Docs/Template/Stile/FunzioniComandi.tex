% Creazione della copertina
\newcommand{\copertina}{
  \newgeometry{top=4cm}
  
  \begin{titlepage}
  \begin{center}

  \begin{center}
    %% qui metteteci l'immagine di copertina. Io ho messo quella dell'uni,
    %voi mettete quella del vostro grupo
    \centerline{\includegraphics[scale=0.24]{../../CommonImages/logo.jpg}}
  \end{center}
  
  \vspace{1cm}

  \begin{Huge}
    \textbf{\Titolo{}} \\
  \end{Huge}

  \vspace{9pt}  
  
  \begin{large}
  \Gruppo{}\ - \Data{}
  \end{large}	  
  
  \vspace{15pt}

  \bgroup
  \def\arraystretch{1.3}
   \centering
   \begin{tabular}{c|L{5cm}}
      \multicolumn{2}{c}{\textbf{Informazioni sul documento} } \\ \hline
      Versione &  \Versione{}\\
      Redazione & \ACapoRedazione{} \\
      Verifica & \Verifica{} \\ 
      Approvazione & \Approvazione{} \\
      Uso & \Uso \\
      Distribuzione & \Distribuzione{}
    \end{tabular}
  \egroup

  \vspace{15pt}

  \begin{center}
    \textbf{Descrizione\\}
    \DescrizioneDoc{}
  \end{center}

  \end{center}
  \end{titlepage}
  
  \restoregeometry
}

\newcommand{\code}[1]{\flextt{\texttt{#1}}}
