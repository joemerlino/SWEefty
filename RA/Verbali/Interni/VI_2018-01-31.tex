% - PER STILARE QUESTO DOCUMENTO VANNO INSERITI I DATI.
% - VISTO CHE IL FILE È LUNGO ED INTRICATO PER SEMPLIFICARE HO MESSO DEI COMMENTI
% - CHE INDICANO I PUNTI DOVE INSERIRE LE INFORMAZIONI CHE MANCANO.
% - INSERISCILI IN CORRISPONDENZA DEI COMMENTI:

% - "INSERT TITOLO"
% - "INSERT USO"
% - "INSERT DATA"
% - "INSERT DESCRIZIONE"
% - "INSERT INFORMAZIONI INCONTRO"
% - "INSERT DOMANDE E RISPOSTE" (intanto lo strutturiamo così, con domande e rispose, se servono altri tipi di struttura cambieremo)

% - "INSERT TITOLO" (verosimilmente sarà una cosa tipo "Verbale Esterno/Interno del xx/yy") 
\newcommand{\Titolo}{Verbale riunione 31 Gennaio 2018}

\newcommand{\Gruppo}{SWEefty}

\newcommand{\Redazione}{Elia Montecchio}

\newcommand{\ACapoRedazione}{Elia Montecchio}


\newcommand{\Distribuzione}{Prof.Tullio Vardanega \newline Prof.Riccardo Cardin \newline SWEefty}

\newcommand{\InformazioniDocumento}{}
% - "INSERT USO (INTERNO/ESTERNO)"
\newcommand{\Uso}{Interno}
% - "INSERT DATA"
\newcommand{\Data}{31 Gennaio 2018}
% - "INSERT DESCRIZIONE"
\newcommand{\DescrizioneDoc}{Questo documento riporta in modo formale il riassunto della rinuione tenutasi nella data sopra riportata.  }



\documentclass[a4paper, oneside, openany]{article}

% \usepackage{graphbox}
% permette di modificare i margini
\usepackage[top=3.1cm, bottom=3.1cm, left=2.2cm, right=2.2cm]{geometry}

\usepackage{lastpage} %info sul # dell'ultima pagina del documento
\usepackage{fancyhdr} %per modificare dimensioni,margini, intestazioni e righe a piè di pagina
\fancypagestyle{plain}{
  % cancella tutti i campi di intestazione e piè di pagina
  \fancyhf{}
  
  \lhead{\includegraphics[width=3cm]{../../../CommonImages/logo.jpg}}
  
  \rhead{sweeftyteam@gmail.com}
  
  \lfoot{ %piè di pagina
   {\Titolo} \ - \textit{{\Gruppo}}
  }
  \rfoot{Pagina \thepage{} di \pageref{LastPage}} %es: pag: 4 di 10

  %linea orizzontale alle posizioni top e bottom della pagina
  \renewcommand{\headrulewidth}{0.2	pt}  
  \renewcommand{\footrulewidth}{0.2pt}
}
\pagestyle{plain}

%Comando Spazio
\newcommand{\Spazio}{\mbox{} \\ \mbox{} \\ }  


%\usepackage{calc} %introduce la notazione infissa per le op. aritmetiche interne a LaTeX

\usepackage[utf8]{inputenc}
\usepackage[T1]{fontenc}
\usepackage[italian]{babel} %il documento è in italiano
%\usepackage{textcomp} %The pack­age sup­ports the Text Com­pan­ion fonts, which pro­vide many text sym­bols
%(such as baht, bul­let, copy­right, mu­si­cal­note, onequar­ter, sec­tion, and yen), in the TS1 en­cod­ing.

\usepackage{graphicx}       %permette di inserire delle immagini
\usepackage{caption}        %numerazione figure e loro descrizione testuale
\usepackage{subcaption}     %sottofigure numerabili
\usepackage{float}  %permette di inserire un # qualsiasi di figure fluttuanti
\usepackage{xcolor}
\usepackage{rotating} %permette di ruotare le immagini
%\usepackage{changepage} %utile se c'è bisogno di aggiustare margini per centrare figure

%package utili per la math mode ( $ ... $ o \[ ... \] )
\usepackage{amsmath}
\usepackage{amssymb}
\usepackage{amsfonts}
%\usepackage{euler}    %font 'ams euler', lo stesso di 'Concrete Mathematics' di Knuth
\usepackage{amsthm}
\usepackage{mathtools}

% package utili per tabelle(\thead in particolare)
\usepackage{array, booktabs, caption}
\usepackage{makecell}
\renewcommand\theadfont{\bfseries}
\usepackage{boldline}

\usepackage{listings} %permette di inserire degli spezzoni di codice

% \usepackage{tikz} %disegno di immagini vettoriali a schermo. Utile per grafi
% \usetikzlibrary{arrows.meta}
% \usetikzlibrary{graphs}
% \usetikzlibrary{arrows}
%\usepackage{tikz-uml} %serve per disgnare l'UML, fantastica guida:
%https://perso.ensta-paristech.fr/~kielbasi/tikzuml/var/files/doc/tikzumlmanual.pdf
%download package: http://perso.ensta-paristech.fr/~kielbasi/tikzuml/

%package per le tabelle
\usepackage{booktabs} %permette di poter usare delle liste nelle tabelle
\usepackage{tabularx} 
\usepackage{longtable} %una tabella può continuare su più pagine
\usepackage{multirow} %utile per visualizzare una cella su più righe
%\usepackage{multicolumn} %cella su più colonne
%\usepackage[table]{xcolor} %rende disponibile l'utilizzo di un colore per lo sfondo
                        %delle celle di una tabella

%crea una cella per le tabelle in grado di andare a capo con \newline
%https://tex.stackexchange.com/questions/12703/how-to-create-fixed-width-table-columns-with-text-raggedright-centered-raggedlef
\usepackage{array}
\newcolumntype{L}[1]{>{\raggedright\let\newline\\\arraybackslash\hspace{0pt}}m{#1}}
\newcolumntype{C}[1]{>{\centering\let\newline\\\arraybackslash\hspace{0pt}}m{#1}}
\newcolumntype{R}[1]{>{\raggedleft\let\newline\\\arraybackslash\hspace{0pt}}m{#1}}


%indice con i puntini
\usepackage{tocloft}
\renewcommand\cftsecleader{\cftdotfill{\cftdotsep}}

%http://ctan.mirror.garr.it/mirrors/CTAN/macros/latex/contrib/appendix/appendix.pdf
\usepackage{appendix} %aggiunge dei comandi per l'appendice
\usepackage{parskip} %aiuta LaTeX a trovare il miglior stile per i page break
\setcounter{secnumdepth}{5} % numera i sottoparagrafi
\setcounter{tocdepth}{5} %aggiunge all'indice i sottoparagrafi
%\usepackage{titlesec} %\begin{paragraph} si può usare come subsubsubsection!


\usepackage{breakurl}%\url{...} può continare alla linea successiva. (si può andare a capo)

\definecolor{Maroon}{cmyk}{0, 0.87, 0.68, 0.32}
\usepackage[colorlinks=true]{hyperref}
\hypersetup{
    colorlinks=true,
    citecolor=black,
    filecolor=black,
    linkcolor=black, % colore dei link interni
    urlcolor=Maroon  % colore dei link interniesterni
}

%impostazioni per il codice che deve finire dentro a
%\begin{lstlisting}

\definecolor{listinggray}{gray}{0.9}
\definecolor{lbcolor}{rgb}{0.9,0.9,0.9}
\lstset{
backgroundcolor=\color{lbcolor},
    tabsize=4,    
%   rulecolor=,
    language=[GNU]C++,
    basicstyle=\scriptsize,
    upquote=true,
    aboveskip={1.5\baselineskip},
    columns=fixed,
    showstringspaces=false,
    extendedchars=true,
    inputencoding=utf8,
    breaklines=true,
    prebreak = \raisebox{0ex}[0ex][0ex]{\ensuremath{\hookleftarrow}},
    frame=single,
    numbers=left,
    showtabs=false,
    showspaces=false,
    showstringspaces=false,
    identifierstyle=\ttfamily,
    keywordstyle=\color[rgb]{0,0,1},
    commentstyle=\color[rgb]{0.026,0.112,0.095},
    stringstyle=\color[rgb]{0.627,0.126,0.941},
    numberstyle=\color[rgb]{0.205, 0.142, 0.73},
%        \lstdefinestyle{C++}{language=C++,style=numbers}’.
}
\lstset{
  backgroundcolor=\color{lbcolor},
  tabsize=4,
  language=C++,
  captionpos=b,
  tabsize=3,
  frame=lines,
  numbers=left,
  numberstyle=\tiny,
  numbersep=5pt,
  breaklines=true,
  showstringspaces=false,
  basicstyle=\footnotesize,
  identifierstyle=\color{magenta},
  keywordstyle=\color[rgb]{0,0,1},
  commentstyle=\color{orange},
  stringstyle=\color{red}
}


 \newgeometry{top=4cm}

\begin{document}
\begin{titlepage}
	\begin{center}
		
		\begin{center}
			%% qui metteteci l'immagine di copertina. Io ho messo quella dell'uni,
			%voi mettete quella del vostro grupo
			\centerline{\includegraphics[scale=0.24]{../../../CommonImages/logo.jpg}}
		\end{center}
		
		\vspace{1cm}
		
		\begin{Huge}
			\textbf{\Titolo{}} \\
		\end{Huge}
		
		\vspace{9pt}  
		
		\begin{large}
			\Gruppo{}\ - \Data{}
		\end{large}	  
		
		\vspace{15pt}
		
		\bgroup
		\def\arraystretch{1.3}
		\centering
		\begin{tabular}{c|L{5cm}}
			\multicolumn{2}{c}{\textbf{\InformazioniDocumento{}} } \\ \hline
			Redazione & \ACapoRedazione{} \\

			Uso & \Uso \\
			Distribuzione & \Distribuzione{}
		\end{tabular}
		\egroup
		
		\vspace{15pt}
		
		\begin{center}
			\textbf{Descrizione\\}
			\DescrizioneDoc{}
		\end{center}
		
	\end{center}
\end{titlepage}

\restoregeometry
	
	\section{Informazioni generali}
		\subsection{Informazioni incontro}
			% - INSERT INFORMAZIONI INCONTRO
			\begin{itemize}
				\item { \textbf{Luogo:} Torre Archimede }
				\item { \textbf{Data:} 31 Gennaio 2018 }
				\item { \textbf{Ora Inizio:} 15.30 }
					\item { \textbf{Ora Fine:} 17.15 }
				\item { \textbf{Partecipanti del gruppo:} Gruppo al completo }
				\item { \textbf{Partecipanti esterni:} Nessuno }
			\end{itemize}
		
	
	\subsection{Argomenti affrontati}
    Nel corso della riunione sono state affrontate le varie problematiche emerse dopo la correzione dei documenti presentati alla prima revisione del progetto. È stato pertanto necessario organizzare il lavoro in modo tale da porre rimedio alle lacune evidenziate dal docente.
    
	\subsubsection{Organizzazione del lavoro}
    \begin{itemize}
    \item  \textbf{Norme di Progetto:} Riguardo alle lacune evidenziate nel documento \emph{NormeDiProgetto v1.0.0} si è deciso di:
    	\begin{itemize}
    	\item  \textbf{ 01/VI\_2018-01-31 } Aggiungere l'attività di Pianificazione descrivendola e di inserire all'interno Piano di progetto e di qualifica, task assegnato ad Elia Montecchio;
    	
    	\item  \textbf{ 02/VI\_2018-01-31 } Ampliare la sezione dei processi organizzativi e di supporto inserendo validazione,manutezione dei processi e formazione, task assegnato a Francesco Parolini;
    	
    	\item  \textbf{ 03/VI\_2018-01-31 } Trasformare l'attività di documentazione in un processo aggiungendo l'attività di stesura e aggiornamento di un documento, task assegnato a Davide Zago;
    		
       \end{itemize}
   
   
    \item  \textbf{Analisi dei Requisiti:} Riguardo alle lacune evidenziate nel documento \emph{AnalisiDeiRequisiti v1.0.0} si è deciso di:
   \begin{itemize}
   	\item  \textbf{ 04/VI\_2018-01-31 } Rivedere i casi d'uso segnalati in modo da ricrearli opportunamente,
   	 aggiungendo casi d'uso generali infine è stato ritenuto oppurturno riordinarli. Il compito è assegnato a Lisa Parma e Joe Merlino;
   	
   	
   \end{itemize}


    \item  \textbf{Piano di progetto:} Riguardo alle lacune evidenziate nel documento \emph{PianoDiProgetto v1.0.0} si è deciso di:
     \begin{itemize}
    	\item  \textbf{ 05/VI\_2018-01-31 } Rivedere le sezioni del documento che presentavano difetti correggendole in modo opportuno, rivedere la pianificazione futura in modo da inserire le ore di codifica necessarie per sviluppare il Proof of concept. Il task è stato assegnato a Paolo Eccher;
    	
	
      \end{itemize}
  
   \item  \textbf{ 06/VI\_2018-01-31 } Si è deciso di riorganizzare il diario secondo le segnalazioni del docente, task assegnato ad Elia Montecchio.
   
    \item  \textbf{ 07/VI\_2018-01-31 } È necessario introdurre nel documento \emph{NormeDiProgetto v2.0.0} il codice dei verbali che verrà assegnato per ogni decisione dei verbali.
   
   \item  \textbf{ 08/VI\_2018-01-31 } Si dovrà realizzare una piccola demo del plugin per poi illustrarne il funzionamento agli altri membri del team. Se ne occuperanno Alberto Gallinaro e Paolo Eccher.
   
   \item  \textbf{ 09/VI\_2018-01-31 } È prevista, per ogni membro del gruppo, un'attività di formazione autonoma seguendo dei video messi a disposizione, inerenti le tecnologie necessarie per lo sviluppo del progetto.
   
   
    

    \end{itemize}
 

	
\end{document}