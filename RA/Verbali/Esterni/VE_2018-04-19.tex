% - PER STILARE QUESTO DOCUMENTO VANNO INSERITI I DATI.
% - VISTO CHE IL FILE È LUNGO ED INTRICATO PER SEMPLIFICARE HO MESSO DEI COMMENTI
% - CHE INDICANO I PUNTI DOVE INSERIRE LE INFORMAZIONI CHE MANCANO.
% - INSERISCILI IN CORRISPONDENZA DEI COMMENTI:

% - "INSERT TITOLO"
% - "INSERT USO"
% - "INSERT DATA"
% - "INSERT DESCRIZIONE"
% - "INSERT INFORMAZIONI INCONTRO"
% - "INSERT DOMANDE E RISPOSTE" (intanto lo strutturiamo così, con domande e rispose, se servono altri tipi di struttura cambieremo)

% - "INSERT TITOLO" (verosimilmente sarà una cosa tipo "Verbale Esterno/Interno del xx/yy") 
\newcommand{\Titolo}{Verbale riunione 19 Aprile 2018}

\newcommand{\Gruppo}{SWEefty}


\newcommand{\Redazione}{Elia Montecchio}

\newcommand{\ACapoRedazione}{Elia Montecchio}


\newcommand{\Distribuzione}{Prof.Tullio Vardanega \newline Prof.Riccardo Cardin \newline SWEefty \newline IKS S.r.l.}

\newcommand{\InformazioniDocumento}{}
% - "INSERT USO (INTERNO/ESTERNO)"
\newcommand{\Uso}{Esterno}
% - "INSERT DATA"
\newcommand{\Data}{19 Aprile 2018}
% - "INSERT DESCRIZIONE"
\newcommand{\DescrizioneDoc}{Questo documento riporta in modo formale il riassunto della rinuione tenutasi nella data sopra riportata.  }



\documentclass[a4paper, oneside, openany]{article}

% \usepackage{graphbox}
% permette di modificare i margini
\usepackage[top=3.1cm, bottom=3.1cm, left=2.2cm, right=2.2cm]{geometry}

\usepackage{lastpage} %info sul # dell'ultima pagina del documento
\usepackage{fancyhdr} %per modificare dimensioni,margini, intestazioni e righe a piè di pagina
\fancypagestyle{plain}{
  % cancella tutti i campi di intestazione e piè di pagina
  \fancyhf{}
  
  \lhead{\includegraphics[width=3cm]{../../../CommonImages/logo.jpg}}
  
  \rhead{sweeftyteam@gmail.com}
  
  \lfoot{ %piè di pagina
   {\Titolo} \ - \textit{{\Gruppo}}
  }
  \rfoot{Pagina \thepage{} di \pageref{LastPage}} %es: pag: 4 di 10

  %linea orizzontale alle posizioni top e bottom della pagina
  \renewcommand{\headrulewidth}{0.2	pt}  
  \renewcommand{\footrulewidth}{0.2pt}
}
\pagestyle{plain}

%Comando Spazio
\newcommand{\Spazio}{\mbox{} \\ \mbox{} \\ }  


%\usepackage{calc} %introduce la notazione infissa per le op. aritmetiche interne a LaTeX

\usepackage[utf8]{inputenc}
\usepackage[T1]{fontenc}
\usepackage[italian]{babel} %il documento è in italiano
%\usepackage{textcomp} %The pack­age sup­ports the Text Com­pan­ion fonts, which pro­vide many text sym­bols
%(such as baht, bul­let, copy­right, mu­si­cal­note, onequar­ter, sec­tion, and yen), in the TS1 en­cod­ing.

\usepackage{graphicx}       %permette di inserire delle immagini
\usepackage{caption}        %numerazione figure e loro descrizione testuale
\usepackage{subcaption}     %sottofigure numerabili
\usepackage{float}  %permette di inserire un # qualsiasi di figure fluttuanti
\usepackage{xcolor}
\usepackage{rotating} %permette di ruotare le immagini
%\usepackage{changepage} %utile se c'è bisogno di aggiustare margini per centrare figure

%package utili per la math mode ( $ ... $ o \[ ... \] )
\usepackage{amsmath}
\usepackage{amssymb}
\usepackage{amsfonts}
%\usepackage{euler}    %font 'ams euler', lo stesso di 'Concrete Mathematics' di Knuth
\usepackage{amsthm}
\usepackage{mathtools}

% package utili per tabelle(\thead in particolare)
\usepackage{array, booktabs, caption}
\usepackage{makecell}
\renewcommand\theadfont{\bfseries}
\usepackage{boldline}

\usepackage{listings} %permette di inserire degli spezzoni di codice

% \usepackage{tikz} %disegno di immagini vettoriali a schermo. Utile per grafi
% \usetikzlibrary{arrows.meta}
% \usetikzlibrary{graphs}
% \usetikzlibrary{arrows}
%\usepackage{tikz-uml} %serve per disgnare l'UML, fantastica guida:
%https://perso.ensta-paristech.fr/~kielbasi/tikzuml/var/files/doc/tikzumlmanual.pdf
%download package: http://perso.ensta-paristech.fr/~kielbasi/tikzuml/

%package per le tabelle
\usepackage{booktabs} %permette di poter usare delle liste nelle tabelle
\usepackage{tabularx} 
\usepackage{longtable} %una tabella può continuare su più pagine
\usepackage{multirow} %utile per visualizzare una cella su più righe
%\usepackage{multicolumn} %cella su più colonne
%\usepackage[table]{xcolor} %rende disponibile l'utilizzo di un colore per lo sfondo
                        %delle celle di una tabella

%crea una cella per le tabelle in grado di andare a capo con \newline
%https://tex.stackexchange.com/questions/12703/how-to-create-fixed-width-table-columns-with-text-raggedright-centered-raggedlef
\usepackage{array}
\newcolumntype{L}[1]{>{\raggedright\let\newline\\\arraybackslash\hspace{0pt}}m{#1}}
\newcolumntype{C}[1]{>{\centering\let\newline\\\arraybackslash\hspace{0pt}}m{#1}}
\newcolumntype{R}[1]{>{\raggedleft\let\newline\\\arraybackslash\hspace{0pt}}m{#1}}


%indice con i puntini
\usepackage{tocloft}
\renewcommand\cftsecleader{\cftdotfill{\cftdotsep}}

%http://ctan.mirror.garr.it/mirrors/CTAN/macros/latex/contrib/appendix/appendix.pdf
\usepackage{appendix} %aggiunge dei comandi per l'appendice
\usepackage{parskip} %aiuta LaTeX a trovare il miglior stile per i page break
\setcounter{secnumdepth}{5} % numera i sottoparagrafi
\setcounter{tocdepth}{5} %aggiunge all'indice i sottoparagrafi
%\usepackage{titlesec} %\begin{paragraph} si può usare come subsubsubsection!


\usepackage{breakurl}%\url{...} può continare alla linea successiva. (si può andare a capo)

\definecolor{Maroon}{cmyk}{0, 0.87, 0.68, 0.32}
\usepackage[colorlinks=true]{hyperref}
\hypersetup{
    colorlinks=true,
    citecolor=black,
    filecolor=black,
    linkcolor=black, % colore dei link interni
    urlcolor=Maroon  % colore dei link interniesterni
}

%impostazioni per il codice che deve finire dentro a
%\begin{lstlisting}

\definecolor{listinggray}{gray}{0.9}
\definecolor{lbcolor}{rgb}{0.9,0.9,0.9}
\lstset{
backgroundcolor=\color{lbcolor},
    tabsize=4,    
%   rulecolor=,
    language=[GNU]C++,
    basicstyle=\scriptsize,
    upquote=true,
    aboveskip={1.5\baselineskip},
    columns=fixed,
    showstringspaces=false,
    extendedchars=true,
    inputencoding=utf8,
    breaklines=true,
    prebreak = \raisebox{0ex}[0ex][0ex]{\ensuremath{\hookleftarrow}},
    frame=single,
    numbers=left,
    showtabs=false,
    showspaces=false,
    showstringspaces=false,
    identifierstyle=\ttfamily,
    keywordstyle=\color[rgb]{0,0,1},
    commentstyle=\color[rgb]{0.026,0.112,0.095},
    stringstyle=\color[rgb]{0.627,0.126,0.941},
    numberstyle=\color[rgb]{0.205, 0.142, 0.73},
%        \lstdefinestyle{C++}{language=C++,style=numbers}’.
}
\lstset{
  backgroundcolor=\color{lbcolor},
  tabsize=4,
  language=C++,
  captionpos=b,
  tabsize=3,
  frame=lines,
  numbers=left,
  numberstyle=\tiny,
  numbersep=5pt,
  breaklines=true,
  showstringspaces=false,
  basicstyle=\footnotesize,
  identifierstyle=\color{magenta},
  keywordstyle=\color[rgb]{0,0,1},
  commentstyle=\color{orange},
  stringstyle=\color{red}
}


 \newgeometry{top=4cm}

\begin{document}
\begin{titlepage}
	\begin{center}
		
		\begin{center}
			%% qui metteteci l'immagine di copertina. Io ho messo quella dell'uni,
			%voi mettete quella del vostro grupo
			\centerline{\includegraphics[scale=0.24]{../../../CommonImages/logo.jpg}}
		\end{center}
		
		\vspace{1cm}
		
		\begin{Huge}
			\textbf{\Titolo{}} \\
		\end{Huge}
		
		\vspace{9pt}  
		
		\begin{large}
			\Gruppo{}\ - \Data{}
		\end{large}	  
		
		\vspace{15pt}
		
		\bgroup
		\def\arraystretch{1.3}
		\centering
		\begin{tabular}{c|L{5cm}}
			\multicolumn{2}{c}{\textbf{\InformazioniDocumento{}} } \\ \hline
			Redazione & \ACapoRedazione{} \\

			Uso & \Uso \\
			Distribuzione & \Distribuzione{}
		\end{tabular}
		\egroup
		
		\vspace{15pt}
		
		\begin{center}
			\textbf{Descrizione\\}
			\DescrizioneDoc{}
		\end{center}
		
	\end{center}
\end{titlepage}

\restoregeometry
	
	\section{Informazioni generali}
		\subsection{Informazioni incontro}
			% - INSERT INFORMAZIONI INCONTRO
			\begin{itemize}
				\item { \textbf{Luogo:} Torre Archimede  }
				\item { \textbf{Data:} 19 Aprile 2018 }
				\item { \textbf{Ora inizio:} 17.30 }
				\item { \textbf{Ora fine:} 17.50 }
				\item { \textbf{Partecipanti del gruppo:} Gruppo al completo }
				\item { \textbf{Partecipanti esterni:} IKS Srl }
			\end{itemize}
		
	
	\subsection{Argomenti affrontati}
    Nel corso della riunione Skype è stato mostrato all'azienda lo stato dei plugin e sono state discusse alcune problematiche emerse durante lo sviluppo del software.
    

 
	Nello specifico sono stati illustrati al proponente i vari incrementi effettuati alla stack trace e alla mappa topologica, i due plugin si presentano in un'unica pagina ma in seguito verranno divisi.\newline
	Il proponente come prima cosa chiede se si può interagire con la mappa topologica e se nel caso vi siano più nodi nel sistema, rispetto a quelli presenti, questi vengano visualizzati nella mappa.
	Viene quindi mostrato che è possibile ingrandire e rimpicciolire la mappa, spostare ciascun nodo visualizzato e spostare interamente la mappa
	Inoltre viene chiarito al proponente che la mappa è dinamica dunque nel caso venissero aggiunti nodi al sistema, importando appositi documenti in Elasticsearch, questi verranno visualizzati. 
	In seguito viene spiegato al proponente che il tempo medio di risposta che viene visualizzato nella linea che unisce due nodi del sistema è anch'esso dinamico e viene calcolato in un modo specifico.
	
	Successivamente viene portato alla luce un problema riguardante la visualizzazione dei server cluster all'interno della mappa topologica, infatti il gruppo non riesce a capire quali sono i dati da utilizzare per rappresentare questo tipo di nodo.
	Assieme al proponente si è cercato di sviscerare il problema andando effettivamente a controllare i vari documenti presenti all'interno di ElasticSearch effettuando delle query.   
	In seguito ad una verifica accurata del proponente è emerso che non è presente alcuna informazione riguardante i server cluster all'interno del sistema, in quanto il sistema che si sta monitorando attualmente registra solo chiamate da client a server e non vi sono cluster.
	Per questo motivo si è deciso \textbf{01/VE\_2018-04-19} di spostare il requisito riguardo questo funzionalità da obbligatorio a desiderabile.
	
	Il proponente in seguito chiede al team di poter visionare lo stato e il funzionamento del secondo plugin: la stack trace.
	Viene illustrata la stack trace che si presenta come una tabella dove ogni riga rappresenta una specifica richiesta effettuata da un utente al sistema dell'applicazione monitorata. Ogni richiesta ha un nome, un tempo di esecuzione e indica se durante l'esecuzione si è presentato un errore. Cliccando su ogni richiesta viene visualizzata una estensione della trace nella quale sono visibili informazioni più dettagliate. 
	Nello specifico vengono visualizzate informazioni riguardanti il call tree ovvero l'insieme di tutte le funzioni chiamate dal sistema per gestire la richiesta dell'utente, per ogni funzione viene visualizzato il self time e total time. Si può visualizzare anche la query list dove sono indicate tutte le query effettuate nel database per fornire una risposta alla richiesta dell'utente, per ogni query viene visualizzato il nome del database a cui è stata fatta la richiesta, il tempo di esecuzione e uno timestamp.
	Il proponente chiede come sono stati ottenuti i dati relativi ad ogni trace,viene dunque spiegato che i JSON memorizzati su Elasticsearch che contengono le varie informazioni ottenute dall’interazione dell’utente con l’applicazione monitorata possono essere di tre tipi: HTTP, PAGELOAD e JDBC. Tutte le trace costruite contengono dati di un JSON con type “HTTP”. A queste possono essere aggiunti dei dati dai JSON con type “pageload” e “JDBC” se il campo “id” risulta uguale.
	
	Il proponente risulta essere soddisfatto del lavoro svolto finora e dopo un confronto si è deciso di procedere nel seguente modo:
	    \begin{itemize}
	    \item { \textbf{02/VE\_2018-04-19} Il team dovrà inviare al proponente degli screenshot inerenti al lavoro svolto; }
	    \item { \textbf{03/VE\_2018-04-19} Il team dovrà inviare al proponente una mail presentando il link alla repository github che contiene il codice del prodotto. }	
	    	
	    \end{itemize}
	
	
	
	
	
	
	

		

\end{document}