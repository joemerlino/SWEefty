\subsection{Dati}
\label{sec:dati}

\begin{figure}[H]
    \centering
    \includegraphics[width=1\textwidth]{Images/classiDati.png}
    \caption{Diagramma delle classi \gl{UML} che rappresenta la struttura di una risposta di Elasticsearch}
    \label{img:dati}
\end{figure}

\subsubsection{Generalità}
Sia il lato server che il lato client devono elaborare ed utilizzare dati presenti all'interno del database Elasticsearch. In questa sezione viene esplicitata la forma che essi hanno.

\subsubsection{Rappresentazione}
Quello che viene restituito come risposta ad una interrogazione ad Elasticsearch è un documento in formato \gl{JSON}. Le classi in Figura \ref{img:dati} sono una rappresentazione di tale documento JSON, che dunque non ha metodi ed ha tutti gli attributi pubblici.

\subsubsection{Componenti}

\paragraph{ElasticsearchIndexResult} \Spazio
\label{sec:elasticsearchIndexResult}
Rappresenta il risultato \emph{grezzo} di una risposta del database Elasticsearch ad un client che esegua una query \texttt{\_search} su un determinato \emph{indice}. Il suo contenuto è quello presentato da \ref{sec:shards} a \ref{sec:RequestDB}, come si può vedere dalla figura \ref{img:dati}.

\paragraph{Shards} \Spazio
\label{sec:shards}
Rappresenta il campo dati \texttt{\_shards} all'interno di una risposta di Elasticsearch. Sostanzialmente indica il numero di shards all'interno dell'indice, quanti sono stati letti, quanti sono stati ignorati e se ci sono stati dei fallimenti della lettura degli shards. Per una descrizione più esaustiva si rimanda alla documentazione ufficiale di Elasticsearch:\\ \href{https://www.elastic.co/guide/en/elasticsearch/guide/current/replica-shards.html}{https://www.elastic.co/guide/en/elasticsearch/guide/current/replica-shards.html}.

\paragraph{Hits} \Spazio
Rappresenta il campo dati di una risposta di Elasticsearch contenente i documenti ritornati dal database e informazioni amministrative. Esse sono contenute nei campi \texttt{total}, che indica il numero di documenti ritornati nella risposta, e \texttt{max\_score}. I documenti ritornati sono contenuti nell'array \texttt{hits}.

\paragraph{ElasticsearchDocument} \Spazio
Rappresenta una \emph{entry} dell'array \texttt{hits}. É composto da informazioni amministrative quali:
\begin{itemize}
	\item \texttt{\_index}, che indica a che indice appartiene il documento;
	\item \texttt{\_type}, che indica che type ha il documento;
	\item \texttt{\_id}, che indica l'id del documento all'interno di Elasticsearch; 
	\item \texttt{\_score}, che indica la rilevanza del documento.
\end{itemize}
Inoltre c'è un campo dati \texttt{\_source} che contiene al suo interno il documento vero e proprio.

\paragraph{Request} \Spazio
Rappresenta il contenuto vero e proprio dei documenti salvati sull'istanza di Elasticsearch. Essendo Elasticsearch un database non relazionale non è possibile conoscere a priori la struttura di tale documento, ma nell'applicazione che andremo a descrivere con i plugin i documenti possono avere tre differenti formati, descritti nelle sezioni: \ref{sec:requestHTTP}, \ref{sec:requestPageload} e \ref{sec:RequestDB}. Il fatto che un documento possa avere una di queste tre forme è rappresentato in Figura \ref{img:dati} tramite un'interfaccia. La tipologia di \texttt{Request} è riconoscibile da un campo dati all'interno del documento: il campo \texttt{type}. 
\paragraph{RequestHTTP} \Spazio
\label{sec:requestHTTP}
Rappresenta il tracciamento di una richiesta di tipo HTTP eseguita all'applicazione monitorata da un browser.
\textcolor{red}{DESCRIVERE I CAMPI DATI}

\paragraph{RequestPageload}\Spazio
\label{sec:requestPageload}
Rappresenta il tracciamento di una richiesta di caricamento di una pagina. È sempre associata ad una richiesta HTTP principale grazie al campo \texttt{trace\_id} che risulta uguale in entrambe. 
\textcolor{red}{DESCRIZIONE DEI CAMPI DATI DELLA RICHIESTA E COSA RAPPRESENTA}

\paragraph{RequestDB}\Spazio
\label{sec:RequestDB}
Rappresenta il tracciamento di una richiesta che è stata eseguita su un database all'interno dell'applicazione monitorata. Anche questo tipo di richiesta è associata un una richiesta HTTP in quanto il testo della query eseguita è riportata anche tra i metodi del campo \texttt{call\_tree} della richiesta HTTP.
\textcolor{red}{DESCRIZIONE DEI CAMPI DATI DELLA RICHIESTA E COSA RAPPRESENTA}
