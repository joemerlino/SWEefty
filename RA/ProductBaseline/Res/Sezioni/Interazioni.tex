
\section{Interazioni fra componenti}
\label{sec:Interazioni}
\subsection{Costruzione del grafo della mappa topologica}

\begin{figure}[H]
	\centering
	\includegraphics[width=1\textwidth]{Images/DiagrammaSequenzaGraph.png}
	\caption{Diagramma di Sequenza UML rappresentante la costruzione del grafo della mappa topologica}
	\label{img:seqGraph}
\end{figure}

Un'istanza di \texttt{GraphBuilder} richiama il metodo \texttt{readData()} su un oggetto \texttt{DataReader} che si occupa prima di selezionare gli indici dell'istanza di Elasticsearch che hanno al loro interno delle traces tramite la funzione \texttt{tracesIndices()} e successivamente di effettuare una richiesta per ogni indice recuperato al fine di recupera i dati associati.
Ricevuti i dati, viene invocato il metodo \texttt{cleanData()} passandogli come parametro i dati, che si occupa della pulizia dei dati superflui provenienti da Elasticsearch tramite la funzione \texttt{removeDataFromIndices()}. Tramite la funzione \texttt{clean()} invece vengono puliti i dati superflui alla costruzione della mappa topologica attraverso un'istanza di \texttt{GraphCleaner}.
In seguito alla pulizia dei dati, l'istanza di \texttt{GraphBuilder} costruisce l'insieme dei nodi che faranno parte della mappa topologica tramite \texttt{buildNodes()} e l'insieme di collegamenti che ci saranno tra i nodi del grafo tramite \texttt{buildLinks()}, lavorando sul'istanza di \texttt{Graph} al suo interno.


\subsection{Costruzione dei dati della stack trace}
\begin{figure}[H]
	\centering
	\includegraphics[width=1\textwidth]{Images/DiagrammaSequenzaStack.png}
	\caption{Diagramma di Sequenza UML rappresentante la costruzione della stack trace}
	\label{img:seqGraphstack}
\end{figure}
La sequenza di azioni per quanto riguarda l'interazione tra \texttt{StackBuilder}, \texttt{DataReader}, \texttt{ElasticsearchClient}, \texttt{DataCleaner} e \texttt{StackCleaner} è uguale a quella riguardande \texttt{GraphBuilder.}\\
Dopo aver ricevuto i dati e averli ripuliti dai dati superflui \texttt{StackBuilder} utilizza il metodo \texttt{build\_request()} per la costruzione delle richieste da visualizzare nella stack trace.


% \section{Tracciamento dei requisiti}
% \label{sec:Tracciamento}

% \normalsize
% \begin{longtable}{|c|c|}
% 	\hline
% 	\textbf{Codice Requisiti} & \textbf{Implementazione} \\
% 	\hline
% 	\endhead
% 	R0F1 & Implementato\\
% 	\hline
% 	R0F1.1  & Implementato\\
% 	\hline
% 	R0F1.1.1  & Implementato\\
% 	\hline
% 	R0F1.1.2  & Implementato\\
% 	\hline
% 	R0F1.1.3 & Non implementato\\
% 	\hline
% 	R1F1.2.1 & Implementato\\
% 	\hline
% 	R1F1.2.2 & Implementato\\
% 	\hline
% 	R1F1.2.3 & Non implementato\\
% 	\hline
% 	R1F1.3 & Implementato\\
% 	\hline
% 	R2F1.3.1 & Non implementato\\
% 	\hline
% 	R1F1.3.2 & Implementato\\
% 	\hline
% 	R1F1.4 & Implementato\\
% 	\hline
% 	R2F1.4.1 & Implementato\\
% 	\hline
% 	R2F1.4.2 & Non implementato\\
% 	\hline
% 	R0F1.5 & Implementato\\
% 	\hline
% 	R2F1.5.1 & Implementato\\
% 	\hline
% 	R2F1.5.1.1 & Non implementato\\
% 	\hline
% 	R2F1.5.1.2 & Implementato\\
% 	\hline
% 	R0F1.5.2 & Implementato\\
% 	\hline
% 	R0F1.5.2.1 & Implementato\\
% 	\hline
% 	R0F1.5.2.2 & Implementato\\
% 	\hline
% 	R1F1.6 & Implementato\\
% 	\hline
% 	R1F1.6.1 & Implementato\\
% 	\hline
% 	R1F1.6.2 & Implementato\\
% 	\hline
% 	R1F1.6.2.1 & Implementato\\
% 	\hline
% 	R1F1.6.2.2 & Non implementato\\
% 	\hline
% 	R2F1.7 & Implementato\\
% 	\hline
% 	R2F1.7.1 & Implementato\\
% 	\hline
% 	R2F1.7.2 & Implementato\\
% 	\hline
% 	R2F1.7.3 & Non implementato\\
% 	\hline
% 	R1F1.8 & Non implementato\\
% 	\hline
% 	R0F2 & Implementato\\
% 	\hline
% 	R0F2.1 & Implementato\\
% 	\hline
% 	R1F2.1.1 & Implementato\\
% 	\hline
% 	R0F2.1.2 & Implementato\\
% 	\hline
% 	R1F2.1.3 & Implementato\\
% 	\hline
% 	R1F2.1.4 & Implementato\\
% 	\hline
% 	R0F2.1.5 & Implementato\\
% 	\hline
% 	R1F2.1.5.1 & Non implementato\\
% 	\hline
% 	R1F2.2 & Non implementato\\
% 	\hline
% 	R1F2.2.1 & Non implementato\\
% 	\hline
% 	R1F2.2.1.1 & Non implementato\\
% 	\hline
% 	R1F2.2.1.2 & Non implementato\\
% 	\hline
% 	R1F2.2.2 & Non implementato\\
% 	\hline
% 	R1F2.2.2.1 & Non implementato\\
% 	\hline
% 	R1F2.2.2.2 & Non implementato\\
% 	\hline
% 	R1F2.4 & Non implementato\\
% 	\hline
% 	R0F3 & Implementato\\
% 	\hline
% 	R0F3.1 & Implementato\\
% 	\hline
% 	R0F3.1.1 & Implementato\\
% 	\hline
% 	R1F3.1.2 & Implementato\\
% 	\hline
% 	R1F3.1.3 & Implementato\\
% 	\hline
% 	R2F3.1.4 & Implementato\\
% 	\hline
% 	R2F3.2 & Implementato\\
% 	\hline
% 	R2F3.2.1 & Implementato\\
% 	\hline
% 	R2F3.2.2 & Implementato\\
% 	\hline
% 	R1F4 & Implementato\\
% 	\hline
% 	R1F4.1 & Implementato\\
% 	\hline
% 	R1F4.1.1 & Implementato\\
% 	\hline
% 	R1F4.1.2 & Implementato\\
% 	\hline
% 	R1F4.1.3 & Implementato\\
% 	\hline
% 	R1F4.1.4 & Implementato\\
% 	\hline
% 	R1F4.1.5 & Implementato\\
% 	\hline
% 	R1F4.2 & Non implementato\\
% 	\hline
% 	R1F4.2.1 & Non implementato\\
% 	\hline
% 	R1F4.2.1.1 & Non implementato\\
% 	\hline
% 	R1F4.2.1.2 & Non implementato\\
% 	\hline
% 	R1F4.2.2 & Non implementato\\
% 	\hline
% 	R1F4.2.2.1 & Non implementato\\
% 	\hline
% 	R1F4.2.2.2 & Non implementato\\
% 	\hline
% 	\caption[Tracciamento Requisiti Funzionali]{Tracciamento Requisiti Funzionali}
% \end{longtable}

% \begin{figure}[H]
% 	\centering
% 	\includegraphics[scale = 0.6]{Images/obbligatori.png}
% 	\caption{Copertura requisiti funzionali obbligatori}
% 	\label{img:seqGraph}
% \end{figure}
% \begin{figure}[H]
% 	\centering
% 	\includegraphics[scale = 0.6]{Images/desiderabili.png}
% 	\caption{Copertura requisiti funzionali desiderabili}
% 	\label{img:seqGraph}
% \end{figure}
% \begin{figure}[H]
% 	\centering
% 	\includegraphics[scale = 0.6]{Images/opzionali.png}
% 	\caption{Copertura requisiti funzionali opzionali}
% 	\label{img:seqGraph}
% \end{figure}

% \begin{longtable}{|c|c|}
% 	\hline
% 	\textbf{Codice Requisiti} & \textbf{Implementazione} \\
% 	\hline
% 	\endhead
% 	R0Q1 & Implementato\\
% 	\hline
% 	R0Q2 & Implementato\\
% 	\hline
% 	R0Q3 & Implementato\\
% 	\hline
% 	R0Q4 & Implementato\\
% 	\hline
% 	R0Q5 & Implementato\\
% 	\hline
% 	R0Q6 & Implementato\\
% 	\hline
% 	\caption[Tracciamento Requisiti Qualità]{Tracciamento Requisiti Qualità}
% \end{longtable}

% \begin{longtable}{|c|c|}
% 	\hline
% 	\textbf{Codice Requisiti} & \textbf{Implementazione} \\
% 	\hline
% 	\endhead
% 	R0V1 & Implementato\\
% 	\hline
% 	R0V2 & Implementato\\
% 	\hline
% 	R1V2.1 & Implementato\\
% 	\hline
% 	R1V2.2 & Implementato\\
% 	\hline
% 	R1V2.3 & Implementato\\
% 	\hline
% 	R1V2.4 & Non implementato\\
% 	\hline
% 	R1V2.5 & Non implementato\\
% 	\hline
% 	R1V2.6 & Non implementato\\
% 	\hline
% 	R1V2.7 & Non implementato\\
% 	\hline
% 	R0V3 & Implementato\\
% 	\hline
% 	R0V4 & Implementato\\
% 	\hline
% 	R0V5 & Implementato\\
% 	\hline
% 	R0V6 & Implementato\\
% 	\hline
% 	\caption[Tracciamento Requisiti Vincolo]{Tracciamento Requisiti Vincolo}
% \end{longtable}
% \clearpage
