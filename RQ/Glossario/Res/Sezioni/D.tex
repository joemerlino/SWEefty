\lettera{D}

\parola{Dapp}{Applicazione il cui backend viene eseguito in una rete peer-to-peer decentralizzata.}

\parola{Dashboard}{Termine usato in genere per fare riferimento a una postazione o a una pagina web basata su una tecnologia che visualizza informazioni, raccolte in tempo reale da varie fonti.}

\parola{Database}{Archivio di dati strutturato in modo da razionalizzare la gestione e l'aggiornamento delle informazioni e da permettere lo svolgimento di ricerche complesse.}

\parola{Design Pattern}{Soluzione progettuale generale ad un problema ricorrente.}

\parola{Diagramma di Gantt}{Strumento di supporto alla gestione dei progetti costruito partendo da un asse orizzontale, a rappresentazione dell'arco temporale totale del progetto, suddiviso in fasi incrementali (ad esempio, giorni, settimane, mesi), e da un asse verticale, a rappresentazione delle mansioni o attività che costituiscono il progetto.}

\parola{Diagramma di robustezza}{Rappresenta visivamente il comportamento di un caso d'uso, mostrando sia le classi partecipanti che il comportamento del software. Tuttavia, non descrive quale classe sia responsabile per quali parti del comportamento. Gli oggetti rappresentati nel diagramma comunicano l'un l'altro mediante l'uso di una linea che li collega.}

\parola{Driver}{Insieme di procedure, scritte in un certo linguaggio, che permette a un dato sistema operativo di gestire un ben preciso dispositivo hardware.}

\parola{Dropbox}{Servizio di file hosting che offre cloud storage, sincronizzazione automatica dei file, cloud personale e software client.}

\parola{D3.js}{Libreria JavaScript per creare visualizzazioni dinamiche ed interattive, partendo dati organizzati, visibili attraverso un comune browser.}