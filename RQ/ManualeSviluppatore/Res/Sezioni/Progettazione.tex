\section{Requisiti di sistema}
\subsection{Browser Supportati}

Di seguito viene fornito un breve elenco delle versioni minime dei browser sui quali il funzionamento del nostro plugin è garantito:
\begin{itemize}
	
	\item Google Chrome v.55
	\item Mozilla Firefox v.50
	\item Safari v.10
	\item Internet Explorer v.11
	
\end{itemize}

Per un corretto funzionamento del plugin è necessario che sia abilitato Javascript




\section{Configurazione ambiente di lavoro}
\label{sec:configurazione}
\subsection{Installazione di Kibana}
Per poter installare correttamente il prodotto è necessario scaricare e configurare antecedentemente Kibana, reperibile dal link URL \url{https://github.com/elastic/kibana/releases}.\\
I plugin sono stati sviluppati con la versione 6.2.2 che si raccomanda anche per l'utilizzo.\\
Una volta scaricato l'archivio compresso sarà necessario estrarre il contenuto in una cartella con nome \texttt{kibana} da raggiungere tramite l'utilizzo di un terminale.
Prima di procedere alla configurazione di Kibana deve essere installato il framework Node.js ed il relativo gestore di pacchetti chiamato Node package manager (npm).\\
Dalla directory corrente, dare il comando \texttt{npm install} per procedere all'installazione dei pacchetti necessari.\\
Una volta terminato il processo, spostare le cartelle del prodotto all'interno della cartella plugins di Kibana.\\
Raggiungere le cartelle del prodotto tramite terminale e dare il comando \texttt{npm install} per avviare l'installazione dei moduli node necessari al funzionamento dei plugin.
Una volta terminata la configurazione, spostarsi tramite terminale nella cartella \texttt{kibana/bin} ed avviare l'eseguibile \texttt{kibana}.\\
Dopo l'avvio, aprire una finestra di browser e navigare all'indirizzo URL \url{localhost:5601} per raggiungere la dashboard di Kibana.
\subsection{Installazione dei plugin}
Per prima cosa bisogna provvedere a scaricare i plugin. Questi si possono scaricare da \href{https://github.com/SWEeftyTeam/Havana}{https://github.com/SWEeftyTeam/Havana} cliccando sul pulsante verde "Clone or download" e poi su "Download ZIP".
Per l'installazione dei plugin in una \gl{directory} a scelta utilizzare il comando \texttt{install} con l'opzione \texttt{-d} o \texttt{--plugin-dir}  specificando una directory per i plugin, come nell'esempio seguente:
\begin{lstlisting}
$ bin /kibana-plugin install file:///some/local/path/my-plugin.zip -d path/to/directory 
\end{lstlisting}
Una volta completata l'installazione è possibile navigare tra tutti i plugin installati nella lista presente nel menù a sinistra di Kibana.
\subsection{Configurazione di ElasticSearch}
Kibana si appoggia ad ElasticSearch per prelevare i dati da monitorare, quindi sarà necessaria un'istanza di esso che si può avere in locale o remoto configurando il file \texttt{kibana.yml} all'interno della cartella \texttt{config}, in particolare il campo \texttt{elasticsearch.url: "localhost:9200"} che dovrà essere modificato con l'indirizzo dove risiede l'istanza remota di ElasticSearch.\\
Localmente invece, scaricare il pacchetto dal link URL \url{https://www.elastic.co/downloads/elasticsearch}, versione raccomandata 6.2.2.
Una volta scaricato l'archivio compresso, estrarre il contenuto in una cartella locale, raggiungerla tramite terminale ed entrare nella sottocartella \texttt{bin} da dove avviare l'eseguibile \texttt{elasticsearch}.\\
L'istanza sarà reperibile all'indirizzo URL \url{localhost:9200}.
\newpage
\section{Estensibilità}
\subsection{Modifica delle funzionalità presenti}
\subsubsection{Mappa topologica}
\label{sec:graph}
Per poter cambiare gli aspetti di visualizzazione della mappa topologica è necessario modificare il file \texttt{app.js} presente nella cartella \texttt{plugin} ove è possibile trovare le seguenti variabili usate dalla libreria d3.js per comporre gli elementi del grafo:
\begin{itemize}
	\item \texttt{svg}: Canvas del grafo;
	\item \texttt{color}: Schema colori;
	\item \texttt{simulation}: Gestore per i comportamenti fisici del grafo;
	\item \texttt{link}: Archi tra componenti;
	\item \texttt{linkLabel}: Etichette degli archi rappresentanti i tempi medi;
	\item \texttt{connectionLabel}: Etichette degli archi rappresentanti il tipo di richiesta effettuata;
	\item \texttt{node}: Nodi del grafo rappresentanti i componenti monitorati;
	\item \texttt{nodelabels}: Etichette con il nome identificativo dei nodi.
\end{itemize}
Ad esempio modificando l'attributo \texttt{style} della variabile \texttt{link} si possono aggiungere ulteriori colorazioni condizionate per gli archi tra i componenti, mentre editando l'attributo \texttt{xlink:href} della variabile \texttt{node} si possono introdurre nuove icone per nuovi componenti monitorati. \\
Per immettere ulteriori icone, copiare i file svg nella directory dedicata \texttt{public/res/img} ed importarle poi con il seguente codice:
\begin{lstlisting}
import exampleSvg from "plugins/plugin-havana/res/img/example.svg";
\end{lstlisting}
nel file \texttt{app.js}.\\
Inoltre sono a disposizione funzioni come \texttt{getLimit()} che ritorna la soglia limite oltre la quale il tempo di risposta è da considerarsi critico comportando quindi la colorazione rossa dell'arco interessato. 
Si è prestabilito un valore di ritorno pari a 3000ms che può essere modificato a seconda delle necessità anche introducendo algoritmi per un calcolo dinamico della soglia limite. \\
Infine sono disponibili le funzioni per la gestione di azioni come zoom, drag e click per i vari componenti della mappa:
\begin{itemize}
	\item \texttt{zoomed()}: applica il metodo \texttt{d3.event.transform};
	\item \texttt{ticked()}: posiziona ogni oggetto alle coordinate x ed y nel canvas;
	\item \texttt{dragged()}: sposta alle coordinate \texttt{d3.event.x} e \texttt{d3.event.y} l'oggetto coinvolto nel trascinamento;
	\item \texttt{dragstarted()}: attiva l'elemento da trascinare;
	\item \texttt{dragended()}: rimuove la classe d'attivazione dell'oggetto a trascinamento completato.
\end{itemize}
Modificando ad esempio l'attributo x ed y di \texttt{nodelabels} all'interno della funzione \texttt{ticked} è possibile spostare il posizionamento dell'etichetta rappresentante l'identificativo del nodo rispetto all'icona del nodo stesso.\\

\subsubsection{Stack trace}
\label{sec:stack}
Per poter cambiare gli aspetti di visualizzazione della stack trace è necessario modificare il file \texttt{index.html} presente nella cartella \texttt{plugin/templates}.\\
Oltre alle modifiche apportabili al codice HTML e CSS, è possibile cambiare il comportamento di alcune funzioni JavaScript che regolano l'espansione e la compressione di una riga della stack trace o lo switch tra la visualizzazione di un call tree e le query ad esso associate.\\
Attraverso delle direttive AngularJs è possibile sfruttare in maniera ottimale i dati provenienti da \$scope.nodes che rappresenta un array contenente tutte le trace di nostro interesse.\\
Inoltre nel file \texttt{app.js} presente nella cartella \texttt{plugin} sono state implementate delle direttive personalizzate AngularJs per la gestione del call tree di ogni trace, che sono:
\begin{lstlisting}
.directive("tree", function() {..})
\end{lstlisting}
e
\begin{lstlisting}
.directive("branch", function($compile) {...})
\end{lstlisting}
Queste due direttive funzionano in modo innestabile tra loro permettendo di visualizzare i vari metodi appartenenti ad un call tree seguendo la struttura gerarchica con cui sono stati salvati nella variabile \$scope.nodes.calltree.


\subsection{Sviluppo di nuove funzionalità}
\label{sec:nuovefunzionalita}
Per poter sviluppare nuove modalità di visualizzazione dei dati è possibile avvalersi delle classi \texttt{DataReader} e \texttt{DataCleaner} già presenti all'interno della cartella \texttt{public/components} dei plugin, creando quindi in questa directory una classe \texttt{Builder} per strutturare le informazioni da passare al frontend dell'applicazione secondo una nuova strategia da creare nella sottocartella \texttt{strategies}.\\
Lo sviluppo del frontend può essere eseguito modificando adeguadatamente alle esigenze i file:
\begin{itemize}
	\item \texttt{app.js} presente nella cartella \texttt{public} dei plugin;
	\item \texttt{index.html} presente nella cartella \texttt{public/templates}.
\end{itemize}
Nel primo file sarà necessario importare la classe \texttt{Builder} e la relativa strategia prodotta:
\begin{lstlisting}
let ExampleBuilder = require("./components/exampleBuilder");
let ExampleCleaner = require("./components/strategies/examplecleaner");
\end{lstlisting}
per poter poi istanziare i relativi oggetti necessari allo sviluppo del frontend.
\newpage
\appendix
\section{Glossario}
\lettera{C}

\parola{Call tree}{Albero delle chiamate, rappresenta l'insieme delle chiamate a funzioni o funzionalità del sistema eseguite da un programma.}

\parola{Cluster}{Insieme di computer connessi tra loro tramite una rete telematica. Lo scopo di un cluster è quello di distribuire una elaborazione molto complessa tra i vari computer.}


\lettera{D}

\parola{Database}{Archivio di dati strutturato in modo da razionalizzare la gestione e l'aggiornamento delle informazioni e da permettere lo svolgimento di ricerche complesse.}

\parola{Directory}{Entità del file system che elenca altre entità, tipicamente file e altre directory, e che permette di organizzarle in una struttura ad albero.}

\lettera{E}

\parola{Elastic Search}{Motore di ricerca open-source. Dispone di API molto estensive ed elaborate. Può effettuare ricerche in maniera molto perfomante.}

\lettera{G}
\parola{Google Chrome}{Google Chrome è un browser web sviluppato da Google, basato sul motore di rendering Blink.}

\lettera{H}
\parola {Hardware}{Insieme delle componenti fisiche, non modificabili, di un sistema di elaborazione dati.}

\lettera{I}
\parola{Internet Explorer}{Browser web grafico proprietario sviluppato da Microsoft e incluso in Microsoft Windows a partire dal 1995.}

\lettera{J}

\parola{JavaScript}{Linguaggio di scripting utilizzato principalmente per applicazioni web.}

\lettera{K}

\parola{Kibana}{Framework open source per la visualizzazione dei dati salvati su Elasticsearch.}

\lettera{M}

\parola{Mappa topologica}{Visualizzazione sotto forma di grafo dei componenti di un'applicazione in cui i componenti come server e database sono i nodi e le interazioni fra essi, ovvero le richieste fatte da un componente ad un altro (come richieste HTTP o queries), sono gli archi.}

\parola{Mozilla Firefox}{Web browser libero e multipiattaforma, mantenuto da Mozilla Foundation.}

\lettera{P}
\parola{Plugin}{Programma non autonomo che interagisce con un altro programma per ampliare o estendere le funzionalità originarie.}  

\parola{Prodotto}{Indica il risultato di un'attività, sia esso un documento, del codice sorgente o un qualsiasi risultato verificabile che possa essere offerto per soddisfare un bisogno o un’esigenza.}  

\parola{Progetto}{Insieme di attività e compiti che prevedono il raggiungimento di determinati obiettivi con specifiche fissate. Sono definite date di inizio e fine durante la quale si può predisporre di limitate risorse che vengono consumate nello svolgersi delle attività. }  

\lettera{S}

\parola{Safari}{Web browser multipiattaforma, sviluppato e mantenuto da Apple.}

\parola{Stack Trace}{Lista di un insieme di traces prese in esame.}

\lettera{T}

\parola{Trace}{Struttura dati che rappresenta una singola richiesta effettuata fra due componenti di una applicazione monitorata oppure fra un utente esterno e un componente dell'applicazione. Viene solitamente memorizzata in un database per successiva analisi.}

\lettera{U}

\parola{URL}{\emph{Uniforme Resource Locator}, indirizzo simoboli per riferirsi ad una risorsa in rete.}








