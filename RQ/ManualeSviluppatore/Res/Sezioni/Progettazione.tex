\section{Requisiti di sistema}
\subsection{Browser Supportati}

Di seguito viene fornito un breve elenco delle versioni minime dei browser sui quali il funzionamento del nostro plugin è garantito:
\begin{itemize}
	
	\item Google Chrome v.55
	\item Mozilla Firefox v.50
	\item Safari v.10
	\item Internet Explorer v.11
	
\end{itemize}

Per un corretto funzionamento del plugin è necessario che sia abilitato Javascript




\section{Manuale Sviluppatore}
\subsection{Configurazione ambiente di lavoro}
\paragraph{Installazione}
Per poter installare correttamente il prodotto è necessario scaricare e configurare antecedentemente Kibana, scaricabile dal link URL \url{https://github.com/elastic/kibana/releases}.\\
I plugin sono stati sviluppati con la versione 6.2.2 che si raccomanda anche per l'utilizzo.\\
Una volta scaricato l'archivio compresso sarà necessario estrarre il contenuto in una cartella con nome \texttt{kibana} da raggiungere tramite l'utilizzo di un terminale.
Prima di procedere alla configurazione di Kibana deve essere installato il framework Node.js ed il relativo gestore di pacchetti chiamato Node package manager (npm).\\
Dalla directory corrente, dare il comando \texttt{npm install} per procedere all'installazione dei pacchetti necessari.\\
Una volta terminato il processo, spostare le cartelle del prodotto all'interno della cartella plugins di Kibana.\\
Raggiungere le cartelle del prodotto tramite terminale e dare il comando \texttt{npm install} per avviare l'installazione dei moduli node necessari al funzionamento dei plugin.
Una volta terminata la configurazione, spostarsi tramite terminale nella cartella \texttt{kibana/bin} ed avviare l'eseguibile \texttt{kibana}.\\
Dopo l'avvio, aprire una finestra di browser e navigare all'indirizzo URL \url{localhost:5601} per raggiungere la dashboard di Kibana.
\paragraph{Configurazione ElasticSearch}
Kibana si appoggia ad ElasticSearch per prelevare i dati da monitorare, quindi sarà necessaria un'istanza di esso che si può avere in locale o remoto configurando il file \texttt{kibana.yml} all'interno della cartella \texttt{config}, in particolare il campo \texttt{elasticsearch.url: "localhost:9200"} che dovrà essere modificato con l'indirizzo dove risiede l'istanza remota di ElasticSearch.\\
Localmente invece, scaricare il pacchetto dal link URL \url{https://www.elastic.co/downloads/elasticsearch}, versione raccomandata 6.2.2.
Una volta scaricato l'archivio compresso, estrarre il contenuto in una cartella locale, raggiungerla tramite terminale ed entrare nella sottocartella \texttt{bin} da dove avviare l'eseguibile \texttt{elasticsearch}.\\
L'istanza sarà reperibile all'indirizzo URL \url{localhost:9200}.
\subsection{Estensibilità}











