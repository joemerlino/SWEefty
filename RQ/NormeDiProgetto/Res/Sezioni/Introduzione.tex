\section{Introduzione}
\label{sec:intro}
\subsection{Scopo del documento}
Il seguente documento ha l'obiettivo di esplicitare le norme, le convenzioni e la strumentazione che sarà adottata dal gruppo SWEefty durante l'intero svolgimento del \gl{progetto}. Con questa prospettiva deve essere visionato da tutti i membri del gruppo, i quali sono tenuti ad osservare quanto scritto per mantenere consistenza ed omogeneità in ogni aspetto, durante il ciclo di vita del software che sarà prodotto durante il progetto.
\subsection{Scopo del prodotto}
	Il \gl{prodotto} che SWEefty è tenuto a realizzare consiste in una coppia di \gl{plugin} per \gl{Kibana} che devono fornire due funzionalità fondamentali:
	\begin{itemize}
		\item \textbf{Visualizzazione \gl{mappa topologica} del sistema:} il plugin deve visualizzare in maniera chiara ed intuitiva come le componenti del sistema interagiscono tra di loro, con annesse informazioni utili;
		\item \textbf{Visualizzazione della \gl{stack trace}:} vengono visualizzate sotto forma di lista le interazioni fra i componenti e le richieste HTTP effettuate ai \gl{server}. Per ogni trace della lista inoltre verrà visualizzata la rispettiva \gl{call tree} e le query effettuate ai \gl{database}.
	\end{itemize}

\subsection{Glossario}
	Volendo evitare incomprensioni  ed equivoci per rendere la lettura del documento più semplice e chiara viene allegato il \emph{Glossario v2.0.0} nel quale sono contenute le definizioni dei termini tecnici, dei vocaboli ambigui, degli acronimi e delle abbreviazioni. Questi termini sono evidenziati nel presente documento con una g al pedice (esempio: $Glossario_{g}$).

\subsection{Riferimenti}
	\subsubsection{Normativi}
		\begin{itemize}
			\item \textbf{ISO/IEC 12207:} \href{https://en.wikipedia.org/wiki/ISO/IEC\_12207}{https://en.wikipedia.org/wiki/ISO/IEC\_12207} (ultima consultazione effettuata in data 2017-12-29)
			\item \textbf{Verbali di incontri esterni:} \emph{VE\_2017-12-06}, \emph{VE\_2017-12-19}, \emph{VE\_2018-03-09};
			\item \textbf{Verbali di incontri interni:}	\emph{VI\_2017-12-07}, \emph{VI\_2017-12-20}, \emph{VI\textunderscore2018-01-31},\newline \emph{VI\_2018-03-20}.
		\end{itemize}
	\subsubsection{Informatici}
	\begin{itemize}
		\item git:
		\href{https://git-scm.com/docs/user-manual.html}{https://git-scm.com/docs/user-manual.html} (ultima consultazione effettuata in data 2018-01-08);
		
		\item \LaTeX:  \href{https://en.wikibooks.org/wiki/LaTeX}{https://en.wikibooks.org/wiki/LaTeX} (ultima consultazione effettuata in data 2018-01-12);
		
		\item Javascript:
		\href{https://developer.mozilla.org/bm/docs/Web/JavaScript/Guide}{https://developer.mozilla.org/bm/docs/Web/JavaScript/Guide} (ultima consultazione effettuata in data 2017-12-15);
		
		\item Kibana: \href{https://www.elastic.co/guide/en/kibana/6.1/index.html}{https://www.elastic.co/guide/en/kibana/6.1/index.html} (ultima consultazione effettuata in data 2018-01-08);
		
		\item TexStudio: 
		\href{http://texstudio.sourceforge.net/manual/current/usermanual\_en.html}{http://texstudio.sourceforge.net/manual/current/usermanual\_en.html} (ultima consultazione effettuata in data 2018-01-02).
	\end{itemize}
	