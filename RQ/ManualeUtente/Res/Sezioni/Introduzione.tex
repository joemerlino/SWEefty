\section{Introduzione}
\label{sec:intro}
	\subsection{Scopo del documento}
	Questo documento ha lo scopo di semplificare l'apprendimento delle funzionalità offerte dal \gl{plugin} \emph{Havana}, \gl{prodotto} sviluppato dal gruppo SWEefty.
	
	\subsection{Scopo del prodotto}
	Il prodotto che SWEefty è tenuto a realizzare consiste in una coppia di plugin per \gl{Kibana} che devono fornire due funzionalità fondamentali:
	\begin{itemize}
		\item \textbf{Visualizzazione \gl{mappa topologica} del sistema:} il plugin deve visualizzare in maniera chiara ed intuitiva come le componenti del sistema interagiscono tra di loro, con annesse informazioni utili;
		\item \textbf{Visualizzazione della \gl{stack trace}:} vengono visualizzate sotto forma di lista le interazioni fra i componenti e le richieste HTTP effettuate ai \gl{server}. Per ogni trace della lista inoltre verrà visualizzata la rispettiva \gl{call tree} e le query effettuate ai \gl{database}.
	\end{itemize}

	\subsection{Glossario}
	Volendo evitare incomprensioni ed equivoci per rendere la lettura del documento più semplice
	e chiara si riporta in appendice un glossario nel quale sono contenute le definizioni dei
	termini tecnici, dei vocaboli ambigui, degli acronimi e delle abbreviazioni. Questi termini
	sono evidenziati nel documento con una g al pedice (esempio: \gl{Glossario}).