\section{Introduzione}
\label{sec:intro}
	\subsection{Scopo del documento}
	Lo scopo di questo documento è di riportare dettagliatamente tutte le scelte architetturali che SWEefty ha deciso di adottare durante lo sviluppo del prodotto.
	
	\subsection{Scopo del prodotto}
	Il prodotto che SWEefty è tenuto a realizzare consiste in una coppia di plugin per Kibana che devono fornire due funzionalità fondamentali:
	\begin{itemize}
		\item \textbf{Visualizzazione mappa topologica del sistema:} il plugin deve visualizzare in maniera chiara ed intuitiva come le componenti del sistema interagiscono tra di loro, con annesse informazioni utili;
		\item \textbf{Visualizzazione della stack trace:} vengono visualizzate sotto forma di lista le interazioni fra i componenti e le richieste HTTP effettuate ai server. Per ogni trace della lista inoltre verrà visualizzata la rispettiva call tree e le query effettuate ai database.
	\end{itemize}

	% \subsection{Glossario}
	% Volendo evitare incomprensioni  ed equivoci per rendere la lettura del documento più semplice e chiara viene allegato il \emph{Glossario v3.0.0} nel quale sono contenute le definizioni dei termini tecnici, dei vocaboli ambigui, degli acronimi e delle abbreviazioni. Questi termini sono evidenziati nel presente documento con una g al pedice (esempio: $Glossario_{g}$).
	\subsection{Riferimenti}
		\paragraph{Normativi}
			\begin{itemize}
				\item \textbf{Norme di Progetto:} \emph{Norme di Progetto v3.0.0}.
			\end{itemize}

		\paragraph{Informativi} 
			\begin{itemize}
				\item \textbf{Kibana:} \href{https://www.elastic.co/guide/en/kibana/6.1/index.html}{https://www.elastic.co/guide/en/kibana/6.1/index.html}(ultima consultazione effettuata in data 2018-04-09)
				\item \textbf{Elasticsearch:} \href{https://www.elastic.co/guide/index.html}{https://www.elastic.co/guide/index.html}(ultima consultazione effettuata in data 2018-04-09)
				\item \textbf{AngularJS:} \href{https://docs.angularjs.org/guide}{https://docs.angularjs.org/guide}(ultima consultazione effettuata in data 2018-04-09)
				\item \textbf{JavaScript:} \href{https://developer.mozilla.org/bm/docs/Web/JavaScript/Guide}{https://developer.mozilla.org/bm/docs/Web/JavaScript/Guide}(ultima consultazione effettuata in data 2018-04-09)
			\end{itemize}