
\subsection{Lato Client}
Il lato client si occupa di interrogare il lato server per ottenere informazioni grezze, elaborarle per renderle utili ed infine presentarle all'utente.

\label{sec:Componenti}
\subsubsection{Rappresentazione}
Il codice prodotto è stato scritto in JavaScript ES6, quindi molti concetti quali classi ed interfacce non sono presenti all'interno del linguaggio. Per produrre un diagramma delle classi dunque sono state considerate \emph{ classi } sia oggetti Javascript, sia funzioni e talvolta costrutti più complessi. Tali stratagemmi verranno comunque chiariti nella descrizione testuale di ciascun componente. Per quanto riguarda le interfacce che sono presenti nel diagramma delle classi, figura \ref{img:diagrammaClassiClient}, nel codice non sono effettivamente presenti tali interfacce, ma tutte le classi che implementano tale interfaccia \emph{devono} possedere i metodi esposti da tale interfaccia. Questo stratagemma è stato particolarmente utile nell'implementazione del componente \texttt{DataCleaner}, come descritto in \ref{sec:DataCleaner}.

\begin{figure}[H]
    \centering
    \includegraphics[width=1\textwidth]{Images/classi.png}
    \caption{Diagramma delle classi client-side dell'applicazione}
    \label{img:diagrammaClassiClient}
\end{figure}

\subsubsection{Componenti} \Spazio
\paragraph{ElasticsearchIndexResult} \Spazio
\paragraph{Shards} \Spazio
\paragraph{Hits} \Spazio
\paragraph{ElasticsearchDocument} \Spazio
\paragraph{RawTrace} \Spazio
Rappresenta il contenuto vero e proprio del documenti salvato sull'istanza di Elasticsearch. Essendo Elasticsearch un database non relazionale non è possibile conoscere a priori la struttura di tale documento, dunque il suo contenuto viene qui rappresentato tramite un attributo pubblico chiamato content, di tipo JSON con struttura variabile.

\paragraph{ElasticsearchClient} \Spazio
\texttt{ElasticsearchClient} è il componente che si occupa di eseguire le chiamate API al lato server. Non è una classe vera e propria all'interno del codice, ma un \emph{Service}. 
I Service sono funzioni, od oggetti, che possono essere passati a tutte le altri componenti di angularJs attraverso \emph{Dependency Injection}, permettono di condividere codice e funzionalità facilmente. Essi sono \emph{lazily instantiated}, cioè generati solo quando un componente esprime una dipendenza verso un determinato service e realizzano il design pattern \emph{singleton}. Permettono la realizzazione di servizi come i client API, dove avere piú istanze dello stesso client non sarebbe ragionevole. 
Utilizzare un service è l'unico modo per poter eseguire chiamate REST parametrizzate al lato server. Ciascun metodo qui presentato nasconde in realtà una chiamata GET all'opportuno componente lato server. I metodi da esso esposti sono: 
\begin{itemize} 
  \item \texttt{tracesIndices(): Array<String>} ritorna un array di stringhe che rappresenta gli indici contenuti nell'istanza di ElasticSearch che hanno all'interno delle traces, ovvero i dati utili al plugin; 
  \item \texttt{getIndex(index:String): ElasticsearchIndexResult} ritorna al chiamante i dati grezzi che vengono restituiti da Elasticsearch eseguendo una ricerca priva di parametri sull'indice passato come parametro, ovvero \texttt{index}. Essi comprendono, oltre tutti i documenti dell'indice, anche i metadati di Elasticsearch dato che sono di tipo \texttt{ElasticsearchIndexResult}. È anche possibile specificare un numero massimo di documenti ritornati al chiamante o una particolare tipologia letta nel database di tipo \texttt{server} o \texttt{database}; 
  \item \texttt{getData(indices:Array<String>): Array<ElasticsearchIndexResult>}: ritorna al chiamante un array di dati grezzi, ognuno dei quali rappresenta il risultato di una ricerca priva di parametri su un indice. Gli indici letti sono quelli ritornati da \texttt{tracesIndices()}, ovvero tutti e soli quelli che contengono traces. 
\end{itemize} 
Osserviamo che anche se in questa rappresentazione questi metodi sono all'interno di un oggetto, non avendo i diagrammi delle classi UML un costrutto per rappresentare precisamente un \emph{Service}, essi non hanno una struttura ben definita e oggetti di tipo \texttt{ElasticsearchClient} chiaramente non sono istanziabili direttamente dal client. Dunque, per interfacciarsi ad Elasticsearch si è deciso di racchiudere tutte queste funzionalità in un oggetto vero e proprio, \texttt{DataReader} (descritto nella sezione \ref{sec:DataReader}), per poter avere una più agevole interazione con i dati. 

\paragraph{DataReader} \Spazio
\label{sec:DataReader}
\texttt{DataReader} è il componente del sistema che si occupa di reperire i dati da Elasticsearch, ciò viene fatto utilizzando il \emph{Service} \texttt{ElasticsearchClient}. Sarebbe effettivamente possibile utilizzare le chiamate al \emph{Service} ogni qualvolta fossero necessari dei dati, tuttavia ricordiamo che tali chiamate non vengono racchiuse all'interno di un oggetto ben strutturato. DataReader restituirà ai richiedenti le informazioni grezze i tipo \texttt{ElasticsearchIndexResult}, esattamente come vengono restituite da Elasticsearch. I metodi sono:  

\begin{itemize} 
	\item \texttt{constructor(esClient:ElasticsearchClient): void} che costruisce il \texttt{DataReader} parametrizzato con un \texttt{ElasticsearchClient};
	\item \texttt{setElasticsearchClient(esClient:ElasticsearchClient): void}: permette di impostare una sorgente dei dati;
	\item \texttt{tracesIndices(): Array<String>}: ritorna un array di stringhe, i quali elementi sono tutti e soli gli indici all'interno dell'istanza di Elasticsearch che hanno al loro interno delle traces;
	\item \texttt{readIndex(index:String): ElasticsearchIndexResult}: ritorna al chiamante i dati grezzi che vengono restituiti da Elasticsearch eseguendo una ricerca priva di parametri sull'indice passato come parametro, ovvero \texttt{index}. Essi comprendono oltre tutti i documenti dell'indice anche i dati amministrativi di Elasticsearch;
	\item \texttt{readData(): Array<ElasticsearchIndexResult>}: ritorna al chiamante un array di dati grezzi, ognuno dei quali rappresenta il risultato di una ricerca priva di parametri su un indice. Gli indici letti sono quelli contententi delle traces, ovvero quelli utili al plugin. Ciò è realizzato leggendo tali indici grazie al metodo \texttt{tracesIndices()}.
\end{itemize}



\paragraph{DataCleaner}\Spazio

\label{sec:DataCleaner}
\texttt{DataCleaner} è il componente del sistema che si occupa di pulire i dati grezzi che sono stati reperiti da Elasticsearch. Esso compie questo lavoro in due momenti: prima estraendo i dati utili (ovvero i documenti veri e propri) da Elasticsearch e poi pulendoli attraverso una \emph{strategy}, descritta in \ref{sec:CleanerStrategy}.\\
I suoi metodi sono:

\begin{itemize} 
	\item  \texttt{constructor(strategy:CleanerStrategy): void} costruisce il \texttt{DataCleaner} con una specifica strategia di pulizia;
	\item \texttt{setStrategy(in strategy:CleanerStrategy): void} il quale riceve in input un oggetto che implementa l'interfaccia \texttt{CleanerStrategy} e lo assegna al riferimento proprio dell'oggetto;

	\item \texttt{cleanData(in elasticsearchIndicesResults:Array<ElasticsearchIndexResult>): Array<RawTrace>} il quale prima rimuove i metadati (chiamando il suo metodo \texttt{removeMetaDataFromIndices}) di Elasticsearch dal parametro in input e successivamente chiama l'implementazione di \texttt{clean} della strategia che possiede al suo interno.  Tale strategia è incaricata di raffinare ulteriormente i dati per la costruzione della Mappa Topologica oppure per la costruzione dello Stack Trace tramite il metodo \texttt{clean};

	\item \texttt{removeMetadataFromIndices(elasticsearchIndicesResults:Array<ElasticsearchIndexResult>): Array<RawTrace>}  si occupa di rimuovere i metadati di Elasticsearch da un Array di indici invocando il metodo \texttt{removeMetaDataFromIndex} su ognuno dei singoli indici. Viene ritornato un array di \texttt{RawTrace} che rappresenta le traces lette dai vari indici in un singolo array;

	\item \texttt{removeMetadataFromIndex(elasticsearchIndexResult:ElasticsearchIndexResult): Array<RawTrace>} il quale si occupa di rimuovere i metadati di Elasticsearch che sono presenti in un \emph{singolo} indice. Questo metodo ritorna un array contenente i documenti JSON così come essi sono stati inseriti dall'applicazione di monitoring negli indici di Elasticsearch. Elasticsearch, infatti, immagazzina i documenti JSON nel campo \texttt{\_source} il quale è inserito in oggetti che contengono altri dati ed informazioni amministrative. Il metodo si occupa di rimuovere questi dati e restituire un array contenente i documenti di un indice;
\end{itemize}

Risulta chiaro che esiste dunque una dipendenza fra la struttura dei dati che vengono restituiti da ElasticSearch, la classe \texttt{DataCleaner} e le classi che implementano \texttt{CleanerStrategy}. Tuttavia, i client di \texttt{DataCleaner} potranno avere accesso diretto ai dati di ElasticSearch, privi di dati amministrativi.
	
\paragraph{CleanerStrategy}\Spazio
\label{sec:CleanerStrategy}
\texttt{CleanerStrategy} è l'interfaccia che verrà implementata per realizzare una pulizia più raffinata dei dati di un \texttt{DataCleaner}. Espone il metodo \texttt{clean(data)}, che avrà comportamenti diversi a seconda della strategia che si desidera applicare. Dato che il prodotto è implementato in Javascript il costrutto interfaccia non è presente nel linguaggio. Per ovviare a questa mancanza tutte le classi che dovrebbero implementare questa interfaccia \emph{devono} possedere un metodo \texttt{clean(data)} che funzioni come ci si aspetti. Questa interfaccia implementa il design pattern \emph{Strategy}, come descritto in \textcolor{red}{QUI DAI IL RIFERIMENTO, LO DEVI ANCORA SCRIVERE}.
	
	
\paragraph{GraphCleanerStrategy}\Spazio
\label{sec:GraphCleaner}

\texttt{GraphCleaner} è l'implementazione di \texttt{CleanerStrategy} utilizzata da \texttt{DataCleaner} per pulire i dati che dovranno essere utilizzati per la costruzione del grafo rappresentante la mappa topologica. Esso dispone dei seguenti metodi:
\begin{itemize}
	\item \texttt{clean(traces:Array<RawTrace>): Array<RawTrace>} che riceve come input un Array di indici che contengono al loro interno i documenti JSON sotto forma di Array. Tale metodo agisce scorrendo l'array rimuovendo i documenti inutili alla costruzione del grafo.
\end{itemize}

\paragraph{StackCleanerStrategy} \Spazio
\label{sec:StackCleaner}

\texttt{StackCleaner} è l'implementazione di \texttt{CleanerStrategy} utilizzata da \texttt{DataCleaner} per pulire i dati che dovranno essere utilizzati durante la costruzione della stack trace. Essa dispone del metodo
\begin{itemize}
	\item \texttt{clean(traces:Array<RawTrace>): Array<RawTrace>} il quale scorre tutti i documenti JSON disponibili ed ha come output l'insieme di documenti che rappresentano chiamate HTTP, JDBC oppure Pageload, ovvero tutte e sole quelle necessarie per la costruzione dei dati strutturati della stack trace.
\end{itemize}



\paragraph{Graph}
\label{sec:Graph}
	\texttt{Graph} è il tipo di dato che rappresenta il grafo della mappa topologica dell'applicazione. Esso contiene due Array, uno chiamato \texttt{Nodes} e uno \texttt{Links}.
	\label{sec:Nodes}
	L'Array di \texttt{Nodes} contiene degli oggetti, detti \texttt{Node} rappresentanti i nodi della mappa che contengono i seguenti campi:
	\begin{itemize}
		\item \texttt{id}, contiene un id numerico intero univoco che individua un nodo;
		\item \texttt{name}, contiene il nome del nodo che verrà visualizzato nella mappa. Anche esso è univoco;
		\item \texttt{type}, contiene la tipologia del nodo, che potrà essere ad esempio Server oppure Database.
	\end{itemize}
	\label{sec:Links}
	L'Array di \texttt{Links} contiene oggetti che rappresentano la di comunicazione tra due \texttt{Node} e contengono i seguenti campi:
	\begin{itemize}
		\item{\texttt{source}, che rappresenta il nodo che ha effettuato la chiamata sotto forma di id numerico intero;}
		\item{\texttt{target}, che rappresenta il nodo che ha ricevuto la chiamata sotto forma di id numerico intero;}
		\item{\texttt{type}, che può essere "Database" o "HTTP", per rappresentare i diversi tipi di comunicazione;}
		\item{\texttt{average\_response\_time}, che contiene il tempo medio di risposta di tutte le trace tra source e target;}
		\item{\texttt{number\_of\_requests}, che serve per mantenere consistente il campo \texttt{average\_response\_time}.}
	\end{itemize}
	Esso è individuato univocamente dalla coppia di campi \texttt{source} e \texttt{target}.
	Il tipo di dato \texttt{Graph} contiene i seguenti metodi:
	\begin{itemize}
		\item{\texttt{checkIfNodeIsNotPresent(Object)} controlla se l'oggetto che gli viene passato, che è assunto essere un \texttt{Node} risulta già essere presente nell'Array \texttt{Nodes} attraverso un controllo sul campo \texttt{name}; }
		\item{\texttt{checkIfLinkIsNotPresent(Object)} controlla se l'oggetto che gli viene passato, che è assunto essere un \texttt{Link} risulta già essere presente nell'Array \texttt{Links} attraverso un controllo sulla coppia di campi \texttt{source} e \texttt{target}; }
		\item{\texttt{getIdOfNode(String)} scorre l'Array \texttt{Nodes} finchè non trova un \texttt{Node} con campo \texttt{name} uguale a quello passatogli come parametro; }
		\item{\texttt{getIndexOfSameLink(Object)} scorre l'Array \texttt{Links} finchè non trova un \texttt{Link} con campi \texttt{source} e \texttt{target} uguali a a quelli dell'oggetto che gli è passato, che è assunto essere un \texttt{Link};}
		\item{\texttt{addNode(Object)} aggiunge l'oggetto passatogli, che è assunto essere un \texttt{Node} all'Array \texttt{Nodes}, se esso non è già presente;}
		\item{\texttt{addLink(Object)} aggiunge l'oggetto passatogli, che è assunto essere un \texttt{Link} all'Array \texttt{Links}, se esso non è già presente.}					
	\end{itemize}

\paragraph{GraphBuilder}
\label{sec:GraphBuilder}
	\texttt{GraphBuilder} è il componente che si occupa di costruire il grafo di nodi e archi rappresentanti la mappa topologica dell'applicazione, secondo la struttura descritta in \ref{sec:Graph}. Esso verrà successivamente visualizzato utilizzando la libreria D3.js. \texttt{GraphBuilder} viene istanziato con un oggetto \texttt{DataReader} per poter ottenere i dati di cui ha bisogno.
	I metodi della classe sono:
	\begin{itemize}
		\item{\texttt{setDataReader(DataReader)}, il quale permette di impostare una diversa fonte di dati; }
		
		\item{\texttt{buildNodes(Array)} il quale riceve un Array di richieste HTTP e a Database ed ha costruisce i nodi dell'oggetto \texttt{Graph}, rispettando la struttura descritta in \ref{sec:Nodes}. Esso agisce in maniera analoga a \texttt{buildLinks(Array,Array)} scorrendo tutte le richieste che gli vengono passate e costruendo un candidato ad entrare nell'insieme dei \texttt{Nodes}. L'unicità del candidato viene garantita dal metodo \texttt{checkIfNodeIsNotPresent(Array,Object)};}
				
		\item{\texttt{buildLinks(Array)} il quale riceve in input l'Array di \texttt{Nodes} dell'applicazione e un Array di richieste HTTP e a Database e costruisce nell'istanza di \texttt{Graph} i Links che rispettano la struttura definita in \ref{sec:Links}. Il metodo scorre tutti i documenti JSON che gli vengono passati come Array e per ogni documento, che contiene la trace di una comunicazione tra due componenti dell'applicazione costruisce un oggetto candidato ad entrare nell'insieme. Il fatto che tale \texttt{Link} candidato non sia già presente nell'insieme viene garantito dal metodo \texttt{checkIfLinkIsNotPresent(Array, Object)} che controlla che nell'Array dei links non sia già presente un collegamento con gli stessi campi \texttt{source} e \texttt{target} del candidato; }

		\item{\texttt{getGraph()} il quale ritorna un oggetto di tipo \texttt{Graph} secondo la struttura definita in \ref{sec:Graph} costruendo i due Array attribuendo loro l'output dei metodi \texttt{buildNodes()} e \texttt{buildLinks()}.  }
	\end{itemize}




















	\paragraph{Trace}
\label{sec:Trace}
\textcolor{red}{QUI WIPA}
\texttt{Trace} rappresenta una richiesta che viene fatta al sistema contenente solo i dati veramente utili alla successiva costruzione della Stack Trace. Questi sono:
\begin{itemize}
	\item \texttt{type}: che riconduce ad una delle tipologie di trace;
	\item \texttt{trace\_id}: numero univoco che permette l'associazione tra trace HTTP, JDBC e Pageload;
	\item \texttt{name}: identificazione generale della richiesta;
	\item \texttt{call\_tree}: array contenente gerarchicamente la lista dei metodi invocati dall'applicazione per compiere la richiesta;
	\item \texttt{duration}: tempo impiegato dal sistema per lo svolgimento della richiesta, compreso il tempo dei metodi, delle query e del caricamento della pagina;
	\item \texttt{timestamp}: data e ora dello svolgimento della richiesta;
	\item \texttt{error}: presenza o meno di errori;
	\item \texttt{status\_code}: tipologia di errore;
	\item \texttt{DBrequest}: (opzionale) array contenente le query con cui è stato interrogato il database.
\end{itemize}
 La tipologia delle trace viene identificata tramite i metodi \texttt{checkPageload()} e \texttt{checkQueries()} e queste possono essere:
	\begin{itemize}
	\item HTTP: insieme di metodi scatenati da un evento generico;
	\item HTTP + Pageload: insieme di metodi scatenati dal caricamento di una pagina web;
	\item HTTP + JDBC + Pageload: insieme di eventi che portano all'interrogazione di un database e che provocano un successivo caricamento di una pagina.
	\end{itemize}
Inoltre i dati vengono calcolati in base alla tipologia della trace tramite  \texttt{setPageload()} e  \texttt{setQueries()} che utilizzano determinate funzioni per la manipolazione dei dati:
\begin{itemize}
	\item Per il campo \texttt{call\_tree} avviene un'accurata manipolazione attraverso il metodo \texttt{build\_tree()} per passare da un tipo stringa ad una gerarchia di oggetti.
	Al metodo \texttt{build\_tree()} viene passato il dato sotto forma di stringa in cui, attraverso l'ASCII art, viene rappresentato un albero dei metodi che vengono invocati con delle informazioni particolari per ognuno di essi. Questo utilizza a sua volta il metodo \texttt{more\_data()} per costruire un array bidimensionale che elenca tutti i metodi e per ognuno ne specifica i dati aggiuntivi rappresentati dall'ASCII art (come total time e self time), il grado di indentazione nell'albero, il numero di metodi figli e il numero di metodi discendenti da esso. \\ Da questa struttura il metodo \texttt{tableToTree()} ricava un oggetto con innestati altri oggetti rappresentanti i metodi e i propri dati strutturati in maniera da rispecchiare i legami di parentela dell'ASCII art.
	\item 	Il metodo \texttt{change\_duration} permette di far variare il tempo di esecuzione delle richieste in base alla tipologia in quanto per ognuna bisognerà considerare più parametri per calcolare i millisecondi impiegati.
\end{itemize}

\paragraph{StackBuilder}
\label{sec:StackBuilder}

\texttt{StackBuilder} si occupa della creazione della struttura della stack trace per la sua successiva visualizzazione.
Esso viene istanziato con un oggetto \texttt{DataReader} per poter ottenere i dati.
Il metodo principale è \texttt{getStack()} che controlla tutte le chiamate HTTP ricevute dal \texttt{DataReader} e pulite dal \texttt{DataCleaner}, creato e istanziato con strategia \texttt{StackCleaner}, creando delle \texttt{Trace} da ogni richiesta con dati grezzi.