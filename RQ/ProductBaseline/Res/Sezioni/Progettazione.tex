
\section{Premessa}
\label{sec:premessa}
Prima di proseguire con la lettura del documento è bene puntualizzare alcune cose.
Innanzitutto va ricordato che il nostro prodotto consta di due plugin \emph{distinti} che svolgono funzioni diverse, ma che comunque interagiscono con i dati presenti sul server ElasticSearch in modo simile. Per semplificare la rappresentazione grafica e per evitare ripetizioni essi sono stati, in questa fase di progettazione, considerati come facenti parte dello stesso sistema. Questo è reso possibile dall'alta sovrapposizione di componenti necessari al loro funzionamento. Tuttavia, nella fase di produzione, essi verranno completamente disaccoppiati e saranno presentati come due plugin distinti che avranno delle parti in comune, che per come è progettato Kibana, dovranno essere per necessariamente ripetute. Va segnalato che il linguaggio di programmazione da noi usato, JavaScript, è \emph{weakly typed}. Questo comporta. ad esempio, poca precisione per quanto riguarda il tipaggio dei parametri passati ai metodi. Per questo, soprattutto nella fase di rappresentazione grafica delle classi del sistema, ci terremmo talvolta vaghi sul tipo dei parametri passati e ritornati da metodi. Tali imprecisioni verranno tuttavia chiarite nella parte di testo che riguarda ogni specifico componente. Inoltre, essendo ElasticSearch un database non relazionale i dati da esso ritornati non hanno una struttura fissa, dunque molte volte verranno descritti semplicemente come \texttt{Array} o \texttt{Object}, chiarendo nella descrizione testuale di che tipo di array o oggetto si tratta.

\section{Componenti}

Ogni plugin per Kibana è articolato in due sezioni principali: lato client e lato server. Si presenta di seguito il dettaglio di ciascuna sezione.
