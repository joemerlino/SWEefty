\section{Componenti}
\subsection{Premessa}
Il codice prodotto è stato scritto in Javascript \textcolor{red}{qui versione!}, quindi molti concetti quali classi ed interfacce non sono presenti all'interno del linguaggio. Per produrre un diagramma delle classi dunque sono state considerate \emph{ classi } sia oggetti Javascript, sia funzioni. Per quanto riguarda le interfacce che sono presenti nel diagramma delle classi \ref{diagrammaClassi}, nel codice non sono effettivamente presenti tali interfacce, ma tutte le classi che implementano tale interfaccia \emph{devono} possedere i metodi esposti da tale interfaccia.

\begin{figure}[H]
    \label{diagrammaClassi}
    \centering
    \includegraphics[width=1\textwidth]{Images/logo.jpg}
    \caption{Diagramma delle classi dell'applicazione}
\end{figure}

\subsection{DataReader}
\subsection{DataCleaner}
\subsection{CleanerStrategy}
\subsection{GraphCleaner}
\subsection{StackCleaner}
\subsection{StackBuilder}
\subsection{GraphBuilder}

\section{Interazioni fra componenti}

Qui diagrammi di sequenza

\section{Tracciamento dei requisiti}

Qui swego 
