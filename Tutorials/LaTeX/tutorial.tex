% questo è un commento

% questè la classe del documento: caratteri da 12 pt su un a4, article è per indicare che è un articolo (le altre che esistono sono peresempio book per i libri)
\documentclass[a4paper,12pt]{article}

% i seguenti sono i package, pensali come degli "include" che aumentano la tua capacità espressiva (paragone con html: come se aggiungessero nuovi tag semantici)
\usepackage{color}
\usepackage{graphicx}

% inizio di quello che verrà messo a video
\begin{document}

%
\title{\LaTeX    tutorial}
\author{Francesco Parolini}
\date{\today}
\maketitle

% indice
\tableofcontents

%lista delle figure
\listoffigures

% new page forzato
\newpage

\section{Introduzione}
Per imparare le basi di \LaTeX vi potete rifare a questo documento, leggendo \emph{prima} il sorgente e poi guardando il pdf vero e proprio. Come editor di \LaTeX vi consiglio Gummi.

\section{Sezioni}
Qui sopra apro una nuova sezione che verrà numerata in modo automatico, il testo al suo interno non deve avere dei tag associati. Posso creare sottosezioni e sotto-sottosezioni:

\subsection{Sottosezione}
ciauz

\subsubsection{Sottosottosezione}
ciauzzone

\section{Decorazione di testo}
Ecco un pò di effetti {\color{red}\textit{cool}} per scrivere roba:\\ %il \\ va a capo
	\emph{words in italics}\\
	\textsl{words slanted}\\
	\textsc{words in smallcaps}\\
	\textbf{words in bold}\\
	\texttt{words in teletype}\\
	\textsf{sans serif words}\\
	\textrm{roman words}\\
	\underline{underlined words}\\

\section{Tabelle}

Poi io faccio anche le tabelle sono un talento di \LaTeX \\
	\begin{tabular} % corrisponde a <table> in html
	{|l|l|} % colonne: l sta ad indicare che il testo deve essere allineato a sinistra, r significherebbe right
		\hline % bordo orizzontale che non viene automaticamente aggiunto
		\textbf{header1} & % & va alla cella successiva nella tabella
			\textbf{header2}\\
		\hline
		cella1 & cella3 \\
		\hline
		cella3 & cella4\\
		\hline
	\end{tabular}
\\ \\
Ora una senza bordi \\

	\begin{tabular}{l|r|r}
		Item & Quantity & Price(\$) \\
		\hline
		Nails & 500 & 4.00 \\
		Ani & 1500 & 1.05 \\
		Pedro & 1 & $\infty$ \\ 
	\end{tabular}
\\ 

\section{Immagini}

Ora immagini virili \\
	\begin{figure}[h]
	\label{figuragattino} % questo da un' "ancora" a un qualsiasi elemento: come il tag id. Tale elemento potrà essere referenziato automaticamente come vedi a riga 84
	\centering % centra
	\includegraphics[width=1\textwidth]{gattino.jpg}
	\caption{Questo e' un gattino miao miao} % descrizione
	\end{figure}
\\

\section{Matematica}

Invece adesso scrivo un poò di roba matematica: 1+1=3 $1+1=3$ $$1+1=3$$ % cerca su google: math mode, è quella che ti fa scrivere simboli e "roba" di matematica. si apre e si chiude con i dollari
	\begin{equation}
		\label{miaequazione}
		1+1=3
	\end{equation}
E mi riferisco alla mia equazione: \ref{miaequazione}
	\begin{eqnarray*} %devi mettere il * per non fare numerare le righe!
		23a & = & b + c \\
		& = & y - z
	\end{eqnarray*}
E adesso simboli\\
	$n^2$\\
	$n_2$\\
	$n_2^2$\\
	$n^{2^{2^2}}$\\
	$n_{2_{a-1}}$\\
	$$\frac{y}{\frac{3}{x}+b}$$\\
	$$\sqrt{y^2}$$\\
	$$\sqrt[x]{y^2}$$\\
	$$ \sum_{i=1}^{n} n^i $$\\
	$$\int_{-\infty}^{\pi+1}sin(x)dx$$\\
	$$\alpha$$
	$\beta$ 
	$\delta, \Delta$
	$\theta, \Theta$ 
	$\mu$
	$\pi, \Pi$
	$\sigma, \Sigma$
	$\phi, \Phi$ 
	$\psi, \Psi$ 
	$\omega, \Omega$ 

\section{Liste}
Eccone alcune

	\begin{enumerate}
	\item First thing
	\item Second thing ciauzzione
		\begin{itemize}
		\item A sub-thing
		\item Another sub-thing
		\end{itemize}
	\item Third thing
	\end{enumerate}
	
	Ora un dizionario
	\begin{description}
		\item [Termine1] descrizione
		\item [Termine2] altra descrizione
	\end{description}

\end{document}
